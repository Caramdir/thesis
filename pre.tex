\chapter{Prerequisites}

\section{Equivariant quasi-coherent sheaves}
\section{Local cohomology}
\section{Perverse coherent sheaves}

By a \emph{perversity} we mean a function $p\colon \{0,\dotsc,\dim X\} → ℤ$.
For $x ∈ \Xtop$ we abuse notation and set $p(x) = p(\dim x)$.
Then $p\colon \Xtop → ℤ$ is a perversity function in the sense of~\cite{Bezrukavnikov:arXiv:PerverseCoherentSheaves}.
Note that we insist that $p(x)$ only depends on the dimension of $\overline x$.
A perversity is called \emph{monotone} if it is decreasing and \emph{comonotone} if the \emph{dual perversity} $\overline p(n) = -n - p(n)$ is decreasing.
It is \emph{strictly monotone} (resp.~\emph{strictly comonotone}) if for all $x,y ∈ \Xtop$ with $\dim x < \dim y$ one has $p(x) > p(y)$ (resp.~$\overline p(x) > \overline p(y)$).
Note that a strictly monotone perversity is not necessarily strictly decreasing (e.g.~if $X$ only has even-dimensional $G$-orbits).

Recall that if $p$ is a monotone and comonotone perversity then Bezrukavnikov (following Deligne) defines a t-structure on $D_c^b(X)^G$ by taking the following full subcategories \cite{Bezrukavnikov:arXiv:PerverseCoherentSheaves,ArinkinBezrukavnikov:2010:PerverseCoherentSheaves}:
\begin{align*}
    \perv[p] D^{≤0}(X)^G & = 
    \bigl\{ \sheaf F ∈ D_c^b(X)^G : \mathbf ι_x^*\sheaf F ∈ D^{≤p(x)}(\catModules{\O_x}) \text{ for all $x ∈ \Xtop$}\bigr\}, \\
    \perv[p] D^{≥0}(X)^G & = 
    \bigl\{ \sheaf F ∈ D_c^b(X)^G : \mathbf ι_x^!\sheaf F ∈ D^{≥p(x)}(\catModules{\O_x}) \text{ for all $x ∈ \Xtop$}\bigr\}.
\end{align*}
The heart of this t-structure is called the category of \emph{perverse sheaves} with respect to the perversity $p$.
