\chapter{Prerequisites}

\section{Equivariant quasi-coherent sheaves}

Definition. Do we need anything else?

Coherent $!$ and $*$-restriction.

\section{Local cohomology and dualizing complexes}

Definition. 

All the theorems from [BS], [SGA2] and [H] that we need.

Comparison of local cohomology with qc-restriction functors.

\begin{Lem}
    \label{lem:pre:stalk-and-costalk-vanishing}%
    Let $\sheaf F$ be a coherent sheaf on $X$ and let $x$ be a closed point of $X$.
    Then $\mathbf ι_z^* \dualize \sheaf F ∈ D^{\ge 0}(\O_x)$ if and only if $\mathbf ι_z^! \sheaf F ∈ D^{\le 0}(\O_x)$.
\end{Lem}

\begin{proof}
    Similar to \cite[Lemma~3.3(a)]{ArinkinBezrukavnikov:2010:PerverseCoherentSheaves}\todo{Add this}.
\end{proof}

\begin{Lem}
    \label{lem:pre:top-and-qc-restriction-vanishing}%
    Let $\sheaf F$ be a coherent sheaf on $X$ and let $Z$ be a closed subvariety of $X$.
    Let $n$ be an integer.
    Then $\mathbf ι_Z^! \sheaf F ∈ D^{\ge 0}(Z)$ if and only if $ι_Z^! \sheaf F ∈ D^{\le 0}(Z)$.
\end{Lem}

\begin{proof}
    \cite[Proposition~VII.1.2]{SGA2}\todo{Give explicit statement.}.
\end{proof}

\section{Perverse coherent sheaves}

By a \emph{perversity} we mean a function $p\colon \{0,\dotsc,\dim X\} → ℤ$.
For $x ∈ \Xtop$ we abuse notation and set $p(x) = p(\dim x)$.
Then $p\colon \Xtop → ℤ$ is a perversity function in the sense of~\cite{Bezrukavnikov:arXiv:PerverseCoherentSheaves}.
Note that we insist that $p(x)$ only depends on the dimension of $\overline x$.
A perversity is called \emph{monotone} if it is decreasing and \emph{comonotone} if the \emph{dual perversity} $\overline p(n) = -n - p(n)$ is decreasing.
It is \emph{strictly monotone} (resp.~\emph{strictly comonotone}) if for all $x,y ∈ \Xtop$ with $\dim x < \dim y$ one has $p(x) > p(y)$ (resp.~$\overline p(x) > \overline p(y)$).
Note that a strictly monotone perversity is not necessarily strictly decreasing (e.g.~if $X$ only has even-dimensional $G$-orbits).

Recall that if $p$ is a monotone and comonotone perversity then Bezrukavnikov (following Deligne) defines a t-structure on $D_c^b(X)^G$ by taking the following full subcategories \cite{Bezrukavnikov:arXiv:PerverseCoherentSheaves,ArinkinBezrukavnikov:2010:PerverseCoherentSheaves}:
\begin{align*}
    \perv[p] D^{≤0}(X)^G & = 
    \bigl\{ \sheaf F ∈ D_c^b(X)^G : \mathbf ι_x^*\sheaf F ∈ D^{≤p(x)}(\catModules{\O_x}) \text{ for all $x ∈ \Xtop$}\bigr\}, \\
    \perv[p] D^{≥0}(X)^G & = 
    \bigl\{ \sheaf F ∈ D_c^b(X)^G : \mathbf ι_x^!\sheaf F ∈ D^{≥p(x)}(\catModules{\O_x}) \text{ for all $x ∈ \Xtop$}\bigr\}.
\end{align*}
The heart of this t-structure is called the category of \emph{perverse sheaves} with respect to the perversity $p$.
