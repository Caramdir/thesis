\chapter{Introduction}

[TODO: Introduction to introduction]

\section{Microlocal theory of perverse coherent sheaves}

This project is motivated by two theories that came to recent prominence in algebraic geometry and in particular geometric representation theory.
\begin{itemize}
    \item The notion of perverse coherent sheaves, independently introduced by Bezrukavnikov (following Deligne) \cite{ArinkinBezrukavnikov:2010:PerverseCoherentSheaves}, Kashiwara \cite{Kashiwara:2004:tStructureOnHolonomicDModuleCoherentOModules}, Gabber \cite{Gabber:2004:NotesOnSomeTStructures} and Bridgeland \cite{Bridgeland:2006:DerivedCategoriesOfCoherentSheaves}.
    \item A theory of \enquote{singular support} for coherent sheaves, worked out by Arinkin and Gaitsgory for their formulation of the Geometric Langlands Conjecture \cite{ArinkinGaitsgory:2015:SingularSupport}, based on earlier work by Benson, Iyengar and Krause \cite{BensonIyengarKrause:2008:LocalCohomologyAndSupportForTriangulatedCategories}.
\end{itemize}
Removing the word \enquote{coherent}, these two theories are by now classical and indispensable tools in the study of constructible sheaves.
They can also be beautifully combined to elucidate the microlocal nature of perverse sheaves \cite{KashiwaraSchapira:1994:SheavesOnManifolds}.
Thus the question naturally arises whether the coherent versions of these theories can be combined too.

Coming form representation theory, we will be primarily concerned with the definition of perverse coherent sheaves by Bezrukavnikov, which we will recall below.
However we should note from the outset that the Arinkin--Gaitsgory theory of singular support does not interact well with these perverse coherent sheaves.
For example the perverse t-structure on $\SL2$-equivariant coherent sheaves on $\as2$ is non-trivial, but the singular support of coherent sheaves on either $\as2$ or the stack $\as2/\SL2$ is always trivial.
Thus we have to understand the microlocal nature of perverse coherent sheaves in a different way.

\subsection*{Perverse (constructible) sheaves}

Before discussing our results, let us briefly review the theory of perverse sheaves in the setting that is closed to the coherent analogue that we will use.
For readability, we will restrict to the complex case and the middle perversity.

Thus we le $X$ be a complex variety and fix a stratification $\setset S$ of $X$ be smooth complex subvarieties.
Attached to this setup we have the bounded derived category of $\dconstrb[\setset S]{X}$ of $\setset S$-constructible sheaves, i.e.~the full subcategory of the category of constructible sheaves on $X$ of sheaves which are local systems along each stratum.
The \emph{perverse t-structure} on this category is then given by the two full subcateogries
\begin{equation}
    \begin{aligned}
        \label{eq:intro:pc:constructible_t-structure}
        \dpconstr{\setset S}{\le 0}{X} &=
        \bigl\{ \sheaf F ∈ \dconstrb[\setset S]{X} : i_S^* \sheaf F ∈ \dcat{\le -\frac12\dim_ℝ S}{S} \text{ for all } S ∈ \setset S\bigr\}, \\
        \dpconstr{\setset S}{\ge 0}{X} &=
        \bigl\{ \sheaf F ∈ \dconstrb[\setset S]{X} : i_S^! \sheaf F ∈ \dcat{\ge -\frac12\dim_ℝ S}{S} \text{ for all } S ∈ \setset S\bigr\}, \\
    \end{aligned}
\end{equation}
where $i_S\colon S \hookrightarrow X$ is the inclusion.
That this is indeed a t-structure was proved by Beilinson, Bernstein and Deligne in \cite{BeilinsonBernsteinDeligne:1982:FaisceauxPervers}.
In particular this means that the category
\[
    \PervConstr[\setset S]{X} = \dpconstr{\setset S}{\le 0}{X} ∩ \dpconstr{\setset S}{\ge 0}{X}
\]
is Abelian.
It is called the category of \emph{(middle) perverse sheaves} on $X$ with respect to $\setset S$.
This category both intimately connected to the singularities of $X$ (via intersection cohomology) and, if $X$ is smooth, the category of D-modules on $X$ (via the Riemann-Hilbert correspondence).
It has many nice properties, of which we want to mention the following two which are of particular importance in the coherent analogue.
\begin{itemize}
    \item The Verdier duality functor on $\dconstrb[\setset S]{X}$ restricts to an involution of $\PervConstr[\setset S]{X}$.
    \item The simple objects of $\PervConstr[\setset S]{X}$ are in bijection with pairs $(S, \sheaf L)$ consisting of a stratum $S ∈ \setset S$ and an irreducible local system on $\sheaf L$ on $S$.
\end{itemize}

\subsection*{Perverse coherent sheaves}

With some modification the definition of perverse sheaves can be translated to coherent sheaves.
We will again only discuss the middle perversity here.
The general definition will be reviewed in Section~\ref{sec:pc:pre:pc}.

We let $X$ be a scheme of finite type over a field $k$.
We would like to pick a stratification of $X$ and consider the category of coherent sheaves whose restriction to each stratum is a vector bundle.
Unfortunately this is not a triangulated subcategory of $\dcohb{X}$.
For example for a general function $f$ the cone of $\O_X \xrightarrow{\cdot f} \O_X$ will not be a vector bundle on the open stratum.

The solution to this problem is to introduce a group action replace \enquote{smoothness along a stratification} by \enquote{equivariantness}.
Thus we let $G$ be an affine algebraic group over $k$ acting on $X$ and consider the bounded derived category $\dcohbG{X} = \dcohb{X/G}$ of $G$-equivariant coherent sheaves on $X$.
Instead of the strata $S ∈ \setset S$ we will look at the set $\Xtop$ of generic points of $G$-equivariant subschemes of $X$.

In the constructible setting we used required the stratification to be complex, so that we could have the (real) dimension.
He we will require directly that all $G$-orbits are even-dimensional.
The main example of this situation is the nilpotent cone of a semi-simple algebraic group with the adjoint action.

The only piece remaining to translate from the constructible setting are the restriction functors.
For this it turns out that the $\O$-module functors do not give the correct definition and we have to use the $k$-module functors instead.
Thus, if $ι_x\colon \{x\} → X$ is the inclusion of $x ∈ \Xtop$ and $\sheaf F ∈ \dcohbG{X}$, we let $\mathbf ι_x^* \sheaf F = \sheaf F_x$ be the (derived) functor of talking stalks.
Similarly, we let $\mathbf ι_x^!\sheaf F = \mathbf ι_x^* \lc{\bar{\{x\}}} \sheaf F$ be the derived functor of local cohomology.
Thus the definition~\eqref{eq:intro:pc:constructible_t-structure} becomes the two full subcategories
\begin{align*}
    \dpc{≤0}{X} & =
    \bigl\{ \sheaf F ∈ \dcohb{X} : \mathbf ι_x^*\sheaf F ∈ \dcat{≤ -\frac12 \dim x}{\O_x} \text{ for all $x ∈ \Xtop$}\bigr\}, \\
    \dpc{≥0}{X} & =
    \bigl\{ \sheaf F ∈ \dcohb{X} : \mathbf ι_x^!\sheaf F ∈ \dcat{≥ -\frac12 \dim x}{\O_x} \text{ for all $x ∈ \Xtop$}\bigr\},
\end{align*}
and this is indeed a t-structure \cite{ArinkinBezrukavnikov:2010:PerverseCoherentSheaves}.
The heart of this t-structure is called the category of \emph{(middle) perverse sheaves} on $X$.
It is again an Abelian category with lots of interesting features.
In particular we have the same properties as above:
\begin{itemize}
    \item Grothendieck-Serre duality is an involution of the category of perverse coherent sheaves.
    \item For sufficiently nice $G$-actions the simple objects are indexed by pairs $(O, \sheaf V)$, where $O$ is a G-orbit and $\sheaf V$ is an irreducible $G$-equivariant vector bundle on $O$ (or equivalently an irreducible representation of the stabilizer of $G$ on $O$).
\end{itemize}
Bezrukavnikov used the second property in \cite{Bezrukavnikov:2003:QuasiExceptionalSets} to establish a bijection between pairs $(O, V)$ of a nilpotent orbit $O$ and an irreducible representation $V$ of the stabilizer of $G$ on $O$, and the set $Λ^+$ of dominant weight of $G$, thus proving a conjecture of Lusztig.

\subsection*{Microlocal theory (constructible version)}

On the first view the definition of perverse sheaves is very mysterious.
Why would the category defined this way be important and have nice features?
One answer to this question was proposed by Kashiwara and Schapira \cite{KashiwaraSchapira:1994:SheavesOnManifolds} in their microlocal viewpoint.

One way to see the microlocal nature of perverse sheaves via the vanishing cycles functor\todo{references?}.
For this let $f$ be a (local) holomorphic function on $X$.
Then the vanishing cycles $φ_f\colon \dconstrb{X} → \dconstrb{f^{-1}(0)}$ restrict to a functor on the corresponding categories of perverse sheaves
\[
    \pervVC{f}\colon \PervConstr{X} → \PervConstr{f^{-1}(0)}.
\]
If we in addition let $f$ be a stratified Morse function with critical point $x$, then taking corresponding \emph{microlocal stalks}
\[
    \left(\pervVC{f}({-})\right)_x \colon \dconstrb[\setset S]{X} → \dcat{}{ℂ}
\]
sends perverse sheaves to vector spaces concentrated in degree $0$.
In fact, this property characterizes perverse sheaves among all $\setset S$-constructible sheaves on $X$ \cite{Jin:arXiv:HolomorphicLagrangianBranesCorrespondToPerverseSheaves}.

In a further reformulation, we can take local cohomology along the unstable manifold $Z$ of $\Re f$ and require that this is concentrated in degree $0$.
We note that
\[
    \dim_ℝ Z = \frac12 \dim_ℝ X.
\]

\subsection*{Microlocal theory (coherent version)}

This last observation can be translated to the world of coherent sheaves.
For this purpose we will define \emph{measuring subvarieties} of $X$.
Roughly speaking, these are subvarieties of $X$ which intersects each $G$-orbit in a half dimensional subvariety.
We will give the precise definition in Chapter~\ref{ch:pc:measuring}, where we will also prove the following analogue of the above characterization of perverse (constructible) sheaves.

\begin{Thm}
    A sheaf $\sheaf F ∈ \dcohbG{X}$ is perverse if and only if $\lc{Z}\sheaf F$ is concentrated in cohomological degree $0$ for sufficiently many measuring subvarieties $Z$ of $X$.
\end{Thm}

The exact statement and some variants will be given in Theorem~\ref{thm:main}.

\section{Hochschild cohomology of categories of D-modules}

Given a manifold $X$ and a category of sheaves on $X$, microlocal geometry asks whether these sheaves can be localized not just on $X$ but also with respect to codimension, i.e.~on the cotangent space $T^*X$.
For example, for constructible sheaves this leads to the notion of microsupport discussed in detail in \cite{KashiwaraSchapira:1994:SheavesOnManifolds}.
More generally, given a category of sheaves on a space $X$, we can ask whether it is possible to localize them on some space that is strictly larger than $X$ itself.

Even more generally one can ask the following question: Given a $k$-linear category $\cat C$, can one find a space over which $\cat C$ localizes?
For complete compactly generated triangular categories one answer is provided by \cite{BensonIyengarKrause:2008:LocalCohomologyAndSupportForTriangulatedCategories}:
To each map from a graded-commutative ring $R$ to the center of $\cat C$ the authors associate the \emph{triangulated support} functor $\supp_R$, assigning to each object $A ∈ \cat C$ a subset $\supp_R A \subseteq \Spec R$.
This construction can be used to unify various theories of support in different areas of mathematics (though it does not yield the microlocal support of constructible sheaves).

We are led to consider the universal algebra acting on the category with this construction, i.e.~the \emph{Hochschild cohomology} of $\cat C$.
For a complete (pre-triangulated) dg category $\cat C$ the Hochschild cohomology is the dg algebra of derived endomorphisms of the identity functor $\id[\cat C]$:
\[
    \HCoh(\cat C)
    = \operatorname{\mathbf{R}Hom}(\id[\cat C], \id[\cat C])
    = \Hom_{\cat{Funct}_{\mathrm{cont}}(\cat C, \cat C)}(\id[\cat C], \id[\cat C]).
\]
In particular the ring $R = \bigoplus \operatorname{HH}^{2n}(\HCoh(\cat C)r$ is commutative and hence one can define for each $A ∈ \cat C$ the support $\supp_R A$ as a subset of $\Spec R$.
Hence understanding the Hochschild cohomology of a dg category can be an important step to understanding the category itself.

This construction, applied to the category of (ind-)coherent sheaves on a (quasismooth, dg-) scheme yields the singular support of coherent sheaves which already served as the motivation for the first part of this thesis.
More concretely, Arinkin and Gaitsgory used singular support for the category $\catIndCoh{\LocSys_G}$ in their formulation of the geometric Langlands conjecture \cite{ArinkinGaitsgory:2015:SingularSupport}.
By Langlands duality, one should then have a matching support theory for the category $\catDMod{\Bun_G}$ and the question arises whether it is possible to formulate this theory in a way that is intrinsic to D-modules.

A first step to this -- and also a problem of independent interest -- is to understand the Hochschild cohomology of the category $\catDMod{\stack X}$ of D-modules on a stack $\stack X$.
We will review the general setup and basic properties of D-modules on (QCA) stacks in Chapter~\ref{ch:d-mod:pre}.
The upshot is that we also have an isomorphism of dg algebras
\begin{equation}
    \label{eq:intro:d-mod:d-mod-hcoh}
    \HCoh(\catDMod{\stack X}) \cong \Hom_{\catDMod{X×X}}(Δ_*ω_{\stack X}, Δ_*ω_{\stack X}),
\end{equation}
where $Δ\colon \stack X × \stack X → \stack X$ is the diagonal morphism and $ω_{\stack X}$ is the dualizing module.
In particular if $\stack X$ is a (separated) scheme, then $Δ$ is a closed embedding and $(Δ^*,Δ_*)$ adjunction combined with Kashiwara's Lemma show that $\HCoh(\catDMod{\stack X})$ is isomorphic to the de Rham cohomology of $\stack X$.
However, if $\stack X$ is not an algebraic space (and hence $Δ$ is not proper) then the situation becomes more complicated.

By Verdier duality and adjunction we can always rewrite \eqref{eq:intro:d-mod:d-mod-hcoh} as
\[
    \HCoh(\catDMod{\stack X}) \cong
    \opalg{\Hom_{\catDMod{X}}(k_{\stack X}, Δ^!Δ_!k_{\stack X})} =
    \opalg{\ΓdR(\stack X, Δ^!Δ_!k_{\stack X})}.
\]
It is now tempting to complete to look at the Cartesian square
\[
    \begin{tikzcd}
        \ls \stack X \arrow[r, "p₁"] \arrow[d, "p₂"] & \stack X \arrow[d, "Δ"] \\
        \stack X \arrow[r, "Δ"] & \stack X × \stack X
    \end{tikzcd}
\]
where
\[
    \ls \stack X = \stack X ×\limits_{\stack X × \stack X} \stack X
\]
is the (derived) loop space of $\stack X$ and try to express the Hochschild cohomology as the cohomology of some sheaf on $\ls \stack X$.
Naively we could expect the existence of an isomorphism
\begin{equation}
    \label{eq:intro:d-mod:naive}
    \ΓdR(\stack X, Δ^!Δ_!k_{\stack X}) \cong
    \ΓdR(\stack X, p_{2,!}p₁^! k_{\stack X}).
\end{equation}
Unfortunately, the two sides are in general not isomorphic and it is not too hard to come up with a counter-example.

In Chapter~\ref{ch:d-mod:strategy} we will investigate how to quantify the cone of the morphism
\[
    p_{2,!}p₁^! k_{\stack X} → Δ^!Δ_! k_{\stack X}
\]
and thus the failure of the naive isomorphism \eqref{eq:intro:d-mod:naive} to hold.
As an application, we will prove the following theorem.

\begin{Thm}\label{thm:d-mod:main}
    Let $G \cong \Gm^n$ be a torus acting locally linearly on a scheme $X$ of finite type over $k$.
    Then there is a canonical isomorphism of algebras
    \[
        \HCoh\bigl(\catDMod{X/G}\bigr)
        \cong
        \opalg{\ΓdR\bigl(X/G,\,p_{2,!} p₁^! k_{X/G}\bigr)},
    \]
    where the algebra structure on $\ΓdR\bigl(X/G,\,p_{2,!} p₁^! k_{X/G}\bigr)$ is induced by the groupoid structure on $\ls(X/G)$.
\end{Thm}

Thus for torus quotients the naive expectation actually holds.
