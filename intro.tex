\chapter{Introduction}

\section{Microlocal theory of perverse coherent sheaves}

Basically the content of my talk:
\begin{itemize}
    \item intro to perverse coherent sheaves
    \item microlocal theory in the constructible setting
    \item statement of the result
\end{itemize}

\section{Hochschild cohomology of categories of D-modules}

\begin{itemize}
    \item Why study $\HCoh\left(\catDMod{\stack X}\right)$?
    \item Naive expectation and its problems.
    \item Statement of result.
    \item Outline of approach?
\end{itemize}

\subsection*{Short summary}

Let $\stack X$ be a stack.
The goal of the second part is to understand the Hochschild cohomology
\[
    \HCoh\bigl(\catDMod{\stack X}\bigr) 
    =
    \Hom_{\catDMod{\stack X × \stack X}}(Δ_* ω_{\stack X},\, Δ_* ω_{\stack X}).
\]
By duality and adjunction this is the same as
\[
    \opalg{\Hom_{\catDMod{\stack X}}(k_{\stack X},\, Δ^!Δ_! k_{\stack X})} =
    \opalg{\ΓdR(\stack X,\, Δ^!Δ_! k_{\stack X})}.
\]
For this we consider the (derived) loop space
\[
    \ls \stack X = \stack X ×\limits_{\stack X × \stack X} \stack X
\]
and the Cartesian diagram
\[
    \begin{tikzcd}
        \ls \stack X \arrow[r, "p₁"] \arrow[d, "p₂"] & \stack X \arrow[d, "Δ"] \\
        \stack X \arrow[r, "Δ"] & \stack X × \stack X
    \end{tikzcd}
\]
Naively, one could expect that
\[
    \opalg{\ΓdR(\stack X,\, Δ^!Δ_! k_{\stack X})} \cong
    \opalg{\ΓdR(\stack X,\, p_{2,!}p₁^! k_{\stack X})},
\]
but of course this base-change isomorphism does not exist in general.

We will try to quantify how much the morphism $p_{2,!}p₁^! k_{\stack X} → Δ^!Δ_! k_{\stack X}$ fails to be an isomorphism.
In particular we will show that it is an isomorphism for stacks of the form $\stack X = [X/\Gm^n]$.

\begin{Thm}\label{thm:d-mod:main}
    Let $G$ be a torus acting locally linearly on a scheme $X$ of finite type over $k$.
    Then there is a canonical isomorphism of algebras
    \[
        \HCoh\bigl(\catDMod{X/G}\bigr) 
        \cong
        \opalg{\ΓdR\bigl(X/G,\,p_{2,!} p₁^! k_{X/G}\bigr)}.
    \]
\end{Thm}
