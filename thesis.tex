\documentclass[english]{thesis}
%\includeonly{intro}
\title{Some microlocal aspects of perverse coherent sheaves and equivariant D-modules}
\author{Clemens Koppensteiner}
\version{0}

\usepackage{math-alg,math-ag, math-gl}

\addbibresource{global.bib}
%\bibliography{global.bib}

% general
% -------

\let\bar\overline

\renewcommand\sheafHom{\sheaf{H\mkern-0.3mu o\mkern-2.5mu m}}
% ToDo: make \sheafHom look nice for the given font.
\renewcommand\Forget{\operatorname{Forget}}

\let\stack\mathbf

% TeX Gyre Termes Math does not contain the following symbol
\AtBeginDocument{
    \renewcommand\setminus{-}
}

% Formatting of upper case roman numerals (in references).
\newcommand{\RomanNum}[1]{\textsc{\MakeLowercase{#1}}}

% perverse-coherents
% ------------------

\newcommand\dualize{\mathbb D}
\newcommand\dualizingCx{\sheaf{D\mkern-0.3mu C}}
\newcommand\lc[1]{\Gamma_{\mkern-3mu#1}}
\newcommand\measuringFam{\mathfrak M}
\newcommand\Xtop[1][X]{#1^{G,\mathrm{gen}}}

\newcommand\dcohb[1]{D_{\mathrm{coh}}^b(#1)}
\newcommand\dcohbG[2][G]{D_{\mathrm{coh}}^b(#2)^{#1}}
\newcommand\dconstrb[1]{D_{\mathrm{constr}}^b(#1)}

\newcommand\lciname{strong}

\newcommand\pervVC[1]{φ_{#1}[-1]}
\newcommand\localMorse[3]{M_{#1,#2}(#3)}

% d-modules
% ----------

\newcommand\catDModHol[1]{\catDMod[\mathup{hol}]{#1}}

\newcommand\dR{\mathrm{dR}}
\newcommand{\renstar}{\blacktriangle}

\newcommand{\HCoh}{\operatorname{HH}^\cx}
\newcommand\ΓdR{Γ_{\mkern-4mu\dR}}

\renewcommand\B{\stack{B}}
\newcommand\mappingstack{\mathbfup{Maps}}
\newcommand\X{\mathbb{X}}

\newcommand\ls[1]{\mathbfcal{L} #1}
\newcommand\cls[1]{\overline{\mathbfcal{L}} #1}
\newcommand\lsc[1]{\mathbfcal{L}^c #1}

\newfontfamily\cyrillicfont[Script=Cyrillic]{XITS}
\newcommand\cyrmath[1]{\textit{\cyrillicfont #1}}

%\newcommand\schemels[1]{\mathcal{L}\mkern2.5mu #1}
%\newcommand\schemecls[1]{\overline{\mathcal{L}}\mkern2.5mu #1}
%\newcommand\schemelsc[1]{\mathcal{L}\mkern2.5mu^c #1}
\newcommand\schemels[2][]{\cyrmath{Л}\mkern1.5mu^{#1}#2}
\newcommand\schemelsY[2][]{\schemels[#1]{_{\stack Y}#2}}
\newcommand\schemecls[2][]{\overline{\cyrmath{Л}}\mkern1.5mu^{#1}#2}
\newcommand\schemeclsY[2][]{\schemecls[#1]{_{\stack Y}#2}}
\newcommand\schemelsc[2][]{\cyrmath{Л}\mkern1.5mu^{\if\relax\detokenize{#1}\relax\else#1,\fi c} #2}
\newcommand\schemelscY[2][]{\schemelsc[#1]{_{\stack Y}#2}}

%\newcommand\schemei{\cyrmath{и}}
%\newcommand\schemej{\cyrmath{й}}
%\newcommand\schemep{\cyrmath{п}}
\newcommand\schemei{\tilde{\imath}}
\newcommand\schemej{\tilde{\jmath}}
\newcommand\schemeq{\tilde{q}}
\newcommand\schemeh{\tilde{h}}

\newcommand\Stab{\operatorname{Stab}}

\newcommand\goodstack{good}
\newcommand\Goodstack{Good}
\newcommand\goodness{goodness}


\begin{document}

\frontmatter

\maketitlepage

\begin{abstract}
    We discuss microlocal aspects of two types of sheaves which are of interest to geometric representation theory: perverse coherent sheaves and equivariant D-modules.

    The category of perverse sheaves on a complex variety is characterized by exactness of the microlocal stalks (or vanishing cycles) functor.
    We prove an analogue of this for the category of perverse coherent sheaves on a scheme with a group action.
    The main idea is to understand microlocal stalks via local cohomology along half-dimensional ("Lagrangian") subvarieties.
    We define \enquote{measuring subvarieties} as an analogue of these subvarieties in the coherent setting and show how they can be used to characterize perverse coherent sheaves.

    The second part of this thesis is dedicated to understanding the support theory (in the sense of \cite{BensonIyengarKrause:2008:LocalCohomologyAndSupportForTriangulatedCategories}) of equivariant D-modules.
    We discuss how to compute the Hochschild cohomology of the category of D-modules on a quotient stack via a relative compactification of the diagonal morphism.
    We then apply this construction to the case of torus-equivariant D-modules and describe the Hochschild cohomology as the cohomology of a D-module on the loop space of the corresponding quotient stack.
\end{abstract}

\chapter*{Acknowledgments}

First and foremost, I want to thank my advisor David Nadler.
His support and encouragement as well as his inexhaustible supply of interesting mathematics to think about has been invaluable.
Much of what I know I learned from him and his writing.

The projects constituting this thesis have benefited from discussions with Pramod Achar, Dennis Gaitsgory, Sam Gunningham and Nick Rozenblyum.
They are made possible by the work of many great mathematicians, but in particular by the work of Dima Arinkin, Vladimir Drinfeld and Dennis Gaitsgory.
The first part has benefited greatly from remarks by a referee of the paper \cite{Koppensteiner:pre:ExactFunctorsOnPerverseCoherentSheaves}.

Northwestern University's Mathematics Department has provided an exceptional environment and I would like to thank everyone involved for their work in keeping it this way.
I have learned much from the faculty (current and former) and my fellow graduate students alike.

I am very happy to have some wonderful friends, even though I see many of them way too rarely.
Especially this includes Ira, who went from saying \enquote{hi} a couple of times a week to so much more in these years.
Finally, I am particularly grateful for all the support I have received from my family despite the ever-growing geographical distance between us.


\tableofcontents

\mainmatter

\chapter{Introduction}

\section{Microlocal theory of perverse coherent sheaves}

Basically the content of my talk:
\begin{itemize}
    \item intro to perverse coherent sheaves
    \item microlocal theory in the constructible setting
    \item statement of the result
\end{itemize}

\section{Hochschild cohomology of categories of D-modules}

\begin{itemize}
    \item Why study $\HCoh\left(\catDMod{\stack X}\right)$?
    \item Naive expectation and its problems.
    \item Statement of result.
\end{itemize}


Let\todo{Write this nicely.}
\[
    \ls \stack X = \stack X ×\limits_{\stack X × \stack X} \stack X
\]
be the loop space of $\stack X = X/G$.
Let $p_i\colon \ls\stack X → \stack X$ be the projections.

\begin{Thm}\label{thm:d-mod:main}
    Let $G$ be a torus acting locally linearly\todo{Define \enquote{locally linear action}.}\ on a scheme $X$ of finite type over $k$.
    Then there is a canonical isomorphism of algebras
    \[
        \HCoh\bigl(\catDMod{X/G}\bigr) 
        \cong
        \opalg{\ΓdR\bigl(p_{2,!} p₁^! k_{X/G}\bigr)}.
    \]
\end{Thm}

\begin{itemize}
    \item Outline of approach?
\end{itemize}

\part{A microlocal description of perverse coherent sheaves}

\chapter{Prerequisites}

% finite type (implies Noetherian): dualizing complex is nice
% assumption of G: makes X/G have affine diagonal
Throughout this part we will be concerned with a scheme $X$ with an action by an affine group scheme $G$.
We assume that $X$ and $G$ are both of finite type over a field $k$.

We will make use of the usual notations for derived categories.
Thus $\dqc{X}$ is the derived category of the Abelian category of quasi-coherent sheaves on $X$ and $\dcoh{X}$ is its full subcategory of complexes with coherent cohomology.
More generally $\dcat{}{X}$ is the derived category of sheaves of $k$-modules on $X$.
For a ring $R$, the derived category of $R$-modules will be denoted by $\dcat{}{R}$.
To avoid cluttering the notation we will also usually suppress the signifiers $\mathbb R$ and $\mathbb L$ on functors between derived categories.
%Note that unlike in Part~\ref{part:d-mod} we will only work on the level of triangulated categories and not their dg enhancements.

If $Y$ is a subscheme of $X$ we will always write $ι_Y$ for the inclusion $Y \hookrightarrow X$.

\section{Operations on coherent sheaves}

We will be mainly concerned with the category $\dcohbG{X}$, the bounded derived category of $G$-equivariant coherent sheaves on $X$.
For the reader familiar with stacks, this is the same category as the bounded derived category of coherent sheaves on the quotient stack $[X/G]$.
It is also equivalent to the full subcategory of $\dqcbG{X}$ consisting of complexes with coherent cohomology \cite[Corollary~2.11]{ArinkinBezrukavnikov:2010:PerverseCoherentSheaves}.
There is a forgetful functor
\[
    \Forget\colon \dcohbG{X} → \dcohb{X}
\]
to the non-equivariant bounded derived category of coherent sheaves on $X$.
We will frequently apply functors defined on the latter category to equivariant sheaves without explicitly mentioning the intervening forgetful functor.

Let $Z$ be a closed subscheme of $X$.
Then there are functors $ι_Z^!$ and $ι_Z^*$ from $\dcohb X$ to $\dcohb Z$, defined by
\[
    ι_Z^*({-}) = \O_Z \otimes_{\O_X} {-}
    \qquad\text{and}\qquad
    ι_Z^!({-}) = \sheafHom_{\O_X}(\O_Z,\, {-}).
\]
We note again that the symbols $\otimes$ and $\sheafHom$ denote the corresponding derived functors.
If $U$ is an open subscheme of $X$ then the functor $ι_U^* = ι_U^!$ is the restriction functor $\dcohb X → \dcohb U$.
For a general locally closed subscheme, the restriction functors are defined be composing the above functors.

We will also need the corresponding $k$-module functors, which we will denote by bold letters.
In particular, if $Z$ is a closed subspace of $X$ (as a topological space), then we set
\[
    \mathbf{ι}_{Z}^!({-}) = \sheafHom_{k_X}(k_{Z},\, {-}).
\]
Following \cite[Variation~3 in~\RomanNum{IV}.1]{Hartshorne:1966:ResiduesAndDuality} we write
\[
    \lc{Z} = \mathbf{ι}_{Z,*} \mathbf{ι}_{Z}^!.
\]
Again we note that these functors should be seen as functors between the derived categories.
It is well known that if $\sheaf F$ is a (complex of) quasi-coherent sheaf(s) on $X$, then $\lc{Z} \sheaf F$ is again quasi-coherent \cite[Corollaire~\RomanNum{II}.3]{SGA2}.

Let $x$ be a (not necessarily closed) point of $X$.
We will write $\mathbf ι_x^*$ for the functor of talking stalks at $x$ and set $\mathbf ι_x^! = \mathbf ι_x^*\lc{\overline{\{x\}}}$.
As noted before, we will apply all of these functors to equivariant sheaves without explicitly mentioning the forgetful functor.

Finally we will need the Grothendieck-Verdier duality functor on $\dcohbG X$.
It is defined exactly as in the non-equivariant situation (cf.~\cite[Chapter~\RomanNum{V}]{Hartshorne:1966:ResiduesAndDuality}).
Thus, an equivariant dualizing complex on $X$ is an object $\dualizingCx ∈ \dcohbG X$ such each object $\sheaf F ∈ \dcohbG X$ is $\dualizingCx$-reflexive, i.e.~such that the natural transformation
\[
    \sheaf F → \sheafHom(\sheafHom(\sheaf F, \dualizingCx),\, \dualizingCx) \qquad (\sheaf F ∈ \dcohb X)
\]
is an isomorphism.
We write $\dualize$ for the endofunctor $\sheafHom({-}, \dualizingCx)$ of $\dcohbG X$.
Since $\sheafHom$ commutes with the forgetful functor, if $\dualizingCx$ is an equivariant dualizing complex, then $\Forget(\dualizingCx) ∈ \dcohb X$ is a (non-equivariant) dualizing complex.
Under our assumptions on $X$ there always exists an equivariant dualizing complex $\dualizingCx$ \cite[Theorem~2.18]{ArinkinBezrukavnikov:2010:PerverseCoherentSheaves}.
Using \cite[\RomanNum{V}.7]{Hartshorne:1966:ResiduesAndDuality}, we will further assume that for each (not necessarily closed) point $x ∈ X$ the complex $\mathbf{ι}_x^! \dualizingCx$ is concentrated in cohomological degree $-\dim x$.

We finish this section with two lemmas about vanishing of local cohomology above or below a certain degree.

\begin{Lem}
    \label{lem:pre:stalk-and-costalk-vanishing}%
    Let $\sheaf F$ be a coherent sheaf on $X$ and let $x$ be a closed point of $X$.
    Then $\mathbf ι_x^* \dualize \sheaf F ∈ \dcat{\ge 0}{\O_x}$ if and only if $\mathbf ι_x^! \sheaf F ∈ \dcat{\le 0}{\O_x}$.
\end{Lem}

\begin{proof}
    The proof of this lemma is essentially the same as the one of \cite[Lemma~3.3(a)]{ArinkinBezrukavnikov:2010:PerverseCoherentSheaves}.
    Concretely, by \cite[\RomanNum{V}.6]{Hartshorne:1966:ResiduesAndDuality}, there is an isomorphism of functors
    \[
        \mathbf ι_z^!({-}) \cong \Hom_{\O_x}\bigl(\dualize({-}),\, \sheaf I_x\bigr),
    \]
    where $\sheaf I_x$ is the injective hull of the residue field of $\O_x$.
    The statement now follows from the fact that $\Hom_{\O_x}({-},\, \sheaf I_x)$ is exact and kills no finitely generated $\O_x$-module \cite[\RomanNum{V}.5]{Hartshorne:1966:ResiduesAndDuality}.
\end{proof}

\begin{Lem}
    \label{lem:pre:top-and-qc-restriction-vanishing}%
    Let $\sheaf F$ be a coherent sheaf on $X$ and let $Z$ be a closed subvariety of $X$.
    Then $\lc{Z} \sheaf F ∈ \dcat{\ge 0}{Z}$ if and only if $ι_Z^! \sheaf F ∈ \dcat{\ge 0}{Z}$.
\end{Lem}

\begin{proof}
    This is the equivalence of (i) and (ii) in \cite[Proposition~\RomanNum{VII}.1.2]{SGA2} with $Y = S = Z$, $G = \sheaf F$, $F = \O_Z$ and $n=1$.
\end{proof}

\section{Perverse coherent sheaves}
\label{sec:pc:pre:pc}

We keep the general assumptions on $X$ and the dualizing complex $\dualizingCx$.
We write $\Xtop$ for the set of generic points of $G$-stable subschemes of $X$.
It is a subset of the topological space of $X$ and we will consider it with the induced topology.

By a \emph{perversity} we mean a function $p\colon \{0,\dotsc,\dim X\} → ℤ$.
For $x ∈ \Xtop$ we abuse notation and set $p(x) = p(\dim x)$.
Then $p\colon \Xtop → ℤ$ is a perversity function in the sense of~\cite{Bezrukavnikov:arXiv:PerverseCoherentSheaves}.
Note that we insist that $p(x)$ only depends on the dimension of $x$.

A perversity is called \emph{monotone} if it is decreasing and \emph{comonotone} if the \emph{dual perversity} $\overline p(n) = -n - p(n)$ is decreasing.
It is \emph{strictly monotone} (resp.~\emph{strictly comonotone}) if for all $x,y ∈ \Xtop$ with $\dim x < \dim y$ one has $p(x) > p(y)$ (resp.~$\overline p(x) > \overline p(y)$).
Note that a strictly monotone perversity is not necessarily strictly decreasing (e.g.~if $X$ only has even-dimensional $G$-orbits).

Following \cite{ArinkinBezrukavnikov:2010:PerverseCoherentSheaves} we now have all ingredients to define the perverse t-structure on $\dcohbG X$.

\begin{Def}
    \label{def:perverse-t-structure}%
    Given a perversity $p$ we define the following full subcategories of $\dcohbG X$:
    \begin{align*}
        \dpc{≤0}{X} & =
        \bigl\{ \sheaf F ∈ \dcohb{X} : \mathbf ι_x^*\sheaf F ∈ \dcat{≤p(x)}{\O_x} \text{ for all $x ∈ \Xtop$}\bigr\}, \\
        \dpc{≥0}{X} & =
        \bigl\{ \sheaf F ∈ \dcohb{X} : \mathbf ι_x^!\sheaf F ∈ \dcat{≥p(x)}{\O_x} \text{ for all $x ∈ \Xtop$}\bigr\}.
    \end{align*}
\end{Def}

\begin{Thm}[{\cite[Theorem~3.10]{ArinkinBezrukavnikov:2010:PerverseCoherentSheaves}}]
    \label{thm:perverse-t-structure}%
    If $p$ is monotone and comonotone, then $(\dpc{≤0}{X},\, \dpc{≤0}{X})$ defines a t-structure on $\dcohbG X$.
\end{Thm}

This t-structure is called the \emph{perverse t-structure} with respect to $p$ on $\dcohbG X$.
Objects in its heart are called \emph{perverse coherent sheaves} (with respect to $p$ on $X$).

The perverse t-structure is compatible with duality, exchanging the perversity $p$ with its dual.

\begin{Lem}[{\cite[Lemma~3.3]{ArinkinBezrukavnikov:2010:PerverseCoherentSheaves}}]
    \label{lem:perverse-t-structure-and-duality}%
    Let $p$ be any perversity. Then
    \[
        \dualize\bigl(\dpc{≤0}{X}\bigr) = \dpc[\overline p]{≥0}{X}.
    \]
\end{Lem}

\begin{Ex}
    The best-studied case of perverse coherent sheaves is the nilpotent cone $N$ of a semi-simple algebraic group $G$ with the adjoint action.
    It is well known that there are finitely many $G$-orbits on $N$, all of which are even dimensional.
    Thus there is a \emph{middle perversity} given by
    \[
        p(x) = \overline p(x) = -\frac 12 \dim x, \qquad x ∈ \Xtop[N].
    \]
    This t-structure has important applications in geometric representation theory, for example \cite{Bezrukavnikov:2003:QuasiExceptionalSets} and \cite{BezrukavnikovMircovic:2013:RepresentationsOfSSLieAlgebrasInPrimeCharAndNoncommutativeSpringerResolution}.
    For an overview of the theory of perverse coherent sheaves on nilpotent cones and the related category of exotic sheaves we refer to \cite{Achar:arXiv:NotesOnExoticAndPerverseCoherentSheaves}.
\end{Ex}

For later use we state the following variant of the Grothendieck Finiteness Theorem \cite[Théorème~2.1]{SGA2}.
The given formulation is from \cite[Corollary~3.12]{ArinkinBezrukavnikov:2010:PerverseCoherentSheaves}, where the reader can find a short proof using the theory of perverse coherent sheaves.
\begin{Thm}
    \label{thm:grothendieck-finiteness}%
    Let $p$ be a monotone and comonotone perversity on $X$.
    Let $x ∈ \Xtop$, set $U = X \setminus \overline x$ and let $j\colon U \hookrightarrow X$ be the inclusion.
    Let $\sheaf F ∈ \dpc{≥0}{U}$.
    Then $H^n(j_*\sheaf F)$ is coherent for $n ≤ p(x)-2$.
\end{Thm}

\chapter{Measuring subvarieties}

\begin{Assumption}
    In this chapter we will always assume that $p$ is a monotone and comonotone perversity function.
    Then Theorem~\ref{thm:perverse-t-structure} guarantees the existence the perverse t-structure on $D_c^b(X)^G$.
\end{Assumption}

\section{Some reformulations}

In this  section we will give some reformulations of the perverse t-structure from Definition~\ref{def:perverse-t-structure}.
The equivalent conditions are inspired by Kashiwara's definition of a (non-equivariant) perverse t-structure on $D_c^b(X)$ in~\cite{Kashiwara:2004:tStructureOnHolonomicDModuleCoherentOModules}.

% TODO: Do we actually need the comonotonicity assumption (i.e. the t-structure?)
\begin{Prop}\label{prop:equivDeligneKashiwara:le}%
    Let $\sheaf F ∈ D_c^b(X)^G$.
    The following are equivalent:
    \begin{enumerate}
        \item \label{li:prop:equivDeligneKashiwara:le:1}%
            $\sheaf F ∈ \perv D^{≤0}(X)^G$, i.e.\ $\mathbf ι_x^*\sheaf F ∈ D^{≤p(x)}(\catModules{\O_x})$ for all $x ∈ \Xtop$;
        \item \label{li:prop:equivDeligneKashiwara:le:2}%
            $p(\dim \supp H^{k}(\sheaf F)) ≥ k$ for all $k$.
    \end{enumerate}
\end{Prop}

A crucial fact that we will implicitly use quite often in the following arguments is that the support of a coherent sheaf is always closed.
In particular, this means that if $x$ is a generic point and $\sheaf F$ a coherent sheaf, then $\mathbf ι_x^* \sheaf F = 0$ if and only if $\res{\sheaf F}U = 0$ for some open set $U$ intersecting $\overline x$.

\begin{proof}
    First let $\sheaf F ∈ \perv D^{≤0}(X)^G$ and assume for contradiction that there exists an integer $k$ such that $p(\dim \supp H^{k}(\sheaf F)) < k$.
    Let $x$ be the generic point of an irreducible component of maximal dimension of $\supp H^{k}(\sheaf F)$.
    Then $H^k(\mathbf ι_x^* \sheaf F) \ne 0$. 
    But on the other hand, $\mathbf ι_x^*\sheaf F ∈ D^{≤p(x)}(\catModules{\O_x})$ and $p(x) = p(\dim \supp H^{k}(\sheaf F)) < k$, yielding a contradiction.

    Conversely assume that $p(\dim \supp H^{k}(\sheaf F)) ≥ k$ for all $k$ and let $x ∈ \Xtop$.
    If $H^k(\mathbf ι_x^*\sheaf F) \ne 0$, then $\dim x ≤ \dim \supp H^{k}(\sheaf F)$.
    Thus monotonicity of the perversity implies that $\sheaf F ∈ \perv D^{≤0}(X)^G$.
\end{proof}

% TODO: Where do we use comonotonicity (just for definition of the t-structure? Do we actually need the t-structure?).
\begin{Prop}\label{prop:equivDeligneKashiwara:ge}%
    Let $\sheaf F ∈ D_c^b(X)^G$ and let $p$ be strictly monotone (in addition to our standard comonotonicity assumption).
    \begin{enumerate}
        \item \label{li:prop:equivDeligneKashiwara:ge:1}%
            $\sheaf F ∈ \perv D^{\ge 0}(X)^G$, i.e.\ $\mathbf ι_x^!\sheaf F ∈ D^{≥p(x)}(\catModules{\O_x})$ for all $x ∈ \Xtop$;
        \item \label{li:prop:equivDeligneKashiwara:ge:2}%
            $\lc {\overline x}\sheaf F ∈ D^{≥p(x)}(X)$ for all $x ∈ \Xtop$;
        \item \label{li:prop:equivDeligneKashiwara:ge:3}%
            $\lc {Y}\sheaf F ∈ D^{≥p(\dim Y)}(X)$ for all $G$-invariant closed subvarieties $Y$ of $X$;
        \item \label{li:prop:equivDeligneKashiwara:ge:4}%
            $\dim\left( \overline x ∩ \supp\left( H^k(\dualize \sheaf F) \right) \right) ≤ -p(x) - k$ for all $x ∈ \Xtop$ and all $k$.
    \end{enumerate}
\end{Prop}

\begin{proof}
    The implications from \ref{li:prop:equivDeligneKashiwara:ge:3} to \ref{li:prop:equivDeligneKashiwara:ge:2} and \ref{li:prop:equivDeligneKashiwara:ge:2} to \ref{li:prop:equivDeligneKashiwara:ge:1} are trivial and the equivalence of \ref{li:prop:equivDeligneKashiwara:ge:2} and \ref{li:prop:equivDeligneKashiwara:ge:4} follows from Lemma~\ref{lem:supportAndLocalCohomology+} below.
    Thus we only need to show that \ref{li:prop:equivDeligneKashiwara:ge:1} implies \ref{li:prop:equivDeligneKashiwara:ge:3}.
    So assume that $\sheaf F ∈ \perv D^{≥0}(X)^G$.
    We induct on the dimension of $Y$.
    
    If $\dim Y = 0$, then $Γ(X,\lc Y \sheaf F) = \bigoplus_{y ∈ \Xtop[Y]} \mathbf ι_y^!\sheaf F$ and thus $\lc Y\sheaf F ∈ D^{≥p(0)}(X)$ by assumption.

    Now let $\dim Y > 0$.
    We first assume that $Y$ is irreducible with generic point $x ∈ \Xtop$.
    Let $k$ be the smallest integer such that $H^k(\lc {\overline x} \sheaf F) \ne 0$ and assume that $k < p(x)$.
    We will show that this implies that $H^k(\lc {\overline x}\sheaf F) = 0$, giving a contradiction.

    We first show that $H^k(\lc {\overline x}\sheaf F)$ is coherent.
    Let $j\colon X \setminus {\overline x} \hookrightarrow X$ and consider the distinguished triangle
    \[
        \lc {\overline x} \sheaf F → \sheaf F → j_*j^* \sheaf F \xrightarrow{+1}.
    \]
    Applying cohomology to it we get an exact sequence
    \[
        H^{k-1}(j_*j^*\sheaf F) → H^k(\lc{\overline x} \sheaf F) → H^k(\sheaf F).
    \]
    By assumption, $k-1 \le p(x) - 2$, so that $H^{k-1}(j_*j^*\sheaf F)$ is coherent by the Grothendieck Finiteness Theorem \ref{thm:grothendieck-finiteness}.
    As $H^k(\sheaf F)$ is coherent by definition, this implies that $H^k(\lc{\overline x} \sheaf F)$ also has to be coherent.

    Set $Z = \supp H^k(\lc {\overline x}\sheaf F)$.
    Then, since $ι_x^* H^k(\lc {\overline x}\sheaf F) = H^k(\mathbf ι_x^! \sheaf F)$ vanishes, $Z$ is a proper closed subset of $\overline x$.
    We consider the distinguished triangle
    \[
        H^k(\lc {\overline x}\sheaf F)[-k] →
        \lc {\overline x}\sheaf F →
        τ_{>k}\lc {\overline x}\sheaf F \xrightarrow{+1},
    \]
    and apply $\lc Z$ to it:
    \[
        \lc Z H^k(\lc {\overline x}\sheaf F)[-k] =
        H^k(\lc {\overline x}\sheaf F)[-k] →
        \lc Z \sheaf F →
        \lc Z τ_{>k}\lc {\overline x}\sheaf F \xrightarrow{+1}.
    \]
    Since $\dim Z < \dim x$, we can use the induction hypothesis and monotonicity of $p$ to deduce that $\lc Z \sheaf F$ is in degrees at least $p(\dim Z) \ge p(x) > k$.
    Clearly $\lc Z τ_{>k}\lc {\overline x}\sheaf F$ is also in degrees larger than $k$.
    Hence $H^k(\lc {\overline x}\sheaf F)$ has to vanish.

    If $Y$ is not irreducible, let $Y₁$ be an irreducible component of $Y$ and $Y₂$ be the union of the other components.
    Then there is a Mayer-Vietoris distinguished triangle
    \[
        \lc {Y₁\cap Y₂} \sheaf F → \lc {Y₁} \sheaf F \oplus \lc{Y₂}\sheaf F → \lc{Y} \sheaf F \xrightarrow{+1},
    \]
    where $\lc {Y₁\cap Y₂} \sheaf F ∈ D^{\ge p(\dim Y₁\cap Y₂)}(X) \subseteq D^{\ge p(\dim Y)+1}$ (by the induction hypothesis and strict monotonicity of $p$) and $\lc{Y₁} \sheaf F$ and $\lc{Y₂} \sheaf F$ are in $D^{\ge p(\dim Y)}(X)$ by induction on the number of components of $Y$.
    Thus $\lc Y \sheaf F ∈ D^{\ge p(\dim Y)}$ as required.
\end{proof}

\begin{Lem}[{\cite[Proposition~5.2]{Kashiwara:2004:tStructureOnHolonomicDModuleCoherentOModules}}]%
    \label{lem:supportAndLocalCohomology+}%
    Let $\sheaf F ∈ D_c^b(X)$, $Z$ a closed subset of $X$, and $n$ an integer.
    Then $\lc Z\sheaf F ∈ D_{qc}^{≥n}(X)$ if and only if $\dim(Z∩\supp(H^k(\dualize \sheaf F))) \le - k - n$ for all $k$.
\end{Lem}

This lemma extends \cite[Proposition~5.2]{Kashiwara:2004:tStructureOnHolonomicDModuleCoherentOModules} to singular varieties.
The proof is essentially the same as for the smooth case, but we will include it here for completeness.

\begin{proof}
    By~\cite[Proposition~\textsc{vii}.1.2]{SGA2}, $\lc Z\sheaf F ∈ D_{qc}^{\ge n}(X)$ if and only if 
    \begin{equation}
        \label{eq:supportAndCohomology+}%
        \sheafHom(\sheaf G,\sheaf F) ∈ D_c^{\ge n}(X)
    \end{equation}
    for all $\sheaf G ∈ \catCoh{X}$ with $\supp \sheaf G \subseteq Z$.
    Let $d(n) = -n$ be the dual standard perversity.
    Then by~\cite[Lemma~5a]{Bezrukavnikov:arXiv:PerverseCoherentSheaves}, \eqref{eq:supportAndCohomology+} holds if and only if $\dualize \sheafHom(\sheaf G,\sheaf F) ∈ \perv[d] D^{\le -n}(X)$.
    By~\cite[Proposition~\textsc{v}.2.6]{Hartshorne:1966:ResiduesAndDuality}, $\dualize \sheafHom(\sheaf G,\sheaf F) = \sheaf G \otimes_{\O_X} \dualize \sheaf F$, so that by Proposition~\ref{prop:equivDeligneKashiwara:le} we need to show that
    \[
        \dim \supp H^{k}\left(\sheaf G \otimes_{\O_X} \dualize \sheaf F\right) \le - k - n 
    \]
    for all $k$.
    By~\cite[Lemma~5.3]{Kashiwara:2004:tStructureOnHolonomicDModuleCoherentOModules} (whose proof does not use the smoothness assumption) this is equivalent to
    \[
        \dim \left(Z∩\supp H^{k}(\dualize \sheaf F)\right) \le - k - n
    \]
    for all $k$, completing the proof.
\end{proof}


\section{Perverse coherent sheaves via measuring subvarieties}

\begin{Assumption}
    From now on we will assume that the $G$-action on $X$ has finitely many orbits.
\end{Assumption}

\begin{Def}
    Let $p$ be a perversity.
    A \emph{$p$-measuring subvariety} of $X$ is a closed subvariety $Z$ of $X$ such that 
    \[
        \dim(\overline x ∩ Z) = \dim x + p(x)
    \]
    for each $x ∈ \Xtop$ with $\overline x ∩ Z \ne \emptyset$. 
    If in addition $\overline x \cap Z$ is a set-theoretic local complete intersection in $\overline x$ for each $x ∈ \Xtop$, then $Z$ is called a \emph{\lciname} $p$-measuring subvariety.
    A \emph{(\lciname) $p$-measuring collection of subvarieties} of $X$ is a collection $\measuringFam$ of (\lciname) $p$-measuring subvarieties $Z$ such that for each $x ∈ \Xtop$ there exists $Z ∈ \measuringFam$ with $\overline x \cap Z \ne \emptyset$.
\end{Def}

\begin{Rem}
    \label{rem:measuring_dimensions}%
    Let $Z$ be a $p$-measuring subvariety.
    The condition on $p$-measuring subvarieties can be rewritten as $\dim(\overline x ∩ Z) = - \bar p(x)$ and $\codim_{\overline x}(\overline x ∩ Z)= -p(x)$.
    Thus comonotonicity of $p$ ensures that if $\dim y ≤ \dim x$ then $\dim (\overline y ∩ Z) ≤ \dim (\overline x ∩ Z)$.
    Monotonicity of $p$ then further says that $\codim_{\overline y}(\overline y ∩ Z) \le \codim_{\overline x}(\overline x ∩ Z)$.

    We clearly have $0 \le \dim(\overline x ∩ Z) \le \dim x$ and hence $-\dim x \le p(x) \le 0$.
    We will show in Theorem~\ref{thm:existence_of_lcimeasuring} that the condition $-\dim x \le p(x) \le 0$ is sufficient for the existence of a \lciname\ $p$-measuring collection, at least when $X$ is affine.
\end{Rem}

% Warning: the conditions might be mixed up.
\begin{Thm}\label{thm:main}%
    Let $p$ be a strictly monotone and (not necessarily strictly) comonotone perversity on $X$.
    Let $\sheaf F \in D_c^b(X)^G$.
    \begin{enumerate}
        \item\label{li:thm:main:ge}% strict comonotonicity
            Assume that $X$ has a $p$-measuring collection $\measuringFam$.
            Then the following are equivalent.
            \begin{enumerate}
                \item\label{li:thm:main:ge:def}
                    $\sheaf F \in \perv D^{≥0}(X)^G$;
                \item\label{li:thm:main:ge:qc}
                    $i_Z^!\sheaf F \in D^{≥0}(Z)$ for all $Z ∈ \measuringFam$;
                \item\label{li:thm:main:ge:top}
                    $\lc Z \sheaf F \in D^{≥0}(Z)$ for all $Z ∈ \measuringFam$.
            \end{enumerate}
        \item\label{li:thm:main:le-weak}% strict monotonicity
            Assume that $X$ has a $\bar p$-measuring collection $\measuringFam$.
            Then the following are equivalent.
            \begin{enumerate}
                \item $\sheaf F \in \perv D^{≤0}(X)^G$;
                \item $\lc{z}i_Z^*\sheaf F \in D^{≤0}(Z)$ for all $Z ∈ \measuringFam$ and all $z ∈ Z$.
            \end{enumerate}
        \item\label{li:thm:main:le-lci}% neither(?)
            Assume that $X$ has a \lciname\ $p$-measuring collection $\measuringFam$.
            Then the following are equivalent.
            \begin{enumerate}
                \item $\sheaf F \in \perv D^{≤0}(X)^G$;
                \item $\lc Z \sheaf F \in D^{≤0}(Z)$ for all $Z ∈ \measuringFam$.
            \end{enumerate}
    \end{enumerate}
    In particular, if $X$ has a \lciname\ $p$-measuring collection $\measuringFam$, then $\sheaf F$ is perverse with respect to $p$ if and only if $\lc Z\sheaf F$ is cohomologically concentrated in degree $0$ for each $Z ∈ \measuringFam$.
\end{Thm}

% TODO: Reformulate the arguments in this proof. It is slightly convoluted.
\begin{proof}[Proof of Theorem~\ref{thm:main}\ref{li:thm:main:ge}]
    The equivalence of \ref{li:thm:main:ge:top} and \ref{li:thm:main:ge:qc} follows directly from Lemma~\ref{lem:pre:top-and-qc-restriction-vanishing}.
    We will prove the equivalence of \ref{li:thm:main:ge:def} and \ref{li:thm:main:ge:top}.

    By Proposition~\ref{prop:equivDeligneKashiwara:ge}, $\sheaf F ∈ \perv D^{≥0}(X)^G$ if and only if
    \begin{align}
        \label{eq:main:+supp1}%
        & \dim\left( \overline x ∩ \supp\left( H^k(\dualize F) \right) \right) ≤ -p(x) - k &&  \text{for all $x ∈ \Xtop$ and all $k$}. \\
        %
        \intertext{Using Lemma~\ref{lem:supportAndLocalCohomology+} for $\lc Z\sheaf F ∈ D^{≥0}(X)$, we see that we have to show the equivalence of \eqref{eq:main:+supp1} with}
        %
        \notag
        & \dim\left( Z ∩ \supp\left( H^k(\dualize F) \right) \right) ≤ - k && \text{ for all $k$ and all $Z ∈ \measuringFam$}. \\
        %
        \intertext{Since there are only finitely many orbits, this is in turn equivalent to}
        %
        \label{eq:main:+supp2}%
        & \dim\left( Z ∩ \overline x ∩ \supp\left( H^k(\dualize F) \right) \right) ≤ - k && \text{ $\forall\, x ∈ \Xtop$, $k$ and $Z ∈ \measuringFam$}.
    \end{align}
    We will show the equivalence for each fixed $k$ separately.
    Let us first show the implication from \eqref{eq:main:+supp1} to \eqref{eq:main:+supp2}.
    Since $H^k(\dualize \sheaf F)$ is $G$-equivariant and there are only finitely many $G$-orbits, it suffices to show \eqref{eq:main:+supp2} assuming that $\dim x \le \dim \supp H^k(\dualize F)$ and $\overline x \cap \supp H^k(\dualize F) \ne \emptyset$.
    Then $\dim\left(\overline x ∩ \supp\left( H^k(\dualize F) \right)\right) = \dim \overline x$.
    Thus,
    \begin{multline*}
        \dim\left(Z ∩ \overline x ∩ \supp\left( H^k(\dualize F) \right) \right) \le
        \dim(Z ∩ \overline x) =
        p(x) + \dim x = \\
        p(x) + \dim\left(\overline x ∩ \supp\left( H^k(\dualize F) \right)\right) \le
        p(x) - p(x) - k
        = -k.
    \end{multline*}
    
    Conversely, assume that \eqref{eq:main:+supp2} holds for $k$.
    If $\overline x \cap \supp H^k(\dualize F) = \emptyset$, then \eqref{eq:main:+supp1} is trivially true.
    Otherwise choose a $p$-measuring $Z$ that intersects $\supp H^k(\dualize F)$.
    First assume that $\overline x$ is contained in $\supp H^k(\dualize F)$.
    Then
    \begin{multline*}
        \dim\left(\overline x ∩ \supp\left( H^k(\dualize F) \right)\right) =
        \dim x =
        -p(x) + \dim(Z ∩ \overline x) = \\
        -p(x) + \dim\left(Z ∩ \overline x ∩ \supp\left( H^k(\dualize F) \right) \right) \le
        -p(x) - k.
    \end{multline*}
    Otherwise $\overline x ∩ \supp\left( H^k(\dualize F) \right) = \overline y$ for some $y ∈ \Xtop$ with $\dim y < \dim x$.
    Then \eqref{eq:main:+supp1} holds for $y$ in place of $x$ and hence
    \[
    \dim\left( \overline x ∩ \supp\left( H^k(\dualize F) \right) \right) =
    \dim\left( \overline y ∩ \supp\left( H^k(\dualize F) \right) \right) ≤
    -p(y) - k ≤
    -p(x) - k
    \]
    by monotonicity of $p$.
\end{proof}

% TODO: Some transitional text.

\begin{proof}[Proof of Theorem~\ref{thm:main}\ref{li:thm:main:le-weak}]
    Let $\sheaf F ∈ \perv D^{\le 0}(X)^G$.
    By Lemma~\ref{lem:perverse-t-structure-and-duality} this is equivalent to $\dualize \sheaf F ∈ \perv[\bar p] D^{\ge 0}(X)^G$\todo{(Strict) comonotonicity of $\bar p$?}.
    By part~\ref{li:thm:main:ge} this is in turn equivalent to $\mathbf ι_z^* ι_Z^! \dualize \sheaf F ∈ D^{\ge 0}(\catVect)$.
    The sheaf $ι_Z^! \dualize \sheaf F = \dualize ι_Z^* \sheaf F$ is coherent, so that the statement follows from Lemma~\ref{lem:pre:stalk-and-costalk-vanishing}.
\end{proof}

The following lemma encapsulates the central argument of the proof of the remaining part of Theorem~\ref{thm:main}.

\begin{Lem}\label{lem:supportAndLocalCohomology-}%
    Let $\sheaf F ∈ \catCoh{X}^G$ be a $G$-equivariant coherent sheaf on $X$, let $p$ be a monotone perversity and let $n$ be an integer.
    Assume that $X$ has enough $p$-measuring subvarieties and let $\measuringFam$ be a $p$-measuring family of subvarieties of $X$.
    Then the following are equivalent:
    \begin{enumerate}
        \item $p(\dim \supp \sheaf F) ≥ n$;
        \item \label{li:lem:supportAndLocalCohomology-:2}%
            $H^i(\lc Z\sheaf F) = 0$ for all $i ≥ -n+1$ and all $Z ∈ \measuringFam$.
    \end{enumerate}
\end{Lem}

\begin{proof}
    Since $\supp \sheaf F$ is always a union of the closure of orbits, we can restrict to the support and assume that $\supp \sheaf F = X$.

    First assume that $p(\dim X) = p(\dim \supp \sheaf F) ≥ n$.
    Using a Mayer-Vietoris argument it suffices to check condition \ref{li:lem:supportAndLocalCohomology-:2} in the case that $X$ is irreducible.
    By the definition of a $p$-measuring subvariety and monotonicity of $p$, this implies that, up to radical, $Z$ can be locally defined by at most $-n$ equations.
    Thus $H^i(\lc Z\sheaf F) = 0$ for $i > -n$ \cite[Theorem~3.3.1]{BrodmannSharp:1998:LocalCohomology}. 

    Now assume conversely that $H^i(\lc Z\sheaf F) = 0$ for all $i ≥ -n+1$ and all measuring subvarieties $Z ∈ \measuringFam$.
    We have to show that $p(\dim X) ≥ n$.
    Set $d = \dim X$.
    Choose any $p$-measuring subvariety $Z ∈ \measuringFam$ that intersects a maximal component of $X$ non-trivially.
    Then $\codim_X Z = -p(d)$.
    We will show that $H^{-p(d)}(\lc Z \sheaf F) \ne 0$ and hence $p(d) \ge n$ by assumption.
    Take some affine open subset $U$ of $X$ such that $U \cap Z$ is non-empty, irreducible and of codimension $-p(d)$ in $U$.
    It suffices to show that the cohomology is non-zero in $U$.
    Thus we can assume without loss of generality that $X$ is affine, say $X = \Spec A$, and $Z$ is irreducible.
    Write $Z = V(\ideal p)$ for some prime ideal $\ideal p$ of $A$.
    By flat base change~\cite[Theorem~4.3.2]{BrodmannSharp:1998:LocalCohomology},
    \[
    Γ(X,H^{-p(d)}(\lc Z \sheaf F))_{\ideal p} = 
    \left(H_{\ideal p}^{-p(d)}(Γ(X,\sheaf F))\right)_{\ideal p} =
    H_{\ideal p_{\ideal p}}^{-p(d)}(Γ(X,\sheaf F)_{\ideal p})
    \]
    Since $\dim \supp \sheaf F = \dim X = d$, the dimension of the $A_{\ideal p}$-module $Γ(X,\sheaf F)_{\ideal p}$ is $-p(d)$.
    Thus by the Grothendieck non-vanishing theorem~\cite[Theorem~6.1.4]{BrodmannSharp:1998:LocalCohomology}
    %\cite[Théorème~V.3.1]{SGA2}
    $H_{\ideal p_{\ideal p}}^{-p(d)}(Γ(X,\sheaf F)_{\ideal p}) \ne 0$ and hence $Γ(X,H^{-p(d)}(\lc Z \sheaf F)) \ne 0$ as required.
\end{proof}

\begin{proof}[Proof of Theorem~\ref{thm:main}\ref{li:thm:main:le-lci}]
    We use the description of $\perv[p] D^{≤0}(X)^G$ given by Proposition~\ref{prop:equivDeligneKashiwara:le}, i.e.
    \[
    \perv D^{≤0}(X)^G = \left\{ \sheaf F ∈ D_c^b(X)^G : p\left(\dim\left( \supp H^{n}(\sheaf F) \right)\right) ≥ n \text{ for all $n$}\right\}.
    \]
    We induct on the largest $k$ such that $H^k(\sheaf F) \ne 0$ to show that $\sheaf F ∈ \perv D^{≤0}(X)^G$ if and only if $\lc Z\sheaf F ∈ D^{≤0}(X)$ for all $p$-measuring subvarieties $Z ∈ \measuringFam$.

    The equivalence is trivial for $k \ll 0$.
    For the induction step note that there is a distinguished triangle
    \[
    τ_{<k} \sheaf F → \sheaf F → H^k(\sheaf F)[-k] \xrightarrow{+1}.
    \]
    Applying the functor $\lc Z$ and taking cohomology we obtain an exact sequence
    \begin{multline*}
        \cdots →
        H¹(\lc Z(τ_{<k} \sheaf F)) →
        H¹(\lc Z\sheaf F) →
        H^{k+1}(\lc Z(H^k(\sheaf F))) → \\
        H²(\lc Z(τ_{<k} \sheaf F)) →
        H²(\lc Z\sheaf F) →
        H^{k+2}(\lc Z(H^k(\sheaf F))) →
        \cdots.
    \end{multline*}
    By induction, $H^j(\lc Z(τ_{<k} \sheaf F))$ vanishes for $j ≥ 1$ so that $H^j(\lc Z\sheaf F) \cong H^{k+j}(\lc Z(H^k(\sheaf F)))$ for $j ≥ 1$.
    Thus the statement follows from Lemma~\ref{lem:supportAndLocalCohomology-}.
\end{proof}

\section{Existence of \lciname\ measuring subvarieties}

Of course, for Theorem~\ref{thm:main} to have any content, one needs to show that $X$ has enough $p$-measuring subvarieties.
The next theorem shows that at least for affine varieties there are always enough measuring subvarieties whenever $p$ satisfies the obvious conditions (see Remark~\ref{rem:measuring_dimensions}).

\begin{Thm}\label{thm:existence_of_lcimeasuring}%
    Assume that $X$ is affine and the perversity $p$ is monotone and comonotone and satisfies $-n \le p(n) \le 0$ for $n ∈ \{0,\dotsc,\dim X\}$.
    Then $X$ has enough \lciname\ $p$-measuring subvarieties.
\end{Thm}

\begin{proof}
    Let $X = \Spec A$.
    We induct on the dimension $d$.
    More precisely, we induct on the following statement:
    \begin{quote}
        There exists a closed subvariety $Z_d$ of $X$ such that for all $x ∈ \Xtop$ the following holds:
        \begin{itemize}
            \item $Z_d \cap \overline x \ne \emptyset$ and $Z_d \cap \overline x$ is a set-theoretic local complete intersection in $\overline x$;
            \item if $\dim x \le d$, then $\dim(\overline x ∩ Z_d) = p(x) + \dim x$;
            \item if $\dim x > d$, then $\dim(\overline x ∩ Z_d) = p(d) + \dim x$.
        \end{itemize}
    \end{quote}
    We set $p(-1) = 0$.
    The statement is trivially true for $d = -1$, e.g.~take $Z = X$.
    Assume that the statement is true for some $d \ge -1$.
    We want to show it for $d+1 \le \dim X$.

    If $p(d) = p(d+1)$, then $Z_{d+1} = Z_{d}$ works.
    Otherwise, by (co)monotonicity, $p(d+1) = p(d) - 1$.
    Set $S = \bigcup \{ \overline x ∈ \Xtop : \dim x \le d\}$.
    Since there are only finitely many orbits, we can choose a function $f$ such that $f$ vanishes identically on $S$, $V(f)$ does not share a component with $Z_d$ and $V(f)$ intersects every $\overline x$ with $\dim x > d$.
    Then $Z_{d+1} = Z_d \cap V(f)$ satisfies the conditions.
\end{proof}


%\chapter{Perverse constructible sheaves}

In this chapter we will consider a complex manifold $X$ together with a stratification $\setset S$ by complex submanifold.
We let $D_{\setset S}^b(X)$ be the bounded derived category of $\setset S$-constructible sheaves on $X$.
Recall that there is a \emph{middle perverse} t-structure on $D_{\setset S}^b(X)$, given by
\begin{equation*}
%    \label{eq:def:perverse:bbd}
    \begin{aligned}
        \perv D_{\setset S}^{\le 0}(X) & = \{ \sheaf F ∈ D^b_{\setset S}(X) : H^n(i_S^*\sheaf F) = 0 \text{ for all $n > -\frac12 \dim S$ and all $S ∈ \setset S$}\}; \\
        \perv D_{\setset S}^{\ge 0}(X) & = \{ \sheaf F ∈ D^b_{\setset S}(X) : H^n(i_S^!\sheaf F) = 0 \text{ for all $n < -\frac12 \dim S$ and all $S ∈ \setset S$}\}.
    \end{aligned}
\end{equation*}
In Section~\ref{sec:constructible_and_Lagrangian} we will make Theorem~\ref{thm:motivation} precise and give a proof.
In Section~\ref{sec:constructible_main_thm} we will formulate and prove a precise analogue of Theorem~\ref{thm:main} in the constructible setting.

\section{Perverse sheaves and Lagrangians}\label{sec:constructible_and_Lagrangian}

This section follows the exposition of \cite[Section 2.2]{Jin:arXiv:HolomorphicLagrangianBranesCorrespondToPerverseSheaves}.

We let
\[
    Λ_{\setset S} = \bigcup_{S ∈ \setset S} T^*_S X \subseteq T^*X
\]
be the standard conical Lagrangian associated to the stratification $\setset S$.

\begin{Def}[{\cite[Definition~4.2.7]{Dimca:2004:SheavesInTopology}}]
    Let $\setset S$ be a Whitney stratification of $X$ and $f\colon X → ℂ$ holomorphic.
    The \emph{stratified critical locus} of $f$ is
    \[
        Σ_{\setset S}(f) =
        \bigcup_{X_α ∈ \setset S} Σ(\res{f}{X_α}),
    \]
    where $Σ(\res{f}{X_α})$ is the (ordinary) critical locus of the restriction of $f$ to $X_α$.
\end{Def}

In the following we call a triple $(x, ξ, F)$ consisting of a smooth point $(x, ξ) ∈ Λ_{\setset S}$ and the germ of a holomorphic function $F$ at $x$ such that $F(x) = 0$, $d(\Re F)_x = ξ$\todo{Complex vs.~real cotangent space?}\ and the graph $Γ_{d(\Re F)}$ is transverse to $Λ_{\setset S}$ at $(x, ξ)$ a \emph{test triple}\todo{$F$ should have an isolated singularity at $x$.}.
We will assume that $F$ is defined on a ball $B_{2ε}(x)$ and with $ε$ small enough that $B_ε(x) ∩ Σ_{\setset S}(F) = \{x\}$.

\begin{Rem}
    If $x ∈ S$ then a covector $(x, ξ)$ is called \emph{degenerate} if it is contained in 
    \[
        D^*_SX = T^*_SX ∩ \bigcup_{S' \ne S} \overline{T^*_{S'} X}.
    \]
    The smooth locus of $Λ_{\setset S}$ consists exactly of the non-degenerate covectors\todo{Why?}.\todo{Why the transversality condition?}
\end{Rem}

Recall that for any test triple $(x,ξ,F)$ there is a vanishing cycles functor 
\[
    φ_F\colon D_{\setset S}(B_ε(x)) → \dconstrb{F^{-1}(0) ∩ B_ε(x)}.
\]
Note that for any $\sheaf F ∈ D_{\setset S}(B_ε(x))$ the sheaf $φ_F \sheaf F$ is supported at $x$ \cite[Proposition~4.2.7]{Dimca:2004:SheavesInTopology}\todo{Or (without proof) \cite[(8.6.12)]{KashiwaraSchapira:1994:SheavesOnManifolds}}.
%\begin{Lem}[{\cite[Proposition~4.2.7]{Dimca:2004:SheavesInTopology}}]
%    \label{lem:constructible:supp_of_vc_in_critical_locus}%
%    Let $\sheaf F$ be a constructible sheaf on $X$ with respect to the  Whitney stratification $\setset S$.
%    Then 
%    \[
%        \supp \pervVC{f}\sheaf F \subseteq Σ_{\setset S}(f).
%    \]
%\end{Lem}

\begin{Def}
    Let $(x, ξ, F)$ be a test triple.
    The \emph{local Morse group functor} (or \emph{microlocal stalk functor}) is
    \[
        \localMorse{x}{F}{\sheaf F} = ι_x^* \pervVC{F} j^* {\sheaf F}\colon D^b_{\setset S}(X) → D^b(ℂ),
    \]
    where $j$ is the inclusion $B_ε(x) \hookrightarrow X$.
\end{Def}
%ToDo: Note about microlocal stalks and singular support.

Using standard facts about the vanishing cycles functor\todo{reference?}\ it is easy to see that the local Morse group functor is t-exact from the perverse t-structure on $D^b_{\setset S}(X)$ to the standard t-structure on $D^b(ℂ)$.
The following theorem shows that moreover it can be used to test whether a given sheaf is perverse.

% ToDo: Proof or reference
%\begin{Thm}[{\cite[Corollary~10.3.13]{KashiwaraSchapira:1994:SheavesOnManifolds}}]
%    Let $f\colon X → ℂ$ be a holomorphic function.
%    Then the functor $\pervVC{f}$ is t-exact with respect to the t-structure given by the middle perversity.
%\end{Thm}

\begin{Thm}[{\cite[Proposition 2.9]{Jin:arXiv:HolomorphicLagrangianBranesCorrespondToPerverseSheaves}}]
    For each stratum $S ∈ \setset S$ choose a test triple $(x_S, ξ_S, F_S)$ with $(x_S , ξ_S) ∈ T^*_S X$.
    Then a sheaf $\sheaf F ∈ D^b_{\setset S}(X)$ is perverse if and only $\localMorse{x_S}{F_S}{\sheaf F}$ is concentrated in cohomological degree $0$ for all $S ∈ \setset S$.
\end{Thm}
% ToDo: read proof

The following lemma is supposedly standard.
\begin{Lem}
    Let $(x, ξ, F)$ be a test triple and $\sheaf F ∈ D^b_{\setset S}(X)$.
    Then there are (canonical?) isomorphisms
    \[
        \localMorse{x}{F}{\sheaf F} \cong
        Γ\bigl(B_ε(x),\, B_ε(x) ∩ F^{-1}(t),\, \sheaf F\bigr) \cong
        Γ\bigl(B_ε(x),\, B_ε(x) ∩ \{\Re F < μ\},\, \sheaf F\bigr)
    \]
    for $0 < \abs{t} \ll 1$ and $μ \le 0$ with $\abs{μ} \ll 1$.
\end{Lem}
% ToDo: proof/reference

Next: Unstable manifold.

\section{Constructible version of Theorem~\ref{thm:main}}\label{sec:constructible_main_thm}

We return now to the claim about exact functors on the t-structure of constructible perverse sheaves made in the introduction.
For simplicity we restrict to the complex (i.e.~middle perversity) case.
Thus let $X$ be a complex manifold and $\setset S$ a finite stratification of $X$ by complex submanifolds.
We write $D^b_{\setset S}(X)$ for the bounded derived category of $\setset S$-constructible sheaves on $X$.
We call a sheaf $\sheaf F ∈ \smash[t]{D^b_{\setset S}(X)}$ perverse if it is perverse with respect to the middle perversity function on $\setset S$.
We are going to formulate and prove an analog of Theorem~\ref{thm:main} in this situation.

A closed real submanifold $Z$ of $X$ is called a \emph{measuring submanifold} if for each stratum $S$ of $X$ either $Z ∩ \overline S = \emptyset$ or $\dim_ℝ Z ∩ S = \dim_ℂ S$.
A \emph{measuring family} is a collection of measuring submanifolds $\{ Y_i \}$ such that each connected component of each stratum has non-empty intersection with at least one $Y_i$.
Similarly to Theorem~\ref{thm:existence_of_lcimeasuring}, one shows inductively that such a collection of submanifolds always exists.

\begin{Thm}
    Let $\measuringFam$ be a measuring family of submanifolds of $X$.
    A sheaf $\sheaf F ∈ D^b_{\setset S}(X)$ is perverse if and only if $ι_Z^! \sheaf F$ is concentrated in cohomological degree $0$ for each submanifold $Z ∈ \measuringFam$.
\end{Thm}

The proof of the following lemma is based on a MathOverflow post by Geordie Williamson \cite{MO:VanishingShriekRestrictionConstructible}.
The author takes responsibility for possible mistakes.

\begin{Lem}\label{lem:constructible_local_vanishing}%
    Let $X$ be a real manifold, $\sheaf F$ be a constructible sheaf (concentrated in degree $0$) on $X$ and let $i\colon Z \hookrightarrow X$ be the inclusion of a closed submanifold.
    Then $H^j(i^!\sheaf F) = 0$ for $j>\codim_XZ$.
\end{Lem}

\begin{proof}
    By taking normal slices we can reduce to the case that $Z = \{z\}$ is a point.
    Let $j$ be the inclusion of $X \setminus \{z\}$ into $X$ and consider the distinguished triangle
    \[
        i_!i^! \sheaf F → \sheaf F → j_*j^*\sheaf F \xrightarrow{+1}.
    \]
    By~\cite[Lemma~8.4.7]{KashiwaraSchapira:1994:SheavesOnManifolds} we have
    \[
        H^j(j_*j^*\sheaf F)_z = H^j(S^{\dim X - 1}_ε, \sheaf F)
    \]
    for a sphere $S^{\dim X - 1}_ε$ around $z$ of sufficiently small radius.
    The latter cohomology vanishes for $j \ge \dim X$ and hence $H^j(i^! \sheaf F) = 0$ for $j>\dim X$ as required.
\end{proof}

\begin{proof}[Proof of Theorem]
    Define two full subcategories $\perv[L] D^{\le 0}(X)$ and $\perv[L] D^{\ge 0}(X)$ of $D^b_{\setset S}(X)$ by
    \begin{align*}
        \perv[L] D^{≤0}(X) & = \left\{ \sheaf F ∈ D^b_{\setset S}(X) : ι_Z^! \sheaf F ∈ D^{≤0}(Z) \text{ for all $Z ∈ \measuringFam$} \right\}, \\
        \perv[L] D^{≥0}(X) & = \left\{ \sheaf F ∈ D^b_{\setset S}(X) : ι_Z^! \sheaf F ∈ D^{≥0}(Z) \text{ for all $Z ∈ \measuringFam$} \right\}.
    \end{align*}
    We will show that these categories are the same as the categories $\perv D^{≤0}(X)$ and $\perv D^{≥0}(X)$ defining the perverse t-structure on $D^b_{\setset S}(X)$.

    We induct on the number of strata.
    If $X$ consists of only one stratum and $Z$ is a measuring submanifold, then $ι_Z^! \sheaf F \cong ω_{Z/X} \otimes ι_Z^* \sheaf F$ and hence $ι_Z^! \sheaf F$ is in degree $0$ if and only if $\sheaf F$ is in degree $-\frac 12 \dim_ℝ X$.
    So assume that $X$ has more then one stratum.
    Without loss of generality we can assume that $X$ is connected.
    Let $U$ be the union of all open strata and $F$ its complement.
    Both $U$ and $F$ are unions of strata of $X$.
    Let $j$ be the inclusion of $U$ and $i$ the inclusion of $X$. 
    
    \begin{itemize}
        \item 
            If $\sheaf F ∈ \perv D^{≤0}(X)$, then $\sheaf F ∈ \perv[L] D^{≤0}$ follows in exactly the same way as in the coherent case, using Lemma~\ref{lem:constructible_local_vanishing}.
        \item 
            Let $\sheaf F ∈ \perv D^{≥0}(X)$.
            Then $i^! \sheaf F ∈ \perv D^{≥0}(F)$ and $j^*\sheaf F ∈ \perv D^{≥0}(U)$.
            Let $Z$ be a measuring subvariety.
            Consider the distinguished triangle 
            \[ 
                i_*i^! \sheaf F → \sheaf F → j_*j^*\sheaf F.
            \]
            Using base change, induction and the (left)-exactness of the push-forward functors one sees that $ι_Z^!$ of the outer sheaves in the triangle are concentrated in non-negative degrees.
            Thus so is $ι_Z^! \sheaf F$.
        \item 
            Let $\sheaf F ∈ \perv[L] D^{≥0}(X)$.
            Since all measurements are local this implies that $j^* \sheaf F ∈ \perv[L] D^{≥0}(U) = \perv D^{≥0}(U)$.
            Using the same triangle and argument as in the last point, this implies that also $i^! \sheaf F ∈ \perv[L] D^{≥0}(F) = \perv D^{≥0}(F)$.
            Hence, by recollement, $\sheaf F ∈ \perv D^{≥0}(X)$.
        \item 
            Finally, let $\sheaf F ∈ \perv[L] D^{≤0}(X)$.
            Again this immediately implies that $j^* \sheaf F ∈ \perv[L] D^{≤0}(U) = \perv D^{≤0}(U)$.
            Thus $j_!j^*\sheaf F ∈ \perv D^{≤0}(X)$.
            Let $Z$ be a measuring submanifold and consider the distinguished triangle
            \[
                ι_Z^! j_!j^*\sheaf F → ι_Z^! \sheaf F → ι_Z^! i_*i^* \sheaf F.
            \]
            By what we already know, the first sheaf is concentrated in non-positive degrees and hence so is $ι_Z^! i_*i^* \sheaf F$.
            By base change and the exactness of $i_*$ this implies that $i^* \sheaf F ∈ \perv[L] D^{≤0}(F) = \perv D^{≤0}(F)$.
            Hence, by recollement, $\sheaf F ∈ \perv D^{≤0}(X)$.
            \qedhere
    \end{itemize}
\end{proof}

\begin{Rem}
    The equality $\perv D^{≥0}(X) = \perv[L]D^{≥0}(X)$ could also be proved in exactly the same way as in the coherent case, using~\cite[Exercise~\textsc{x}.10]{KashiwaraSchapira:1994:SheavesOnManifolds}.
\end{Rem}


\part{Hochschild cohomology of D-modules on torus quotient stacks}

\chapter{Prerequisites}%
\label{ch:d-mod:pre}

We fix an algebraically closed base field $k$ of characteristic $0$.
All stacks in this thesis are assumed to be algebraic QCA stacks over $k$.
As we will summarize in Section~\ref{sec:d-mod:pre:d-mods}, the QCA condition ensures that the category of D-modules on stacks is well-behaved.
In particular for any stack $\stack X$ we have:
\begin{itemize}
    \item The diagonal morphism $Δ\colon \stack X → \stack X × \stack X$ is schematic.
    \item There exists a scheme $Z$ with a smooth and surjective map $Z → \stack X$.
    \item $\stack X$ is quasi-compact.
    \item The automorphism groups of the geometric points of $\stack X$ are affine.
    \item The loop space (or inertia stack) $\ls \stack X = \stack X ×_{\stack X × \stack X} \stack X$ is of finite presentation over $\stack X$.
\end{itemize}
The first two conditions ensure that the stack is algebraic, the other three that it is quasi-compact with affine automorphism group (QCA).
For details on QCA stacks we refer to~\cite{DrinfeldGaitsgory:2013:FinitenessQuestions}.
Every quotient of a scheme of finite type over $k$ by an affine algebraic group is a QCA stack, and we will be mainly interested in these.

In order to correctly define categories of D-modules on stacks it is necessary to work with dg-categories.
We refer to \cite{Keller:2006:OnDGCategories} for an introduction to dg categories.
It is often convenient to regard (pretriangulated) dg categories as $k$-linear stable $(∞,1)$-categories \cite{Lurie:2009:HigherToposTheory,Lurie:2014-draft:HigherAlgebra}, which can be done via the nerve construction \cite{Cohn:arXiv:DGCategoriesAreStableInfinityCategories,Faonte:arXiv:SimplicialNerveOfAnAinfinityCategory}. %\cite[Section~1.3.1]{Lurie:2014-draft:HigherAlgebra}
We will switch between those two languages without explicitly mentioning the intervening constructions and apply results from \cite{Lurie:2014-draft:HigherAlgebra} to dg categories.
Fortunately, a superficial knowledge of dg/$∞$-categories should be sufficient for reading this thesis.

\section{D-modules on stacks}
\label{sec:d-mod:pre:d-mods}

The second part of this thesis is primarily concerned with D-modules on quotient stacks.
Unfortunately there is currently no comprehensive text available that covers all the basic constructions and properties of D-modules on stacks (or even the \emph{dg category} of D-modules on schemes).
Thus we collect all the relevant properties (without proof) in this section.
The upshot is that the familiar \enquote{six functors formalism} essentially works for holonomic D-modules and schematic morphisms of stacks.

The category of D-modules on a stack $\stack X$ can be either constructed via descent \cite{BeilinsonDrifeld:unpublished:Hitchin,DrinfeldGaitsgory:2013:FinitenessQuestions} or equivalently as ind-coherent sheaves on the de Rham space of $\stack X$ \cite{GaitsgoryRozenblyum:2014:CrystalsAndDModules}.
While the first construction is more \enquote{hands on}, the second construction is often more useful from a theoretical point of view.
It is explained in detail in the upcoming book \cite{GaitsgoryRozenblyum:prelim:StudyInDAG} (see also \cite{FrancisGaitsgory:2012:ChiralKoszulDuality} for an overview).
Many basic properties of the category $\catDMod{\stack X}$ are explored in \cite{DrinfeldGaitsgory:2013:FinitenessQuestions} and most of the following assertions are taken from there.

For any morphism $f\colon \stack X → \stack Y$ the constructions yield a continuous functor $f^!\colon \catDMod{\stack Y} → \catDMod{\stack X}$ and (after some work) a not necessarily continuous functor $f_*\colon \catDMod{\stack X} → \catDMod{\stack Y}$.
If $p\colon \stack X → \pt$ is the structure map then we set
\[
    \ΓdR(\stack X,\, {-}) = p_*({-}) \colon \catDMod{\stack X} → \catVect.
\]
The functor $\ΓdR$ is representable by a D-module $k_{\stack X}$, i.e.
\[
    \ΓdR(\stack X,\, {-}) = \Hom_{\catDMod{\stack X}}(k_{\stack X},\, {-}).
\]
Again we note that $\ΓdR(\stack X,\, {-})$ is usually not continuous and hence the object $k_{\stack X}$ not compact.

Let $Δ\colon \stack X → \stack X × \stack X$ be the diagonal.
The category $\catDMod{\stack X}$ has a monoidal structure given by the tensor product
\[
    \sheaf F \otimes \sheaf G = Δ^!\bigl( \sheaf F \boxtimes \sheaf G \bigr).
\]
The unit for this monoidal structure is $ω_{\stack X} = p^! k$.

We will be mainly concerned with the subcategory of holonomic D-modules since they enjoy extended functoriality.
\begin{Def}
    A $D$-module $\sheaf F ∈ \catDMod{\stack X}$ is called \emph{holonomic} if $f^!\sheaf F$ is holonomic for any smooth morphism $f\colon Z → \stack X$ from a scheme $Z$.
    The full subcategory of holonomic D-modules will be denoted $\catDModHol{\stack X}$.
\end{Def}

The following assertions mostly follow from their corresponding counterparts for schemes.
We refer to \cite{Braverman:LecturesOnAlgebraicDmodules} for proofs in the case on non-smooth schemes.

\begin{Prop}
    Let $f\colon \stack X → \stack Y$ be a schematic morphism.
    Then $f^!$ and $f_*$ restrict to functors
    \[
        f^!\colon \catDModHol{\stack Y} → \catDModHol{\stack X}
        \quad\text{and}\quad
        f_*\colon \catDModHol{\stack X} → \catDModHol{\stack Y}.
    \]
\end{Prop}

The Verdier duality functor on schemes induces an involutive anti auto-equivalence
\[
    \dualize_{\stack X}\colon \catDModHol{\stack X}^\mathrm{op} → \catDModHol{\stack X}
\]
such that for each smooth morphism $Z → \stack X$ of relative dimension $d$ from a scheme $Z$ one has
\[
    f^! ∘ \dualize_{\stack X} \cong \dualize_{Z} ∘ f^![-2d].
\]
The Verdier duality functor then allows us to define the \emph{non-standard functors} $f_!$ and $f^*$ for any schematic morphism $f\colon \stack X → \stack Y$ by
\begin{align*}
    f^* & = \dualize_{\stack X} ∘ f^! ∘ \dualize_{\stack Y} \colon \catDModHol{\stack Y} → \catDModHol{\stack X} \\
    \intertext{and}
    f_! & = \dualize_{\stack Y} ∘ f_* ∘ \dualize_{\stack X} \colon \catDModHol{\stack X} → \catDModHol{\stack Y}.
\end{align*}
We obtain adjoint pairs $(f_!,\, f^!)$ and $(f^*,\, f_*)$.
In particular we have
\[
    k_{\stack X} = f^* k_{\stack Y}.
\]
In some situations we can identify the non-standard functors with their standard counterparts.
If $f$ is smooth of relative dimension $d$ then $f^* = f^![-2d]$.
In particular, if $\stack X$ is smooth then $k_{\stack X} = ω_{\stack X}[-2\dim \stack X]$.
If $f$ is proper then $f_! = f_*$ and in particular $f_*$ is left adjoint to $f^!$.
The objects $ω_{\stack X}$ and $k_{\stack X}$ are always holonomic and
\[
    \dualize_{\stack X} ω_{\stack X} = k_{\stack X}.
\]

We will make use of the following lemma which follows from \cite[Lemma~5.1.6]{DrinfeldGaitsgory:2013:FinitenessQuestions}.

\begin{Lem}
    For a smooth and schematic morphism $f$ the functor $f^!$ is conservative.
\end{Lem}

\begin{Prop}[{\cite[\RomanNum{III}.4.2.1.3]{GaitsgoryRozenblyum:prelim:StudyInDAG}}]
    \label{prop:d-mod:pre:base-change}%
    Consider a Cartesian square
    \[
        \begin{tikzcd}
            \stack Z \arrow[d, "p"] \arrow[r, "q"] & \stack X₁ \arrow[d, "f"] \\
            \stack X₂ \arrow[r, "g"] & \stack Y
        \end{tikzcd}
    \]
    with schematic morphism $f$ (and hence $p$).
    Then there is a base change isomorphism
    \[
        p_* q^! \isoto g^! f_*
    \]
    of functors from $\catDMod{\stack{X₁}}$ to $\catDMod{\stack{X₂}}$.
    If furthermore $f$ (and hence $p$) is proper, then this isomorphism coincides with the natural transformation
    \[
        p_* q^! →
        p_* q^! f^! f_* =
        p_* p^! g^! f_* →
        g^! f_*
    \]
    induced by $(f_*,\,f^!)$ and $(p_*,\, p^!)$ adjunctions.
\end{Prop}

\begin{Prop}
    \label{prop:d-mod:pre:projection-formula}%
    If $f\colon \stack X → \stack Y$ is a schematic morphism then the projection formula holds, i.e.~there is a functorial isomorphism
    \[
        \sheaf F \otimes f_*(\sheaf G) \cong f_*\bigl( f^! \sheaf F \otimes \sheaf G)
    \]
    for $\sheaf F ∈ \catDMod{\stack Y}$ and $\sheaf G ∈ \catDMod{\stack X}$.
\end{Prop}

\begin{Rem}
    Propositions \ref{prop:d-mod:pre:base-change} and \ref{prop:d-mod:pre:projection-formula} hold more generally when $f$ is merely a \enquote{safe} morphism.
    Alternatively they hold in full generality after replacing $f_*$ by the \enquote{renormalized de Rham pushforward}.
    We fill not use either notion in this thesis and refer the interested reader to \cite{DrinfeldGaitsgory:2013:FinitenessQuestions}.
\end{Rem}

For D-modules on stacks we have the usual recollement package.
Let $i\colon \stack Z \hookrightarrow \stack X$ be a closed embedding and $j\colon \stack U \hookrightarrow \stack X$ the complementary open.
We have adjoint pairs $(i_*,\, i^!)$ and $(j^!,\, j_*)$.

\begin{Prop}[{\cite[Section~2.5]{GaitsgoryRozenblyum:2014:CrystalsAndDModules}}]
    \label{prop:d-mod:recollement-std}%
    There is an exact triangle of functors
    \[
        i_*i^! → \id → j_*j^!
    \]
    on $\catDMod{\stack X}$, the adjunction morphisms
    \[
        \id → i^!i_*
        \quad\text{and}\quad
        j^!j_* → \id
    \]
    are isomorphisms, the functors $j^!i_*$ and $i^!j_*$ vanish and $i_*$ and $j_*$ are full embeddings.
\end{Prop}

On holonomic D-modules we have the additional adjoint pairs $(i^*,\, i_*)$ and $(j_!,\, j^*)$.
By applying duality to Proposition~\ref{prop:d-mod:recollement-std} we obtain the distinguished triangle
\[
    j_! j^* → \id → i_*i^*
\]
and the identity $i^*j_! = 0$ on holonomic D-modules.
Further, the functor $j_!$ is a full embedding $\catDModHol{\stack U} \hookrightarrow \catDModHol{\stack X}$.

It is often useful to consider the pullback of a D-module on $\stack X$ to a smooth cover.

\begin{Def}
    \label{def:d-mod:pre:monodromic}%
    Let $X$ be a scheme with an action of an algebraic group $G$ and let $p\colon X → X/G$ be the quotient map.
    The \emph{monodromic} subcategory $\catDModMon{X}{G} ⊆ \catDMod{X}$ is the full subcategory generated by the essential image of $p^!\colon \catDMod{X/G} → \catDMod{X}$ (or equivalently by the essential image of $p^*$).
\end{Def}

\begin{Thm}[Contraction principle~{\cite[Proposition~3.2.2]{DrinfeldGaitsgory:2014:OnATheoremOfBraden}}]
    \label{thm:d-mod:pre:contraction_principle}%
    Let $X$ be a scheme with an action by $\Gm$ that extends to an action of the monoid $\as 1$.
    Let $i\colon X^0 \hookrightarrow X$ be the closed subscheme of $\Gm$-fixed points and let $π\colon X → X^0$ be the contraction morphism induced by the $\Gm$-equivariant morphism $\as 1 → \{0\}$.
    Then there is an isomorphism of functors
    \[
        i^* \cong π_* \colon \catDModMon{X}{\Gm} → \catDMod{X^0}.
    \]
\end{Thm}


\section{Monads}
\label{sec:d-mod:pre:monads}

We will deduce Theorem~\ref{thm:d-mod:main} from a isomorphism of monads on $\catDMod{\stack X}$.
In this section we give a short introduction to the theory of monads and the specific constructions that we will use.
However, in the interest of readability we will mainly do so informally, skipping over the intricacies of $∞$-categories.
The interested reader can find the correct $∞$-categorical formulations in the given references.

Thus we think of a \emph{monad} on a category $\cat C$ as consisting of a triple $(T, η, μ)$, where $T\colon \cat C → \cat C$ is an endofunctor of $\cat C$, and $η\colon \id_{\cat C} → T$ and $μ\colon T∘T → T$ are natural transformations that the diagrams
\begin{equation}
    \label{eq:d-mod:pre:monad-identities}
    \begin{tikzcd}
        T³ \arrow[r, "Tμ"] \arrow[d, "μT"'] & T² \arrow[d, "μ"] \\
        T² \arrow[r, "μ"'] & T
    \end{tikzcd}
    \quad\text{and}\quad
    \begin{tikzcd}
        T \arrow[r, "Tη"] \arrow[d, "ηT"'] \arrow[dr, equal] & T² \arrow[d, "μ"] \\
        T² \arrow[r, "μ"'] & T
    \end{tikzcd}
\end{equation}
commute.
Alternatively, we can think of $T$ being a monoid in the category of endofunctors of $\cat C$ with the monoidal structure given by composition of endofunctors.
This definition also gives the correct generalization to $∞$-categories \cite[Definition~4.7.0.1]{Lurie:2014-draft:HigherAlgebra}.

Let $X$ be an object of the category $\cat C$.
Then $T$ gives the vector space $\Hom_{\cat C}(X, TX)$ the structure of a dg algebra with multiplication map
\[
    (f,g) \mapsto μ_X ∘ Tf ∘ g,
\]
\[
    \begin{tikzcd}
        X \arrow[r, "g"] & TX \arrow[r, "Tf"] & T²X \arrow[r, "μ_X"] & TX.
    \end{tikzcd}
\]
The identities~\ref{eq:d-mod:pre:monad-identities} ensure that the algebra is associative and unital.

The most common source of monads is from a pair of adjoint functors $F\colon \cat C \rightleftarrows \cat D \cocolon G$.
One simply sets $T = G ∘ F$ and $η$ and $μ$ are given by the adjunction morphisms
\[
    \id[\cat C] → G ∘ F = T
    \qquad\text{and}\qquad
    T² = G ∘ (F ∘ G) ∘ F → G ∘ F = T.
\]
We note that the correct construction in more complicated in the $∞$-categorical case and refer to \cite[Section~4.7]{Lurie:2014-draft:HigherAlgebra}.
For any $X ∈ \cat C$ the algebra construction above gives an isomorphism of algebra
\[
    \Hom_{\cat C}(X, (GF)(X)) \cong
    \Hom_{\cat D}(FX, FX).
\]

Another common way to obtain monads in geometry is via a groupoid.
Recall that a groupoid $G_{\cx}$ in stacks consists of a stack $\stack G₀$ of \enquote{objects} and a stack $\stack G₁$ of \enquote{morphisms} together with
\begin{itemize}
    \item \emph{source} and \emph{target} maps $s,t\colon \stack G₁ \rightrightarrows \stack G₀$,
    \item a \emph{unit} $e\colon G₀ → G₁$,
    \item a \emph{multiplication} (or \emph{composition}) map $m\colon \stack G₁ ×\limits_{s,\stack G₀,t} \stack G₁ → \stack G₁$),
    \item an \emph{inverse} map $ι\colon \stack G₁ → \stack G₁$,
\end{itemize}
such that
\begin{itemize}
    \item $s ∘ e = t ∘ e = \id_{\stack G₀}$,
    \item $s ∘ m = s ∘ p₂$ and $t ∘ m = t ∘ p₁$ (where $p_i\colon \stack G₁ ×_{s,\stack G₀,t} \stack G₁$ are the projection maps).
    \item $m$ is associative,
    \item $ι$ interchanges $s$ and $t$ and is an inverse for $m$,
\end{itemize}
where all identities have to be understood in the correct $∞$-categorical way \cite[Section~6.1.2]{Lurie:2009:HigherToposTheory}.

\begin{Ex}
    For our purpose the most important example is the following:
    Let $f\colon \stack X → \stack S$ be a morphism of stacks.
    We set $\stack G_0 = \stack X$ and $\stack G₁ = \stack X ×_\stack S \stack X$.
    The source and target maps are given by $p₁$ and $p₂$, the unit by the diagonal $Δ\colon \stack X → \stack X×_\stack S\stack X$, the inverse by interchanging the factors and multiplication is $p₁₃\colon \stack X ×_\stack S \stack X ×_\stack S \stack X → \stack X×_\stack S\stack X$.
\end{Ex}

Let us for the moment assume that $s$ (and hence $t$) is proper and schematic.
In this case the maps $e$ and $m$ are also proper, since $s ∘ e = \id_{\stack G₀}$ and $s ∘ m = s ∘ p₂$ are proper.
In particular the functors $e^!\colon \catDMod{\stack G₁} → \catDMod{\stack G₀}$ and $m^!\colon \catDMod{\stack G₁} → \catDMod{\stack G₁ ×_{\stack G₀} \stack G₁}$ have left adjoints given by $e_*$ and $m_*$ respectively.
This allows us to give the endofunctor $T = s_*t^!$ of $\catDMod{\stack G₀}$ the structure of a monad in the following way:

\begin{itemize}
    \item By $(e_*,e^!)$-adjunction we have a transformation
        \[
            \id = (s∘e)_*(t∘e)^! = s_*e_*e^!t^! → s_*t^! = T.
        \]
    \item Consider the following commutative diagram
        \[
            \begin{tikzcd}[column sep=small]
                {} & & \stack G₁ \arrow[dddll, bend right, "s"] \arrow[dddrr, bend left, "t"] & & \\
                & & \stack G₁ ×_{\stack G₀} \stack G₁ \arrow[u, "m"] \arrow[dl, "p₂"] \arrow[dr, "p₁"] & & \\
                & \stack G₁ \arrow[dl, "s"] \arrow[dr, "t"] & & \stack G₁ \arrow[dl, "s"] \arrow[dr, "t"] & \\
                \stack G₀ & & \stack G₀ & & \stack G₀
            \end{tikzcd}
        \]
        with Cartesian middle square.
        Proper base change and $(m_*,m^!)$-adjunction gives a transformation
        \[
            T² =
            s_*t^!s_*t^! =
            (s∘p₂)_*(t∘p₁)^! =
            (s∘m)_*(t∘m)^! =
            s_*m_*m^!t^! →
            s_*t^! =
            T.
        \]
\end{itemize}

In the non-$∞$-categorical setting one could easily check by hand that this is indeed a monad.
To obtain the corresponding derived statement one applies an argument similar to \cite[Section~\RomanNum{II}.1.7.2]{GaitsgoryRozenblyum:prelim:StudyInDAG}.
We will discuss a version of this below.

Let now $f\colon \stack X → \stack Y$ be schematic and proper.
The Cartesian diagram
\[
    \begin{tikzcd}
        \stack X ×_{\stack Y} \stack X \arrow[r, "p_s"] \arrow[d, "p_t"] & \stack X \arrow[d, "f"] \\
        \stack X \arrow[r, "f"] & \stack Y
    \end{tikzcd}
\]
induces a groupoid with $\stack{G}₀ = \stack X$ and $\stack{G}₁ = \stack X ×_{\stack Y} \stack X$.
The above constructions now give two monads on $\catDMod{\stack{X}}$: one by $(f_*,f^!)$ adjunction and one from the groupoid structure.
The base change isomorphism
\[
    p_{t,*} p_s^! → f^! f_*
\]
gives an identification of these monads and hence of the algebras that they induce, i.e.~for any $\sheaf F ∈ \catDMod{\stack X}$ we have
\[
    \Hom(\sheaf F,\, p_{t,*} p_s^!\sheaf F) \cong
    \Hom(\sheaf F,\, f^! f_* \sheaf F) \cong
    \Hom(f_*\sheaf F,\, f_* \sheaf F).
\]

We will need to apply this construction for non-proper $f$.
Unfortunately, in this case none of the adjunctions used to define the monads are available.
We rectify this by restricting to the full subcategory of of holonomic D-modules and using the $!$-pushforward functors instead of the $*$-pushforward ones.
Of course, by doing so we do not automatically have base change isomorphisms available anymore.
Thus we have to explicitly require that all necessary base changes hold (this is usually called the Beck-Chevalley condition).

In order to formulate the condition, we need the concept of the \emph{nerve} of a groupoid.
This is the simplicial stack, also denoted $\stack{G}_\cx$, with
\[
    \stack G_i = \underbrace{\stack G₁ ×_{\stack G₀} \dotsb ×_{\stack G₀} \stack G₁}_{\text{$i$ factors}}.
\]
We refer to \cite[Section~6.1.2]{Lurie:2009:HigherToposTheory} for the correct $∞$-categorical setup.
The following lemma is now an immediate corollary of \cite[Lemma~\RomanNum{II}.1.7.1.4]{GaitsgoryRozenblyum:prelim:StudyInDAG} or \cite[Theorem~4.7.6.2]{Lurie:2014-draft:HigherAlgebra}.

\begin{Lem}
    \label{lem:d-mod:pre:groupoid_monad_hol}%
    Let $f\colon \stack X → \stack Y$ be a schematic morphism of stacks and let $\stack G_\cx$ be the corresponding groupoid.
    For each map $F\colon [n] → [m]$ in $\cat{Δ}^{\mathrm{op}}$ consider the corresponding square
    \[
        \begin{tikzcd}
            \stack G_{n+1} \arrow[r, "p_s"] \arrow[d, "p_{F+1}"] & \stack G_n \arrow[d, "p_F"] \\
            \stack G_{m+1} \arrow[r, "p_s"] & \stack G_m
        \end{tikzcd}
    \]
    where the vertical arrows are induced by $F$.
    Assume that for each such square the base change composition
    \[
        p_{F+1,!} p_s^! →
        p_{F+1,!} p_s^! p_F^! p_{F,!} =
        p_{F+1,!} p_{F+1}^! p_s^!  p_{F,!} →
        p_s^! p_{F,!}
    \]
    given by the adjunction morphisms is an isomorphism of functors $\catDModHol{\stack G_n} → \catDModHol{\stack G_{m+1}}$.
    Then the endofunctor $p_{t,!} p_s^!$ of $\catDModHol{\stack X}$ has a canonical structure of a monad and as such is isomorphic to the adjunction monad $f^!f_!$.
\end{Lem}

\section{Hochschild cohomology}

We recall that the Hochschild cohomology of a dg category $\cat C$ is the algebra of derived endomorphisms of the identity functor,
\[
    \HCoh(\cat C) = \mathbb{R}\mathrm{Hom}(\id[\cat C],\, \id[\cat C]).
\]
For the exact definition of the category $\mathbb{R}\mathrm{Hom} = \cat{Funct}(\cat C, \cat C)$ we refer to \cite{Keller:2006:OnDGCategories}.
Instead we will give a more concrete construction via kernels which can be applied to $\catDMod{\stack X}$.
For this let us restrict to the case of co-complete dg categories and let $\cat{Funct}_{\mathrm{cont}}(\cat C, \cat C)$ be the full subcategory of $\cat{Funct}(\cat C, \cat C)$ spanned by the continuous functors.
Then, since $\id[\cat C]$ is evidently continuous, we have
\[
    \HCoh(\cat C) =
    \Hom_{\cat{Funct}_{\mathrm{cont}}(\cat C, \cat C)}(\id[\cat C], \id[\cat C]).
\]
Let us further assume that $\cat C$ is dualizable with dual $\cat C^\dual$.
Thus there is a unit map
\[
    η\colon \catVect → \cat C^\dual \otimes \cat C
\]
and a counit map
\[
    ε\colon \cat C^\dual \otimes \cat C → \catVect
\]
fulfilling the usual compatibilities (cf.~\cite[Section~2]{BenZviNadler:arXiv:NonlinearTraces}).
Let $u = η(k)$.
Then to each continuous endofunctor $F$ of $\cat C$ we can associate its kernel $\id[\cat C^\dual] \otimes F(u) ∈ \cat C^\dual \otimes \cat C$ and conversely to each kernel $Q ∈ \cat C^\dual \otimes \cat C$ we can associate the endofunctor
\[
    \cat C
    \xrightarrow{\id[\cat C] \otimes Q}
    \cat C \otimes \cat C^\dual \otimes \cat C
    \xrightarrow{ε \otimes \id[\cat C]}
    \cat C.
\]
These assignments are mutually inverse and give an equivalence of dg categories
\[
    \cat{Funct}_{\mathrm{cont}}(\cat C, \cat C)
    \cong
    \cat C^\dual \otimes \cat C.
\]
In particular, the kernel for the identity is $u$ and hence we have
\[
    \HCoh(\cat C) =
    \Hom_{\cat C^\dual \otimes \cat C}(u, u).
\]
Let us now consider the case of $\cat C = \catDMod{\stack X}$ for a stack $X$.
Let $p\colon \stack X → \pt$ be the structure morphism and $Δ\colon \stack X → \stack X × \stack X$ the diagonal.
By \cite[Section~8.4]{DrinfeldGaitsgory:2013:FinitenessQuestions} the category $\catDMod{\stack X}$ is dualizable and there is a canonical indentification
\[
    \catDMod{\stack X}^\dual \otimes \catDMod{\stack X} \cong \catDMod{\stack X × \stack X}
\]
such that the unit map is given by $Δ_*p^!$ and thus we have
\[
    u = Δ_*ω_{\stack X}.
\]
We summarize the above discussion in the following lemma.

\begin{Lem}\label{lem:d-mod:pre:hcoh}
    Let $\stack X$ be a stack.
    Then the Hochschild cohomology of $\catDMod{\stack X}$ is given by the dg algebra
    \[
        \HCoh(\catDMod{\stack X}) =
        \Hom_{\catDMod{\stack X × \stack X}}(Δ_*ω_{\stack X},\, Δ_*ω_{\stack X}).
    \]
\end{Lem}

\chapter{Base change for non-proper maps}
\label{ch:d-mod:strategy}

Let $\stack X$ be any stack, where we recall that all stacks are assumed to be QCA.
We are interested in computing the Hochschild cohomology
\[
    \HCoh\bigl(\catDMod{\stack X}\bigr).
\]
Let $Δ\colon \stack X → \stack X × \stack X$ is the diagonal map, which by assumption is schematic.
The dualizing module $ω_{\stack X}$ is always holonomic.
Thus we have $\dualize Δ_* ω_{\stack X} = Δ_! k_{\stack X}$.
With this we observe that
\begin{align*}
    \HCoh\bigl(\catDMod{\stack X}\bigr)
    & = \Hom_{\catDMod{\stack X × \stack X}}(Δ_* ω_{\stack X},\, Δ_* ω_{\stack X}) & &\text{(Lemma~\ref{lem:d-mod:pre:hcoh})} \\
    & = \opalg{\Hom_{\catDMod{\stack X × \stack X}}(Δ_! k_{\stack X},\, Δ_! k_{\stack X})} & & \text{(duality)} \\
    & = \opalg{\Hom_{\catDMod{\stack X × \stack X}}(k_{\stack X},\, Δ^! Δ_! k_{\stack X})} & & \text{(adjunction)} \\
    & = \opalg{\ΓdR\bigl(\stack X,\, Δ^! Δ_! k_{\stack X}\bigr)},
\end{align*}
where the algebra structure on $\ΓdR\bigl(\stack X,\, Δ^! Δ_! k_{\stack X}\bigr) = \Hom_{\catDMod{\stack X × \stack X}}(k_{\stack X},\, Δ^! Δ_! k_{\stack X})$ is the one induced by the $(Δ_!,Δ^!)$-adjunction monad.
Consider the Cartesian square
\[
    \begin{tikzcd}
        \ls{\stack X} \arrow[r, "p₁"] \arrow[d, "p₂"] & \stack X \arrow[d, "Δ"] \\
        \stack X \arrow[r, "Δ"] & \stack X × \stack X
    \end{tikzcd}
\]
Let us assume for the moment that $Δ$ (and hence $p_i$) was proper.
Then $Δ_* = Δ_!$ and $p_{2,*} = p_{2,!}$ and by Section~\ref{sec:d-mod:pre:monads} we have an isomorphism of monads
\begin{equation}
    \label{eq:d-mod:central_iso}
    p_{2,!} p₁^! → Δ^!Δ_!,
\end{equation}
which induces an isomorphism of algebras
\[
    \ΓdR\bigl(\stack X,\, p_{2,!} p₁^! k_{\stack X}\bigr)
    →
    \ΓdR\bigl(\stack X,\, Δ^! Δ_! k_{\stack X}\bigr).
\]
Of course, if $X$ is not an algebraic space, then $Δ$ is not proper (nor is it in general smooth).
Thus in general \eqref{eq:d-mod:central_iso} is not an isomorphism and there is no canonical structure of monad on $p_{2,!} p₁^!$.
We would like to apply Lemma~\ref{lem:d-mod:pre:groupoid_monad_hol} to construct a monad in special cases.
Thus the goal of this chapter is to give a criterion for the assumptions of Lemma~\ref{lem:d-mod:pre:groupoid_monad_hol}, i.e.~for base change to hold.

\begin{Ex}
    The base change morphism \eqref{eq:d-mod:central_iso} is also an isomorphism if $Δ$ is smooth.
    In particular this implies that the \enquote{naive expectation} holds for $\stack X = \B G$ for any algebraic group $G$, i.e.~we have
    \[
        \HCoh(\catDMod{\B G}) = \opalg{\ΓdR\bigl(\B G,\, p_{2,!} p₁^! k_{\B G}\bigr)}.
    \]
    An argument similar to \cite{BenZvi:mathoverflow:CohomologyOfGmodG} shows that there is a further isomorphism
    \[
        \ΓdR\bigl(\B G,\, p_{2,!} p₁^! k_{\stack X}\bigr) \cong
        \ΓdR(G, k_G)^\dual \otimes \ΓdR(\B G, k_{\B G}).
    \]
    Alternatively, we can use the identification
    \[
        \catDMod{\B G} \cong \catModules{\ΓdR(G, k_{G})^\dual},
    \]
    where the algebra structure on $\ΓdR(G, k_G)^\dual$ is induced by the group multiplication \cite[Section~7.2]{DrinfeldGaitsgory:2013:FinitenessQuestions}.
    If $G$ is reductive, then $\ΓdR(G, k_G)^\dual$ is an exterior algebra and thus its Hochschild cohomology can be computed directly.
\end{Ex}

\section{A lemma on base change}\label{sec:d-mod:strategy:base-change}

Consider a Cartesian diagram of stacks
\[
    \begin{tikzcd}
        \stack Z \arrow[d, "p"] \arrow[r, "q"] & \stack X₁ \arrow[d, "f"] \\
        \stack X₂ \arrow[r, "g"] & \stack Y
    \end{tikzcd}
\]
with $f$ and $g$ schematic.
We have a morphism of functors $\catDModHol{\stack X₁} → \catDModHol{\stack X₂}$,
\begin{equation}
    \label{eq:d-mod:base-change-morphism}
     p_! q^! → g^! f_!
\end{equation}
induced by adjunctions
\begin{equation}
    \label{eq:d-mod:base-change-adjunctions}
    p_! q^! →
    p_! q^! f^! f_! =
    p_! p^! g^! f_! →
    g^! f_!.
\end{equation}
If $f$ is proper, then \eqref{eq:d-mod:base-change-morphism} is an isomorphism by Proposition~\ref{prop:d-mod:pre:base-change}.
To understand the behavior for non-proper $f$, we will approximate it by a proper morphism.

\begin{Def}
    A \emph{relative compactification} of a morphism $f\colon \stack X → \stack Y$ is a commutative diagram
    \[
        \begin{tikzcd}
            \stack X \arrow[r, hook, "j"] \arrow[dr, "f"'] & \bar{\stack X} \arrow[d, "\bar f"] \\
            & \stack Y
        \end{tikzcd}
    \]
    where $j$ is an open embedding and $\bar f$ is proper.
\end{Def}

\begin{Ex}
    A famous example of such a relative compactification is Drinfeld's compactification of the morphism $\Bun_B → \Bun_G$, where $\Bun_G$ is the stack of $G$-bundles on a curve $C$ with $G$ reductive and $B$ is a Borel subgroup of $G$ \cite{BravermanGaitsgory:2002:GeometricEisensteinSeries}\todo{References to original work by Drinfeld?}.
\end{Ex}

Let us assume that in the above situation there exists a relative compactification of $f\colon \stack X₁ → \stack Y$.
Let $\stack X₁^c$ be the closed complement of the open inclusion $j\colon \stack X₁ \hookrightarrow \bar{\stack X}₁$.
Similarly, we let $\bar{\stack Z} = \stack X₂ ×_{\stack Y} \bar{\stack X}₁$ and $\stack Z^c = \stack X₂ ×_{\stack Y} \stack X₁^c$.
The notation for the corresponding inclusion and projection maps is summarized in the following Cartesian diagrams.
\[
    \begin{tikzcd}
        \bar{\stack Z} \arrow[d, "\bar p"] \arrow[r, "\bar q"] & \bar{\stack X}₁ \arrow[d, "\bar f"] \\
        \stack X₂ \arrow[r, "g"] & \stack Y
    \end{tikzcd}
    \qquad\qquad
    \begin{tikzcd}
        \stack Z^c \arrow[r, hook, "i"] \arrow[d] & \bar{\stack Z} \arrow[d, "\bar q"] \\
        \stack X₁^c \arrow[r, hook] & \bar{\stack X}₁
    \end{tikzcd}
\]
We note that $\bar{\stack Z}$ is the disjoint union of the closed substack $\stack Z^c$ and the open substack $\stack Z$.

\begin{Lem}
    \label{lem:d-mod:base-change-criterion}%
    The cone of the morphism~\eqref{eq:d-mod:base-change-morphism} is
    \[
        \bar p_! i_*i^* \bar{q}^! j_!.
    \]
    In particular, if $i^* \bar{q}^! j_! = 0$, then~\eqref{eq:d-mod:base-change-morphism} is an isomorphism of functors.
\end{Lem}

\begin{proof}
    Let $\tilde\jmath\colon \stack Z \hookrightarrow \bar{\stack Z}$ be the open inclusion complement to $i$.
    We split the adjunction is \eqref{eq:d-mod:base-change-adjunctions} in two by using the compositions
    \[
        f = \bar f ∘ j
        ,\
        p = \bar p ∘ \tilde\jmath
        \text{ and }
        q = \bar q ∘ \tilde\jmath.
    \]
    Thus the adjunction $p_!q^!→ p_!q^!f^!f_!$ becomes the sequence
    \[
        p_!q^! →
        p_!q^! j^! j_! →
        p_!q^! j^! \bar f^! \bar f_! j_!.
    \]
    The equality $p_! q^! f^! f_! = p_! p^! g^! f_!$ then becomes
    \[
        p_! q^! j^! \bar f^! \bar f_! j_! =
        p_! \tilde\jmath^! \bar q^! \bar f^! \bar f_! j_! =
        p_! \tilde\jmath^! \bar p^! g^! \bar f_! j_!.
    \]
    Finally the adjunction $p_! p^! g^! f_! → g^! f_!$ becomes
    \[
        p_! \tilde\jmath^! \bar p^! g^! \bar f_! j_! =
        \bar p_! \tilde\jmath_! \tilde\jmath^! \bar p^! g^! \bar f_! j_! →
        \bar p_! \bar p^! g^! \bar f_! j_! →
        g^! \bar f_! j_! =
        g^! f_!.
    \]
    Let us apply the same adjunction morphisms in a different order.
    First the inclusions
    \[
        p_!q^!
        \xrightarrow{α}
        p_!q^! j^! j_!
        =
        \bar p_! \tilde\jmath_! \tilde\jmath^! \bar q^! j_!
        \xrightarrow{β}
        \bar p_! \bar q^! j_!,
    \]
    and then the actual base change
    \begin{equation}
        \label{eq:lem:d-mod:base-change-criterion:pf:split_morphism_base_change}
        \bar p_! \bar q^! j_!
        →
        \bar p_! \bar q^! \bar f^! \bar f_! j_!
        =
        \bar p_! \bar p^! g^! \bar f_! j_!
        →
        g^! \bar f_! j_!
        =
        g^! f_!.
    \end{equation}
    We note that the adjunction $α\colon \id → j^!j_!$ is an isomorphism and the composition of the maps in \eqref{eq:lem:d-mod:base-change-criterion:pf:split_morphism_base_change} is exactly the isomorphism of proper base change (cf.~Proposition~\ref{prop:d-mod:pre:base-change}).
    Thus the cone of the whole composition composition is the same as the cone of the morphism $β$, which is given by the recollement triangle
    \[
        \bar p_! \tilde\jmath_! \tilde\jmath^! \bar q^! j_!
        \xrightarrow{β}
        \bar p_! \bar q^! j_!
        \xrightarrow{\phantom{β}}
        \bar p_! i_* i^* \bar q^! j_!
        \xrightarrow{+1}.
        \qedhere
    \]
\end{proof}

\section{Relative compactification for quotient stacks}
\label{sec:d-mod:strategy:compactification}%

In the preceding section we simply assumed that a relative compactification of the diagonal exists.
We will now construct such a compactification for quotient stacks.
Thus $X$ be a scheme of finite type over $k$ and let $G$ be an affine algebraic group over $k$ acting on $X$.
Let $\stack X = X/G$ be the corresponding quotient stack.

Constructing a relative compactification of $Δ\colon \stack X → \stack X × \stack X$ is the same as a first constructing a $G × G$-equivariant relative compactification of $(\proj2, a)\colon G × X → X × X$ (where $a\colon G × X → X$ is the action map) and then taking the quotient by the $G × G$ action\footnote{%
    Here $G × G$ acts on $G × X$ by $(s₁,s₂) \cdot (g,x) = (s₂gs₁^{-1},\, s₁x)$.
}.
We let
\[
    Γ = \bigl\{(g, x, x, gx) ∈ G × X × X × X\bigr\}
\]
be the graph of $(\proj2, a)$.

We pick a $G$-equivariant compactification $\bar G$ of $G$ and let $\bar Γ$ be the closure of $Γ$ in $\bar G × X × X × X$.
We have an open embedding $j$ of $G × X \cong Γ$ into $\bar Γ$ and proper map $f\colon \bar Γ → X × X$ given by projection on the last two factors.
The composition $f ∘ j$ is equal to $(\proj2, a)$.

Instead of viewing $Γ$ as the graph of $(\proj2, a)$ we can drop the third factor and regard $Γ$ as the graph of the action map, i.e.
\[
    \Γsub{a} = \bigl\{(g, x, gx) ∈ G × X × X\bigr\}.
\]
The closure $\bar{ \Γsub{a}}$ of $\Γsub a$ in $\bar G × X × X$ identifies with $\bar Γ$.
Thus for ease of notation we will from now on always set $Γ = \Γsub a$ and $\bar Γ = \bar{\Γsub a}$.

\begin{Def}
    Let $\stack X = X/G$.
    With the above construction we set
    \[
        \bar{\stack X} = \rquot{\bar Γ}{G×G}.
    \]
    We have an open embedding $j\colon X \hookrightarrow \bar{\stack X}$ and a proper morphism $\bar Δ\colon \bar{\stack X} → \stack X × \stack X$ induced by the map $f$ above, such that $Δ = \bar Δ ∘ j$.
\end{Def}

\begin{Rem}
    In the case of $G = \Gm$ the compactification $\bar Γ$ is explicitly described in \cite{DrinfeldGaitsgory:2014:OnATheoremOfBraden}.
    In particular, if $X$ is smooth it is shown there that $\bar Γ$ is smooth over $\ps1$.
    It is possible to extend the methods of \cite{DrinfeldGaitsgory:2014:OnATheoremOfBraden} to quotients by higher dimensional tori.
    The resulting constructions are highly useful for doing explicit computations.
\end{Rem}

It is useful to consider only partial compactifications.
For this let $V$ be a $G$-stable subvariety of $\bar G$ and let $\bar{\Γsub V}$ be the closure of $Γ$ in $G × V$.
We set
\[
    \bar{\stack X}_V = \rquot{\bar{\Γsub V}}{G×G}.
\]
Clearly, if $\left\{V_i\right\}$ is an open cover of $\bar G$ by $G$-stable subvarieties, then $\left\{\bar{\stack X}_{V_i}\right\}$ is an open cover of $\bar{\stack X}$.

\section{\Goodstack\ stacks}

Let $\stack X = X/G$ be a quotient stack as before.
For any morphism of stacks $h\colon \stack Y → \stack X$ we set $\lsY{\stack X} = \stack X ×_{\stack X × \stack X} \stack Y$.
Thus we have the Cartesian diagram
\[
    \begin{tikzcd}
        \lsY \stack X \arrow[r, "q_{\stack Y}"] \arrow[d] & \stack X \arrow[d, "Δ"] \\
        \stack Y \arrow[r, "Δ∘h"] & \stack X × \stack X
    \end{tikzcd}
\]
Let us fix a relative compactification $\bar Δ \colon \bar{\stack X} → \stack X × \stack X$ as in Section~\ref{sec:d-mod:strategy:compactification}.
Using the notation of Section~\ref{sec:d-mod:strategy:base-change} we set $\clsY{\stack X} = \bar{\stack X} ×_{\stack X × \stack X} \stack Y$ and $\lscY{\stack X} = {\stack X}^c ×_{\stack X × \stack X} \stack Y$.
We let $\bar q_{\stack Y}\colon \clsY\stack X → \bar{\stack X}$ be the projection morphism and $i_{\stack Y}\colon \lscY\stack X \hookrightarrow \clsY\stack X$ the inclusion.

\begin{Def}
    A quotient stack $\stack X = X/G$ is called \emph{\goodstack} if for every quotient stack $\stack Y = Y/G$ and schematic morphism $\stack Y → \stack X$ the functor $i_{\stack Y}^* \bar q_{\stack Y}^! j_!$ vanishes on $\catDModHol{\stack X}$.
\end{Def}

We will show in Chapter~\ref{ch:d-mod:torus} that any stack of the form $\rquot{X}{\Gm^n}$ is \goodstack.
The reason for this definition is the following theorem which lets us compute the Hochschild cohomology of $\catDMod{\stack X}$ for \goodstack\ quotient stacks.

\begin{Thm}\label{thm:d-mod:good-is-good}
    If $\stack X = X/G$ is \goodstack, then there exists a canonical structure of monad on $p_{2,!}p₁^!$ and the morphism $p_{2,!}p₁^! → Δ^!Δ_!$ is an isomorphism of monads.
    In particular there is an isomorphism of algebras
    \[
        \HCoh\bigl(\catDMod{\stack X}\bigr)
        \cong
        \opalg{\ΓdR\bigl(\stack X,\, p_{2,!} p₁^! k_{\stack X}\bigr)}.
    \]
\end{Thm}

In other words, Theorem~\ref{thm:d-mod:main} holds for \goodstack\ stacks.

\begin{proof}
    We apply Lemma~\ref{lem:d-mod:pre:groupoid_monad_hol} to the groupoid $\ls\stack X \rightrightarrows \stack X$.
    Thus we let $\stack G_\cx$ be the simplicial stack with
    \[
        \stack G_i = \underbrace{\ls\stack X ×_{\stack X} \dotsb ×_{\stack X} \ls\stack X}_{\text{$i$ factors}}.
    \]
    Any morphism $F\colon [n] → [m]$ in $\cat{Δ}^{\mathrm{op}}$ induces a diagram
    \[
        \begin{tikzcd}
            \stack G_{n+1} \arrow[d] \arrow[r] & \stack G_{m+1} \arrow[d] \arrow[r] & \stack X \arrow[d, "Δ"] \\
            \stack G_{n} \arrow[r] & \stack G_m \arrow[r] & \stack X × \stack X
        \end{tikzcd}
    \]
    We have to show that base change holds along the right-hand square.
    But by the assumption and Lemma~\ref{lem:d-mod:base-change-criterion}, base change holds along the outer rectangle and the left-hand square.
    Thus it also holds along the right-hand square.
\end{proof}

\begin{Rem}
    We expect that most quotient stacks are not \goodstack.
    For example, a direct computation shows that Theorem~\ref{thm:d-mod:main} does not hold for the stack $\ps1/\as1$, and hence it is not \goodstack.
    For non-\goodstack\ stacks, Lemma~\ref{lem:d-mod:base-change-criterion} instead gives a description of how much the naive expectation for $\HCoh\bigl(\catDMod{\stack X}\bigr)$ fails.
\end{Rem}

We finish this section with some useful observations for proving that a stack is \goodstack.

\begin{Lem}
    If $\stack X₁ = X₁/G₁$ and $\stack X₂ = X₂/G₂$ are \goodstack, then $\stack X₁ × \stack X₂$ is \goodstack.
\end{Lem}

\begin{proof}
    Follows from compatibility of $i_{\stack Y}^*\bar{q}_{\stack Y}^!j_!$ with $\boxtimes$ and coproducts.
    (Note that $\lscY{(\stack X₁ × \stack X₂)} = \lscY \stack{X₁} × \clsY \stack{X₂} ∪ \clsY{\stack X₁} × \lscY{\stack X₂}$.)
\end{proof}

\begin{Lem}
    Let $U$ be a $G$-equivariant open subset of $X$.
    If $X/G$ is \goodstack\ then $U/G$ is \goodstack.
\end{Lem}

\begin{proof}
    Let $\stack U = U/G$ and let $\stack Y$ be a quotient stack mapping into $\stack U$ (and hence also into $\stack X$).
    Consider the diagram
    \begin{equation}
        \label{eq:lem:d-mod:cover:diagram}
        \begin{tikzcd}
            \lscY \stack U \arrow[r, hook, "i_{\stack U, \stack Y}"] \arrow[d, hook, "α"] & \clsY{\stack U} \arrow[r, "\bar q_{\stack U,\stack Y}"] \arrow[d, hook, "β"] & \bar{\stack U} \arrow[d, hook, "γ"] & \stack U \arrow[l, hook', "j_{\stack U}"'] \arrow[d, hook, "δ"] \\
            \lscY \stack X \arrow[r, hook, "i_{\stack X, \stack Y}"]                      & \clsY{\stack X} \arrow[r, "\bar q_{\stack X,\stack Y}"]                      & \bar{\stack X}                      & \stack X \arrow[l, hook', "j_{\stack X}"']
        \end{tikzcd}
    \end{equation}
    The vertical arrows are open embeddings and all squares are Cartesian (where we use the same compactification of $G$ for $\bar{\stack X}$ and $\bar{\stack U}$).
    Thus
    \begin{equation*}
        i_{\stack U, \stack Y}^* \bar q_{\stack U,\stack Y}^! j_{\stack U,!} =
        i_{\stack U, \stack Y}^* \bar q_{\stack U,\stack Y}^! j_{\stack U,!} δ^* δ_*=
        α^* i_{\stack X, \stack Y}^* \bar q_{\stack X, \stack Y}^! j_{\stack X,!} δ_* =
        0.
        \qedhere
    \end{equation*}
\end{proof}

The same argument can be used to reduce the computation to a smooth cover.
We will now introduce notation for the special case of the cover $\bar Γ → \bar{\stack X}$.
The corresponding covers of the other relevant stacks are introduced in the following diagram with Cartesian squares.
\begin{equation}
    \label{eq:d-mod:strategy:scheme-cover}
    \begin{tikzcd}
        \schemelscY \stack X \arrow[r, hook, "\schemei_{\stack Y}"] \arrow[d] & \schemeclsY \stack X \arrow[r, "\bar\schemeq_{\stack Y}"] \arrow[d] & \bar Γ \arrow[d] & Γ \arrow[l, hook', "\schemej"'] \arrow[d] \\
        \lscY \stack X \arrow[r, hook, "i_{\stack Y}"]                        & \clsY \stack X \arrow[r, "\bar q_{\stack Y}"]                       & \bar{\stack X}   & \stack X \arrow[l, hook', "j"']
    \end{tikzcd}
\end{equation}
We note that all vertical morphisms are smooth and the spaces in the top row are schemes.
Let $\schemeh\colon X → Y$ be the $G$-equivariant morphism of schemes inducing $h$ on quotient stacks.
Then the scheme $\schemeclsY \stack X$ is given by
\[
    \schemeclsY \stack X =
    \biggl\{
        \bigl(g₁,\, y,\, g₂\bigr) ∈ G × Y × \bar G : \bigl(\schemeh(y),\, g₂,\, g₁\schemeh(y)\bigr) ∈ \bar Γ
    \biggr\}.
\]

\begin{Lem}\label{lem:d-mod:strategy:scheme-cover}
    A stack $X/G$ is \goodstack\ if for each morphism $Y/G → X/G$ the composition $\schemei_{\stack Y}^* \bar\schemeq_{\stack Y}^! \schemej_!$ vanishes on $\catDModHolMon{Γ}{G×G}$.
\end{Lem}

\begin{proof}
    Follows from the fact that pullback along the smooth vertical morphisms in \eqref{eq:d-mod:strategy:scheme-cover} is conservative \cite[Lemma~5.1.6]{DrinfeldGaitsgory:2013:FinitenessQuestions} and permutes with the other morphisms up to a shift.
\end{proof}

\begin{Lem}
    \label{lem:d-mod:strategy:cover}%
    If there exists a $G$-stable open cover $U_i$ of $X$ such that all stacks $U_i/G$ are \goodstack, then $X/G$ is \goodstack.
\end{Lem}

\begin{proof}
    Let $\stack U_i = U_i/G$ be the corresponding quotient stacks.
    We first show that the stacks $\lscY{\stack U_i}$ form an open cover of $\lscY{\stack X}$\footnote{
        This is not completely obvious, since the $\bar{\stack U}_i$ do not necessarily form a cover of $\bar{\stack X}$.
        For example, consider $\ps1$ with the usual linear $\Gm$-action and the usual affine cover.
    }.
    It suffices to show that the open subschemes $\schemeclsY \stack U_i$ cover $\schemeclsY \stack X$.
    Let $(g₁, y, g₂)$ be a point of $\schemeclsY \stack X$.
    Then there exists some $U_i$ with $h'(y) ∈ U_i$. 
    But then $g₁h'(y)$ is also in $U_i$ and hence $(h'(y), g₂, g₁h'(y)) ∈ \bar{\stack U}_i$.
    Thus $(g₁, y, g₂)$ is in $\schemeclsY \stack U_i$.
    
    It now suffices to show that the restrictions of $i_{\stack Y}^*\bar{q}_{\stack Y}^!j_! \sheaf F$ to $\lscY{\stack U_i}$ vanish for every $\sheaf F ∈ \catDMod{\stack X}$.
    But this follows from the diagram~\eqref{eq:lem:d-mod:cover:diagram} (for $\stack U_i$ instead of $\stack U$) and the \goodness\ of $\stack U_i$.
\end{proof}

Let $\left\{V_i\right\}$ be a $G$-stable open cover of $\bar G$ and consider the corresponding open cover $\left\{\bar{\stack X}_{V_i}\right\}$ of $\stack X$.
We obtain open covers $\left\{\cls_{V_i,\stack Y} \stack X\right\}$ and $\left\{\lsc_{V_i,\stack Y} \stack X\right\}$ of $\clsY{\stack X}$ and $\lscY{\stack X}$ respectively.
We let $i_{V_i,\stack X}$, $\bar q_{V_i,\stack X}$ and $j_{V_i}$ be the corresponding maps, i.e.
\[
    \begin{tikzcd}
        \cls_{V_i,\stack Y} \stack X \arrow[r, hook, "i_{V_i,\stack X}"] &
        \lsc_{V_i,\stack Y} \stack X \arrow[r, "\bar q_{V_i,\stack X}"] &
        \bar{\stack X}_{V_i} &
        \stack X \arrow[l, hook', "j_{V_i}"'].
    \end{tikzcd}
\]

\begin{Lem}\label{lem:d-mod:strategy:cover-by-relative-compactifications}
    With the above notation, the $\stack X$ is good if and only $i_{V_i,\stack Y}^* \bar q_{V_i,\stack Y}^! j_{V_i,!}$ vanishes on $\catDModHol{\stack X}$ for all $V_i$ and all $\stack Y → \stack X$.
\end{Lem}

\begin{proof}
    Similar to the proof of Lemma~\ref{lem:d-mod:strategy:cover}.
\end{proof}

%\chapter{Generalization of a construction by Drinfeld}
% or: The relative compactification in the torus case.

In this chapter we will prove the following key ingredient of the proof of Theorem~\ref{thm:d-mod:main}.

\begin{Thm}
    \label{thm:d-mod:smoothness_of_compactification}%
    Let $X$ be a smooth affine scheme of finite type over $k$ and let $T$ be a torus acting on $X$.
    Set $\bar T = (\ps 1)^n$ and let $\bar Γ$ be the closure of the graph $Γ$ of the action morphism in $\bar T × X × X$.
    Then the morphism $q\colon \bar Γ → \bar T$ is smooth.
\end{Thm}

For $T = \Gm$ this theorem is proved in \cite[Section~2]{DrinfeldGaitsgory:2014:OnATheoremOfBraden}.
The proof of the general case follows the same outline as the one for the $\Gm$-case, though with some additional indicies.
Therefore the constructions and theorems in this chapter are mostly straightforward generalizations of \cite{DrinfeldGaitsgory:2014:OnATheoremOfBraden} to higher dimensions.
The reader is encouraged to read about the $\Gm$ case first.

Throughout this section, we let $X$ be as in Theorem~\ref{thm:d-mod:smoothness_of_compactification} except that we do not assume that $X$ is affine.
We always set $T = \Gm^n$.

\begin{Def}
    The action of $T$ on $X$ is called \emph{locally linear} if $X$ can be covered by $T$-invariant open affine subschemes.
\end{Def}

ToDo: There should be a short reminder of the notation $\mappingstack^T(X,Y)$ here.

\section{Fixed points and attractors}

By a \emph{sequence of signs} of length $n$ we will understand an element of $\{-,\bullet,+\}^n$.
We will usually write $σ = (σ_i)$ for such a sequence.
We write $\bar σ ∈ \{0,1\}^n$ for the sequence given by
\[
    \bar σ_i =
    \begin{cases}
        1 & σ_i = \bullet, \\
        0 & \text{otherwise.}
    \end{cases}
\]
For a sequence $ε ∈ \{0,1\}^n$ we write
\[
    \pt^ε = \prod_i \Gm^{ε_i},
\]
where $\Gm^1 = \Gm$ and $\Gm^0 = \pt$.
In particular $\pt^0 = \pt$ and $\pt^{1} = T$.

\begin{Def}
    Let $ε ∈ \{0,1\}^n$.
    The \emph{subspace of $ε$-fixed points} of $X$ is
    \[
        X^ε = \mappingstack^T(\pt^ε, X),
    \]
\end{Def}
In particular, $X^1 = X$ and $X^0 = \mappingstack^T(\pt, X)$ are the $T$-fixed points.
Note that $X^ε$ is indeed a subspace of $X$ since $\Maps(S, X^ε) = \Maps(\pt^ε × S,\, X)^T$ is a subset of $\Maps(S, X)$.

\begin{Lem}[{cf.~\cite[Proposition~1.3.4]{DrinfeldGaitsgory:2014:OnATheoremOfBraden}}]
    Let $ε ∈ \{0,1\}^n$.
    Then $X^ε$ is a scheme of finite type and the morphism $X^ε → X$ is a closed embedding.
\end{Lem}

\begin{proof}
    Let 
    \[
        T' = \prod_{ε_i = 0} \Gm
    \]
    Then $X^ε$ is the same as $X^0$ with respect to the $T'$ action.
    Thus we can assume that $ε = 0$.
    Now the statement follows from an iterated application of \cite[Proposition~1.3.4]{DrinfeldGaitsgory:2014:OnATheoremOfBraden}.
\end{proof}

ToDo: Maybe add a remark about affine schemes and gradings (cf~\cite[Example~1.3.5]{DrinfeldGaitsgory:2014:OnATheoremOfBraden}.

We write $\as[+]1$ for the space $\as 1$ with the $\Gm$ action given by $λ\cdot x = λx$.
We write $\as[-]1$ for the space $\as 1$ with the $\Gm$ action given by $λ\cdot x = λ^{-1}x$.
Finally, we set $\as[\bullet]1 = \Gm$ with the usual action of $\Gm$ on itself.
For a sequence of signs $σ$ we write $\as[σ]n$ for the product
\[
    \as[σ]n = \prod_{i = 1}^n \as[σ_i]1
\]
with the induced $T$ action.

\begin{Def}
    Let $σ$ be a sequence of signs.
    The \emph{$σ$-attractor} of $X$ is the space
    \[
        X^σ = \mappingstack^{T}(\as[σ]n, X).
    \]
\end{Def}

We write $0\colon \pt^{\bar σ} → \as[σ]n$ for the map which is the identity on the $\Gm$ factors and the inclusion $0 → \as 1$ on the other factors.
It induces a map
\[
    q^σ \colon X^σ → X^{\bar σ}.
\]

[ToDo: Morphism $p^σ\colon X^σ → X$.]
[ToDo: Explicit construction in the affine case via gradings, cf.~\cite[Remark~1.4.7]{DrinfeldGaitsgory:2014:OnATheoremOfBraden}]

\begin{Lem}
    If $X$ is affine, then $p^σ\colon X^σ → X$ is a closed immersion.
\end{Lem}

\begin{proof}
    Follows from the explicit description.
\end{proof}

\begin{Lem}[{cf.~\cite[Lemma~1.4.9(i)]{DrinfeldGaitsgory:2014:OnATheoremOfBraden}}]%
    \label{lem:subspace_attractor}%
    Let $σ$ be a sequence of signs and let $Y \subseteq X$ be a $T$-stable open subspace.
    Then the subspace $Y^σ \subseteq X^σ$ equals $(q^σ)^{-1}(Y^{\bar σ})$.
\end{Lem}

\begin{proof}
    For any test scheme $S$ we have to show that if $f\colon S × \as[σ]n → Z$ is a $T$-equivariant morphism such that $S × \pt^σ \subseteq f^{-1}(Y)$, then $f^{-1}(Y) = S × \as[σ]n$.
    This is clear because $f^{-1}(Y) \subseteq S × \as[σ]n$ is open and $T$-stable.
\end{proof}

\section{The central construction}

\begin{Def}
    We set $\X = \as n × \as n = \Spec k[x₁,\dotsc,x_n, y₁,\dotsc,y_n]$, regarded as a scheme over $\as n$ by the map $(x₁,\dotsc,x_n,y₁,\dotsc,y_n) \mapsto (x₁y₁, \dotsc, x_ny_n)$.
    For any scheme $S$ over $\as n$ we set $\X_S = \X ×_{\as n} S$.
    We equip $\X$ with a $T$ action by 
    \[
        (λ₁,\dotsc,λ_n) \cdot (x₁, \dotsc, x_n, y₁, \dotsc, y_n) = (λ₁x₁, \dotsc, λ_nx_n, λ₁^{-1}x₁, \dotsc, λ_n^{-1}x_n).
    \]
    Since the action preserves the map to $\as n$, we get an action of $T$ on $\X_S$.
\end{Def}

A section $\as n → \X$ induces a section $S → \X_S$ for every scheme $S$.
We will in particular be interested in the sections
\begin{equation}
    \label{eq:interpolation:sections}%
    σ₁\colon (t₁,\dotsc, t_n) \mapsto (1,\dotsc,1, t₁,\dotsc,t_n)
    \quad\text{and}\quad
    σ₂(t₁,\dotsc, t_n) \mapsto (t₁,\dotsc,t_n, 1,\dotsc,1).
\end{equation}

\begin{Def}
    Given a scheme $X$ with a $T$-action we define a space $\widetilde X$ over $\as n$ by
    \[
        \Maps_{\as n}(S, \widetilde X) \coloneq \Maps^{T}(\X_S, X)
    \]
    for any schemes $S$.
    In other words, an $S$-point of $\widetilde X$ consists of a morphism $S → \as{n}$ and a $T$-equivariant morphism $\X_S → X$.
\end{Def}

Note that a section $\as n → \X$ induces a map $\Maps^{T}(\X_S, X) → \Maps(S, X)$ and therefore a morphism $\widetilde X → X$.
We let $π₁, \,π₂\colon \widetilde X → X$ be the morphisms corresponding to the sections $σ₁, σ₂$ of \eqref{eq:interpolation:sections}.
Finally we let $\tilde p\colon \widetilde X → \as n × X × X$ be the morphisms whose components are the tautological map $\widetilde X → \as n$, $π₁$ and $π₂$.

\section{Fibers and subspaces}

If $t = (t₁,\dotsc,t_n) ∈ \as n$, then the fiber $\X_t$ is defined by the equations $x_iy_i = t_i$ for $i = 1,\dotsc,n$.
If $t ∈ T$, then $\X_t$ is isomorphic to $T$ via projection onto the $x_i$.
However if $t \notin T$, then $\X_t$ splits into several components.
The components of $\X_t$ are of the form $\as[σ]n$ where $σ$ runs over all sequences of signs such that $\bar σ = ε(t)$ and $ε(t) ∈ \{0,1\}^n$ is the sequence given by
\[
    ε(t)_i = 
    \begin{cases}
        0 & t_i = 0, \\
        1 & \text{otherwise.}
    \end{cases}
\]
For any $t ∈ \as n$ the fiber $\widetilde{X}_t$ of is given by
\[
    \widetilde{X}_t = \mappingstack^{T}(\X_t, X).
\]
Let 
\[
    \tilde p_t\colon \widetilde X_t → \{t\} × X × X
\]
be the morphism induced by $\tilde p$.

\begin{Lem}[{cf.~\cite[Proposition~2.2.6]{DrinfeldGaitsgory:2014:OnATheoremOfBraden}}]
    \label{lem:graph_contained_in_tilde}%
    The morphism $\tilde p$ induces an isomorphism between $T ×_{\as n} \tilde X \subseteq \tilde X$ and the graph of the action morphism $T × X → X$.
    In particular, for $t ∈ T$ the pair $(\widetilde X_t, \tilde p)$ identifies with $(X, Γ_t)$ where $Γ_t\colon X → \{t\}×X×X$ is the graph of the action morphism $t\colon X → X$.
\end{Lem}

\begin{proof}
    Let $t ∈ T$.
    Points of $\widetilde X_t$ are $T$-equivariant maps $f\colon\X_t → X$, where $\X_t \cong T$.
    Thus $\widetilde X_t \cong X$.
    Let $z = π₁(f) = f(\underline{1},t)$.
    Then 
    \[
        π₂(f) = f(t, \underline{1}) = tf(\underline{1}, t) = tz.
        \qedhere
    \]
\end{proof}

\begin{Lem}\label{lem:tildeFibers}
    The fiber $\tilde{X}_t$ is a subspace of the fiber product of all $X^σ$ with $\bar σ = ε(t)$ taken over $X^{ε(t)}$.
\end{Lem}

\begin{proof}[Outline of proof]
    The intersection of all the components of $\X_t$ is precisely $\pt^{ε(t)}$.
    We write $\X_t$ as a pushout diagram\footnote{See \cite[Corollary~3.9]{Schwede:2005:GluingSchemesAndASchemeWithoutPoints} for the existence of pushouts in schemes along closed subvarieties. More detailed discussion can be found in \cite{Ferrand:2003:ConducteurDescenteEtPincement}.} of the components $\as[σ]n$ over their intersections.
    Thus we see that $\X_t$ receives a surjective morphism from the iterated pushout of the $\as[σ]n$ over $\pt^{ε(t)}$.
    Hence $\tilde{Z}_t = \mappingstack^T(\X_t, Z)$ is a subspace of the fiber product of all $X^σ = \mappingstack^T(\as[σ]n, X)$ over $X^{ε(t)} = \mappingstack^T(\pt^{ε(t)}, X)$.
\end{proof}

\begin{Lem}[{cf.~\cite[Proposition~2.3.2]{DrinfeldGaitsgory:2014:OnATheoremOfBraden}}]
    \label{lem:tilde_subspace}%
    \leavevmode
    \begin{enumerate}
        \item
            \label{lem:tilde_subspace_closed}%
            Let $Y \subseteq X$ be a $T$-stable closed subspace.
            Then the diagram
            \[
                \begin{tikzcd}
                    \widetilde Y \arrow[r] \arrow[d, "\tilde p_Y"] & \widetilde X \arrow[d, "\tilde p_X"] \\
                    \as n × Y × Y \arrow[r, hook] & \as n × X × X
                \end{tikzcd}
            \]
            is Cartesian.
            In particular, the morphism $\widetilde Y → \widetilde X$ is a closed embedding.
        \item 
            \label{lem:tilde_subspace_open}%
            Let $Y \subseteq X$ be a $T$-stable open subspace.
            Then the above diagram identifies $\widetilde Y$ with an open subspace of the fiber product
            \[
                \widetilde X ×\limits_{\as n × X × X} (\as n × Y × Y).
            \]
            In particular, the morphism $\widetilde Y → \widetilde X$ is an open embedding.
    \end{enumerate}
\end{Lem}

\begin{proof}
    Let 
    \[
        \mathring \X = \X \setminus \bigl\{x₁\dotsm x_n = 0 \text{ and } y₁\dotsm x_m = 0 \bigr\}.
    \]
    Thus $\mathring \X$ is exactly the image of the two sections \eqref{eq:interpolation:sections}.
    For a scheme $S$ over $\as n$, set $\mathring\X_S = {\X_S ×_\X \mathring\X}$.
    Now the rest of the proof can be repeated word for word from \cite[Proposition~2.3.2]{DrinfeldGaitsgory:2014:OnATheoremOfBraden}.
\end{proof}

\section{Properties of the compactification}

\begin{Claim}\label{claim:d-mod:tilde_monomorphism}
    Let $X$ be separated. Then the map $\tilde p\colon \tilde X → \as n × X × X$ is a monomorphism.
\end{Claim}

[The proof is the same as for \cite[Proposition~2.3.4]{DrinfeldGaitsgory:2014:OnATheoremOfBraden}.]

\begin{Claim}\label{claim:tildeAffineFiniteType}
    If $X$ is an affine scheme of finite type, then $\tilde p$ is a closed embedding.
    In particular $\tilde X$ is an affine scheme of finite type.
\end{Claim}

\begin{proof}[Outline of proof]
    By Lemma~\ref{lem:tilde_subspace}\ref{lem:tilde_subspace_closed}, it is enough to show the statement for $X = \as m$.
    If the statement holds for $X₁$ and $X₂$, then it also holds for $X₁ × X₂$ [why?].
    Since any $T$-representation decomposes into one-dimensional representations this reduces to $X = \as 1$ and $(t₁,\dotsc,t_n) \cdot x = t₁^{λ₁}\dotsm t_n^{λ_n}x$.
    For this case one does an explicit computation [do it!].
\end{proof}

\begin{Claim}[{cf.~\cite[Theorem~2.4.2]{DrinfeldGaitsgory:2014:OnATheoremOfBraden}}]
    \label{claim:tildeFiniteType}%
    $\tilde X$ is a scheme of finite type.
\end{Claim}

\begin{proof}[Outline of proof]
    Let $\{U_i\}$ be a finite open $T$-stable affine cover of $X$ (recall that we assume that the action of $T$ on $X$ is locally linear).
    By Claim~\ref{claim:tildeAffineFiniteType}, each $\tilde U_i$ is an affine scheme of finite type.
    By Lemma~\ref{lem:tilde_subspace}\ref{lem:tilde_subspace_open}, each $\tilde U_i$ is an open subscheme of $\tilde X$.
    Thus it suffices to prove that $\tilde X$ is covered by the $\tilde U_i$.

    It is enough to check that each fiber $\tilde X_t$ is covered by the open subschemes $(\tilde U_i)_t$.
    For $t ∈ T \subseteq \as n$ this is clear from the analogue of \cite[Proposition~2.2.6]{DrinfeldGaitsgory:2014:OnATheoremOfBraden}.

    Thus let $t ∈ \as n \setminus T$.
    By Lemma~\ref{lem:tildeFibers}, each point of of $\tilde X_t$ is a tuple
    \[
        \tilde x = (x^σ)_{\bar σ = ε(t)} ∈ \prod_{\bar σ = ε(t)} X^σ
    \]
    such that all images $q^σ(x^σ)$ coincide.
    The point $q^σ(x^σ)$ is contained in some $U_i$.
    By Lemma~\ref{lem:subspace_attractor} each $x^σ$ is contained in $U_i$ [for this we need that in the affine case $Z^σ$ is a subscheme of $Z$].
    Hence $\tilde x ∈ (\tilde U_i)_t$.
\end{proof}

\begin{Claim}\label{claim:d-mod:tilde_smooth}
    If $X$ is smooth, then the canonical morphism $\tilde X → \as n$ is smooth.
\end{Claim}

\begin{proof}
    By Claim~\ref{claim:tildeFiniteType}, it suffices to prove that $\tilde p$ is formally smooth\footnote{
        See \cite[Définition~17.1.1]{EGA4.4} for the definition of \enquote{formally smooth}.
        \cite[Section~17.3]{EGA4.4}~shows that for morphisms that are locally of finite presentation, \enquote{smooth} and \enquote{formally smooth} coincide.
        \cite[Remarque~17.1.2(ii)]{EGA4.4}~shows that it is only necessary to consider nilpotent ideals whose square is $0$.
        Somewhere it is probably written that one only needs to test with affine schemes.
    }.
    Thus $R$ be a $k$-algebra with a morphism $\Spec(R) → \as n$.
    Let $I$ be an ideal of $R$ with $I² = 0$ and set $\bar R = \rquot RI$.
    Let $\bar f ∈ \Maps(\X_{\bar R}, X)^T$.
    We have to show that $\bar f$ can be lifted to an element of $\Maps(\X_R, X)^T$.
    Since $\X_R$ is affine and $Z$ is smooth, there is no obstruction to lifting $\bar f$ to an element of $\Maps(\X_R, X)$.
    
    According to \cite{DrinfeldGaitsgory:2014:OnATheoremOfBraden}, \enquote{standard arguments}\todo{What are these standard arguments?}\ show that the obstruction to existence of a $T$-equivariant lift is in $H¹(T, M)$, where $M = H^0(\X_{\bar R},\, \bar f^*(Θ_X)) ⊗_{\bar R} I$ and $Θ_X$ is the tangent bundle of $X$.
    But $H^1$ of $T$ with coefficients in any $T$-module is zero.
\end{proof}

\begin{proof}[Proof of Theorem~\ref{thm:d-mod:smoothness_of_compactification}]
    Let $X$ be a smooth affine scheme with a $T$-action.
    We want to prove that $\bar Γ \subseteq (\ps1)^n × X × X$ is smooth over $(\ps1)^n$.
    Fix a cover of $(\ps1)^n$ by open subsets of the form $\as n$.
    Since smoothness is local on both the source and the target, it suffices to check that the closure of $Γ$ in $\as n × X × X$ is smooth over $\as n$.

    By Claim~\ref{claim:tildeAffineFiniteType}, the morphism $\tilde p$ identifies $\tilde X$ with a closed subscheme of $\as n × X × X$.
    By \ref{lem:graph_contained_in_tilde}, we have $Γ \subseteq \tilde p(\tilde X)$ and hence also $\bar Γ \subseteq \tilde p(\tilde X)$.
    Thus by Claim~\ref{claim:d-mod:tilde_smooth}\todo{Why does smoothness imply the equality?}, $\bar Γ = \tilde p(\tilde X)$ is smooth over $\as n$.
\end{proof}

\chapter{Torus quotients}
\label{ch:d-mod:torus}

In this chapter we will apply the tools from the previous chapter to torus quotient stacks.
Specifically, we will prove the following theorem.

\begin{Thm}
    \label{thm:d-mod:torus:is-good}%
    Let $G$ be a torus acting locally linearly\footnote{
        We only use the \enquote{locally linear} assumption to prove Lemma~\ref{lem:d-mod:finitely_many_stabilizers}, i.e.~that $\Stab\stack X$ is locally finite.
        Thus we could just assume that instead.}
    on a scheme $X$ of finite type over $k$.
    Then the stack $\stack X = X/G$ is good.
\end{Thm}

Together with Theorem~\ref{thm:d-mod:good-is-good} this implies our main result, Theorem~\ref{thm:d-mod:main}.

By Lemma~\ref{lem:d-mod:strategy:cover}, it suffices to prove Theorem~\ref{thm:d-mod:torus:is-good} for stacks $X/G$ with $X$ affine.
We fix an isomorphism $G \cong \Gm^n$ and compactify $G$ to $\bigl(\ps1\bigr)^n$.
The variety $\bigl(\ps1\bigr)^n$ can be covered by $G$-equivariant open subvarieties of the form $\as n$.
Thus by Lemma~\ref{lem:d-mod:strategy:cover-by-relative-compactifications}, it suffices to check \goodness\ for the relative compactification $\Gm^n \subseteq \as n$.
To simplify notation, we drop the subscript $\as n$ from the notation and set $\bar{\stack X} = \bar{\stack X}_{\as n}$ and similarly for the various maps.

We fix a quotient stack $\stack Y = Y/G$ and a morphism $h\colon \stack Y → \stack X$, induced by a $G$-equivariant morphism $\schemeh\colon Y → X$.
According to Lemma~\ref{lem:d-mod:strategy:scheme-cover}, rather than working with stack directly, we can base change to schemes.
We will use the notation of Lemma~\ref{lem:d-mod:strategy:scheme-cover}, but for ease of notation we will drop the subscript $\stack Y$ from the maps.
Thus we are concerned with the diagram
\[
    \begin{tikzcd}
        \schemelscY \stack X \arrow[r, hook, "\schemei"] &
        \schemeclsY \stack X \arrow[r, "\bar\schemeq"] &
        \bar Γ &
        Γ \arrow[l, hook', "\schemej"'],
    \end{tikzcd}
\]
where we have to show that $\schemei^* \bar\schemeq^! \schemej_!$ vanishes on $\catDModHolMon{Γ}{G×G}$.

\todo{Add some motivation here.}
We will cut the scheme
\[
    \schemeclsY \stack X =
    \biggl\{
        \bigl(g₁,\, y,\, g₂\bigr) ∈ G × Y × \bar G : \bigl(\schemeh(y),\, g₂,\, g₁\schemeh(y)\bigr) ∈ \bar Γ
    \biggr\}
\]
into pieces according to the subgroups of $G$ that stabilize them.
For this let $\Stab\stack X$ be the set of all closed subgroups of $G$ that are stabilizers of points of $X$, i.e.
\[
    \Stab\stack X = \{ G_x : x ∈ X \}.
\]

\begin{Lem}
    \label{lem:d-mod:finitely_many_stabilizers}%
    The set $\Stab\stack X$ is finite.
\end{Lem}

\begin{proof}
    Since $X$ is affine, it can be embedded $G$-equivariantly into some $\as m$ with a linear $T$-action.
    For $\as m/T$ the statement is easy to see.
\end{proof}

Let $S$ be closed subgroup of $G$ and let $X^S$ be the $S$-fixed points of $X$.
Since $G$ is Abelian (and hence $S$ a normal subgroup), $X^S$ is a $G$-stable closed subscheme of $X$.
Hence $X^S/G$ is a closed substack of $\stack X$.
Let $\bar S$ be the closure of $S$ in $\bar G = \as n$ and consider the space
\[
    \schemecls[S]_{\stack Y} \stack X =
    \biggl\{
        \bigl(g₁,\, y,\, g₂\bigr) ∈ G × Y × \bar G : \schemeh(y) ∈ X^S,\, \bigl(\schemeh(y),\, g₂,\, g₁\schemeh(y)\bigr) ∈ \bar Γ \text{ and } g₂ ∈ g₁\bar S
    \biggr\}
    ⊆ 
    \schemecls_{\stack Y} \stack X.
\]

\begin{Lem}
    \label{lem:d-mod:stabilizers_cover}%
    The subspaces $\schemecls[S]_{\stack Y} \stack X$ for $S ∈ \Stab\stack X$ cover $\schemecls_{\stack Y}\stack X$.
\end{Lem}

\begin{proof}
    Let $\schemelsY Y \stack X$ be the smooth cover of $\ls_{\stack Y}\stack X$.
    Consider the spaces
    \[
        \schemelsY[S] \stack X = 
        \biggl\{
            \bigl(g₁,\, y,\, g₂\bigr) ∈ G × Y × G : \schemeh(y) ∈ X^S,\, \bigl(\schemeh(y),\, g₂,\, g₁\schemeh(y)\bigr) ∈ Γ \text{ and } g₂ ∈ g₁S
        \biggr\}
        \subseteq \schemelsY \stack X.
    \]
    The closure of $\schemelsY[S] \stack X$ in $\schemeclsY \stack X$ is exactly $\schemeclsY[S] \stack X$.
    It is easy to see that the subspaces $\schemelsY[S] \stack X$ for $S ∈ \Stab \stack X$ cover $\schemelsY[S] \stack X$.
    Now the statement follows from the fact that the closure of a finite union is the union of the individual closures.
\end{proof}

It will be useful to have a slight generalization of the schemes $\schemeclsY[S]$.
Let $S₁ \subseteq S₂$ be two subgroups of $G$ contained in $\Stab\stack X$.
We set
\[
    \schemeclsY[S₁,S₂] \stack X =
    \biggl\{
        \bigl(g₁,\, y,\, g₂\bigr) ∈ G × Y × \bar G : \schemeh(y) ∈ X^{S₂},\, \bigl(\schemeh(y),\, g₂,\, g₁\schemeh(y)\bigr) ∈ \bar Γ \text{ and } g₂ ∈ g₁\bar S₁
    \biggr\}.
\]
Clearly we have $\schemeclsY[S₁,S₂]{\stack X} \subseteq \schemecls[S₁]{\stack X}$ and $\schemeclsY[S] \stack X = \schemeclsY[S,S] \stack X$.

Consider the Cartesian square of closed embeddings
\[
    \begin{tikzcd}
        \schemelscY[S₁,S₂] \stack X \arrow[r, hook, "i_{S₁,S₂}^c"] \arrow[d, hook, "\schemei^{S₁,S₂}"] & \schemelscY\stack X \arrow[d, hook, "\schemei"] 
        \\
        \schemeclsY[S₁,S₂] \stack X \arrow[r, hook, "i_{S₁,S₂}"]   & \schemeclsY\stack X 
    \end{tikzcd}
\]

\begin{Lem}\label{lem:d-mod:key_for_stablizier}%
    For any $S₁ ⊆ S₂$ in $\Stab\stack X$ and any $\sheaf F ∈ \catDModHolMon{Γ}{G×G}$ we have
    \[
        \schemei^{S₁,S₂,*} i_{S₁,S₂}^! \bar\schemeq^! \schemej_! \sheaf F = 0.
    \]
\end{Lem}


\begin{proof}
    The scheme $\schemelsc[S₁,S₂] \stack X$ is given by
    \[
        \biggl\{
            \bigl(g₁,\, y,\, g₂\bigr) ∈ G × Y × \bar G : \schemeh(y) ∈ X^{S₂},\, \bigl(\schemeh(y),\, g₂,\, g₁\schemeh(y)\bigr) ∈ \bar Γ \text{ and } g₂ ∈ g₁(\bar S₁ \setminus S₁)
        \biggr\}.
    \]
    If $S₁ = \bar S₁$ the statement is trivially true.
    Otherwise the scheme $\bar S₁ \setminus S₁$ is the union of hyperplanes $H_i$ of $\bar S₁$.
    It suffices to prove the statement when further restricting to 
    \[
        \biggl\{
            \bigl(g₁,\, y,\, g₂\bigr) ∈ G × Y × \bar G : \schemeh(y) ∈ X^{S₂},\, \bigl(\schemeh(y),\, g₂,\, g₁\schemeh(y)\bigr) ∈ \bar Γ \text{ and } g₂ ∈ g₁H_i
        \biggr\}.
    \]
    for all $i$. 
    Let $H$ be one such hyperplane.
    We will assume that $H$ is contained in the closure of the connected component of $1 ∈ S₁$. 
    The proof for $H$ is a different component is the same, up to a shift by an element of $G$.
    Let $\schemei_H$ be the inclusion of
    \[
        Z = 
        \biggl\{
            \bigl(g₁,\, y,\, g₂\bigr) ∈ G × Y × \bar G : \schemeh(y) ∈ X^{S₂},\, \bigl(\schemeh(y),\, g₂,\, g₁\schemeh(y)\bigr) ∈ \bar Γ \text{ and } g₂ ∈ g₁H
        \biggr\}.
    \]
    into $\schemeclsY[S₁,S₂] \stack X$.
    We want to compute
    \[
        \schemei_H^{*} i_{S₁,S₂}^! \bar\schemeq^! \schemej_! \sheaf F.
    \]

    Write $G \cong G₁ × S₁$ for some subgroup $G₁$ of $G$.
    Then $\bar G = \bar G₁ \bar S₁$.
    Let $H' = \bar G₁ H$.
    We note that $G ∩ H' = \emptyset$.
    
    We chose an action $μ$ of $\Gm$ on $\bar S₁$ that contracts $\bar S₁$ onto $H$.
    This induces an action of $\Gm$ on $\bar G = \bar G₁ \bar S₁$ by $u \cdot ts = tμ(u,s)$, contracting $\bar G$ onto $H'$.
    Further we obtain an action of $\Gm$ on $\bar Γ$ that keeps the first $X$ coordinate fixed.
    By construction this action contracts $\bar Γ$ onto a closed subvariety of $\bar Γ \setminus Γ$.
    We will denote this subvariety by $Z₁$ and the contraction morphism $π\colon \bar Γ → Z₁$ by $π₁$.
    We can also lift this action to $\schemeclsY[S₁,S₂]\stack X$ where it contracts onto $Z$.
    We will denote the corresponding contraction morphism by $π\colon \schemeclsY[S₁,S₂]\stack X → Z$.

    We note that the D-modules $\schemej_!\sheaf F$ and $i_{S₁,S₂}^! \bar\schemeq^! \schemej_! \sheaf F$ are monodromic with respect to these $\Gm$-actions.
    The contraction principle Theorem~\ref{thm:d-mod:pre:contraction_principle} thus implies that
    \[
        \schemei_H^* i_{S₁,S₂}^! \bar\schemeq^! \schemej_! \sheaf F = π_* i_{S₁,S₂}^! \bar\schemeq^! \schemej_! \sheaf F.
    \]
    The square
    \[
        \begin{tikzcd}
            \schemecls[S₁,S₂] \arrow[r, "π"] \arrow[d, "\schemeq ∘ i_{S₁,S₂}"] & Z \arrow[d] \\
            \bar Γ \arrow[r, "π₁"] & Z₁
        \end{tikzcd}
    \]
    is Cartesian\todo{This probably needs some explanation.}. 
    Let us call the right vertical map $f$.
    Base change yields
    \[
        π_* i_{S₁,S₂}^!\bar\schemeq^!\schemej_! \sheaf F = 
        f^!π_{1,*} \schemej_! \sheaf F.
    \]
    Finally, let $i_{Z₁}\colon Z₁ \hookrightarrow \bar Γ$ be the inclusion.
    Applying the contraction principle again we obtain
    \[
        f^!π_{1,*} \schemej_! \sheaf F =
        f^!i_{Z₁}^* \schemej_! \sheaf F =
        0.
        \qedhere
    \]
\end{proof}

\begin{proof}[Outline of proof of Theorem~\ref{thm:d-mod:torus:is-good}]
    Iterated Mayer-Vietoris argument using lemmas~\ref{lem:d-mod:stabilizers_cover} and~\ref{lem:d-mod:key_for_stablizier}.
    Note that if $S₁,\, S₂ ∈ \Stab\stack X$, then
    \[
        \schemeclsY[S₁]\stack X ∩ \schemeclsY[S₂]\stack X = 
        \schemeclsY[S₁∩S₂,\, S₁S₂].
    \]
    \todo{Add some details.}
\end{proof}



\backmatter
\bookmarksetupnext{level=part}
\printbibliography

\end{document}
