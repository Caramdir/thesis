\chapter{Prerequisites}

\begin{itemize}
    \item Everything over an algebraically closed field of characteristic $0$.
    \item Assumptions on stacks.
    \item Note on dg categories / linear stable ($∞$,1)-categories.
\end{itemize}

\section{D-modules on stacks}

\begin{itemize}
    \item Definition via $\catIndCoh{\stack X_{\dR}}$.
    \item $f^!$, $f_*$, $Γ_{\dR}$
    \item ($ω_{\stack X}$)
    \item $k_{\stack X}$
\end{itemize}

In this thesis we will be mainly concerned with the subcategory of holonomic D-modules.
See \cite{Braverman:LecturesOnAlgebraicDmodules} for the Definition of holonomic D-modules on arbitrary $k$-schemes and their compatibility with $f^!$, $f_*$ and $\dualize$.

\begin{Def}
    A $D$-module $\sheaf F ∈ \catDMod{\stack X}$ is called \emph{holonomic} if $f^!\sheaf F$ is holonomic for any smooth morphism $f\colon Z → \stack X$ from a scheme $Z$.
    The full subcategory of holonomic D-modules will be denoted with $\catDModHol{\stack X}$.
\end{Def}

\begin{Prop}
    Let $f\colon \stack X → \stack Y$ be a schematic morphism.
    Then $f^!$ and $f_*$ restrict to functors 
    \[
        f^!\colon \catDModHol{\stack Y} → \catDModHol{\stack X}
        \quad\text{and}\quad
        f_*\colon \catDModHol{\stack X} → \catDModHol{\stack Y}.
    \]
\end{Prop}

\begin{itemize}
    \item duality (on coherent/holonomic)
    \item non-standard functors on holonomic D-modules $f_!$, $f^*$ [We should probably worry a bit that all $_!$ and $^*$ functors are defined on the objects to which we apply them.]
    \item base change (via adjunctios)
\end{itemize}

\begin{Def}
    \label{def:d-mod:pre:monodromic}%
    Let $X$ be a scheme with an action of an algebraic group $G$ and let $p\colon X → X/G$.
    The \emph{monodromic} subcategory $\catDModMon{X}{G} ⊆ \catDMod{X}$ is the subcategory generated by the essential image of $p^!\colon \catDMod{X/G} → \catDMod{X}$.
\end{Def}

\begin{Thm}[Contraction principle~{\cite[Proposition~3.2.2]{DrinfeldGaitsgory:2014:OnATheoremOfBraden}}]
    \label{thm:d-mod:pre:contraction_principle}%
    Let $X$ be a scheme with an action by $\Gm$ that extends to an action of the monoid $\as 1$.
    Let $i\colon X^0 \hookrightarrow X$ be the closed subscheme of $\Gm$-fixed points and let $π\colon X → X^0$ be the contraction morphism induced by the $\Gm$-equivariant morphism $\as 1 → \{0\}$.
    Then there is an isomorphism of functors
    \[
        i^* \cong π_* \colon \catDModMon{X}{G} → \catDMod{X^0}.
    \]
\end{Thm}


\section{Monads}
\label{sec:d-mod:pre:monads}

\begin{itemize}
    \item monads (on dg categories; \cite[Section~4.7]{Lurie:2014-draft:HigherAlgebra}), morphisms between monads
    \item $T$ a monad on $\cat C$, then $\Hom_{\cat C}(X, TX)$ is a (unital, associative) $k$-algebra for all $X ∈ \cat C$.
    \item morphisms of monads give morphisms of algebras
    \item monads from adjunction
\end{itemize}

\begin{Def}
    A \emph{groupoid} $\mathcal G$ in $\cat{Stacks}$ consists of a stack $G₀$ of \enquote{objects} and a stack $G₁$ of \enquote{morphisms} with
    \begin{itemize}
        \item \emph{source} and \emph{target} maps $s,t\colon G₁ \rightrightarrows G₀$,
        \item a \emph{unit} $e\colon G₀ → G₁$,
        \item a \emph{multiplication} (or \emph{composition}) map $m\colon G₁ ×\limits_{s,G₀,t} G₁ → G₁$),
        \item a \emph{inverse} map $ι\colon G₁ → G₁$,
    \end{itemize}
    such that
    \begin{itemize}
        \item $s ∘ e = t ∘ e = \id_{G₀}$,
        \item $s ∘ m = s ∘ p₂$ and $t ∘ m = t ∘ p₁$ (where $p_i\colon G₁ ×_{s,G₀,t} G₂$ are the projection maps).
        \item $m$ is associative,
        \item $ι$ interchanges $s$ and $t$ and is an inverse for $m$.
    \end{itemize}
\end{Def}

\begin{Ex}
    For our purpose the most important example is the following:
    Let $f\colon X → S$ be a map in $\cat{Stacks}$.
    We set $G_0 = X$ and $G₁ = X ×_S X$.
    The source and target maps are given by $p₁$ and $p₂$, the unit by the diagonal $Δ\colon X → X×_SX$, the inverse by interchanging the factors and multiplication is $p₁₃\colon X ×_S X ×_S X → X×_SX$.
\end{Ex}

Let us for the moment assume that $s$ (and hence $t$) is proper.
In this case the maps $e$ and $m$ are also proper, since $s ∘ e = \id_{G₀}$ and $s ∘ m = s ∘ p₂$ are proper [ToDo: reference].
In particular the functors $e^!\colon \catDMod{G₁} → \catDMod{G₀}$ and $m^!\colon \catDMod{G₁} → \catDMod{G₁ ×_{G₀} G₁}$ have left adjoints given by $e_*$ and $m_*$ respectively.
This allows us to give the endofunctor $T = s_*t^!$ of $\catDMod{G₀}$ the structure of a monad in the following way:

\begin{itemize}
    \item By $(e_*,e^!)$-adjunction we have a transformation
        \[
            \id = (s∘e)_*(t∘e)^! = s_*e_*e^!t^! → s_*t^! = T.
        \]
    \item Consider the following commutative diagram
        \[
            \begin{tikzcd}[column sep=small]
                {} & & G₁ \arrow[dddll, bend right, "s"] \arrow[dddrr, bend left, "t"] & & \\
                & & G₁ ×_{G₀} \arrow[u, "m"] \arrow[dl, "p₂"] \arrow[dr, "p₁"] G₁ & & \\
                & G₁ \arrow[dl, "s"] \arrow[dr, "t"] & & G₁ \arrow[dl, "s"] \arrow[dr, "t"] & \\
                G₀ & & G₀ & & G₀
            \end{tikzcd}
        \]
        with Cartesian middle square.
        Proper base change and $(m_*,m^!)$-adjunction gives a transformation
        \[
            T² =
            s_*t^!s_*t^! =
            (s∘p₂)_*(t∘p₁)^! =
            (s∘m)_*(t∘m)^! =
            s_*m_*m^!t^! →
            s_*t^! =
            T.
        \]
\end{itemize}

\begin{itemize}
    \item this is actually a monad
    \item base change morphism from groupoid monad ($X ×_S X$) to adjunction monad (proper case)
    \item monads from groupoids (general case): \cite[Lemma~II.1.7.1.4]{GaitsgoryRozenblyum:prelim:StudyInDAG}, \cite[Section~4.7.6]{Lurie:2014-draft:HigherAlgebra}
\end{itemize}

\section{Hochschild cohomology}

\begin{itemize}
    \item Co-complete categories and continuous functors
    \item Definition via $\Hom_{\cat{Funct}_{\text{(cont)}}}(\id, \id)$.
    \item Continuous functors and kernels.
    \item The case of D-modules: via kernel; adjunction monad; algebra structure is the one induced from the adjunction monad.
\end{itemize}
