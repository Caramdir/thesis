\chapter{Prerequisites}

Everything over an algebraically closed field of characteristic $0$.

\section{Some constructions on prestacks (?)}

\begin{itemize}
    \item $\mappingstack(X,Y)$ and $\mappingstack^G(X,Y)$
    \item QCA stacks
\end{itemize}

\section{D-modules on stacks}

\begin{itemize}
    \item Definition via $\catIndCoh{\stack X_{\dR}}$.
    \item $f^!$, $f_*$, $Γ_{\dR}$
    \item $ω_{\stack X}$
    \item $k_{\stack X}$ and $f^*$.
    \item coherent D-modules
    \item duality (on holonomic/coherent)
    \item safety of schematic maps
    \item non-standard functors on holonomic D-modules $f_!$, $f^*$ [We should probably worry a bit that all $_!$ and $^*$ functors are defined on the objects to which we apply them.]
    \item base change (via adjunctions)
\end{itemize}

\section{Hochschild cohomology}

\begin{itemize}
    \item Definition via $\Hom_{\cat{Funct}_{\text{(cont)}}}(\id, \id)$.
    \item Continuous functors and kernels.
    \item The case of D-modules.
\end{itemize}

\section{Monads}
\label{sec:d-mod:pre:monads}

[This section is not strictly speaking necessary, as we will not use monads. We will however use the constructions restricted to a single object.]

\begin{itemize}
    \item monads (on dg categories), morphisms between monads
    \item monads from adjunction
    \item monads from groupoids (proper case)
    \item base change morphism from groupoid monad ($X ×_S X$) to adjunction monad
    \item $T$ a monad on $\cat C$, then $\Hom_{\cat C}(X, TX)$ is a (unital, associative) $k$-algebra for all $X ∈ \cat C$.
    \item morphisms of monads give morphisms of algebras
\end{itemize}

%In this section:
%Monads from groupoids,
%algebras from monads,
%base change morphism from group monad to adjunction monad.
%
%\begin{Def}
%    A \emph{groupoid} $\mathcal G$ in $\cat{Stacks}$ consists of a stack $G₀$ of \enquote{objects} and a stack $G₁$ of \enquote{morphisms} with
%    \begin{itemize}
%        \item \emph{source} and \emph{target} maps $s,t\colon G₁ \leftleftarrows G₀$,
%        \item a \emph{unit} $e\colon G₀ → G₁$,
%        \item a \emph{multiplication} (or \emph{composition}) map $m\colon G₁ ×\limits_{s,G₀,t} G₁ → G₁$),
%        \item a \emph{inverse} map $ι\colon G₁ → G₁$,
%    \end{itemize}
%    such that
%    \begin{itemize}
%        \item $s ∘ e = t ∘ e = \id_{G₀}$,
%        \item $s ∘ m = s ∘ p₂$ and $t ∘ m = t ∘ p₁$ (where $p_i\colon G₁ ×_{s,G₀,t} G₂$ are the projection maps).
%        \item $m$ is associative,
%        \item $ι$ interchanges $s$ and $t$ and is an inverse for $m$.
%    \end{itemize}
%\end{Def}
%
%\begin{Ex}
%    For our purpose the most important example is the following:
%    Let $f\colon X → S$ be a map in $\cat{Stacks}$. 
%    We set $G_0 = X$ and $G₁ = X ×_S X$.
%    The source and target maps are given by $p₁$ and $p₂$, the unit by the diagonal $Δ\colon X → X×_SX$, the inverse by interchanging the factors and multiplication is $p₁₃\colon X ×_S X ×_S X → X×_SX$.
%\end{Ex}
%
%Consider a groupoid $G₁ \rightarrows G₀$ and a theory of sheaves $\catSh({-})$ on $\cat{Stacks}$ such that the left adjoints $s_!$, $e_!$, $m_!$ and $p_{1,!}$ to $s^!$, $e^!$, $m^!$ and $p₁^!$ respectively exist.
%Then the functor $T = s_!t^!$ is a monad on $G₀$ in the following way:
%
%\begin{itemize}
%    \item By $(e_!,e^!)$-adjunction we have a transformation 
%        \[
%            \id = (s∘e)_!(t∘e)^! = s_!e_!e^!t^! → s_!t^!.
%        \]
%    \item Consider the following commutative diagram
%        \[
%            \begin{tikzcd}[column sep=small]
%                {}& & G₁ \arrow[bend right]{dddll}{s} \arrow[bend left]{dddrr}{t} & & \\
%                & & G₁ ×_{G₀} \arrow{u}{m}\arrow{dl}{π₁}\arrow{dr}{π₂} G₁ & & \\
%                & G₁ \arrow{dl}{s}\arrow{dr}{t} & & G₁ \arrow{dl}{s}\arrow{dr}{t} & \\
%                G₀ & & G₀ & & G₀
%            \end{tikzcd}
%        \]
%        with Cartesian middle square.
%        Then base change and $(m_!,m^!)$-adjuction (assuming e.g.~that $m$ is proper) gives a transformation
%        \[
%            T² =
%            s_*t^!s_*t^! =
%            (s∘π₁)_*(t∘π₂)^! =
%            (s∘m)_*(t∘m)^! =
%            s_*m_*m^!t^! →
%            s_*t^! =
%            T.
%        \]
%\end{itemize}
