\chapter{The strategy}
\label{ch:d-mod:strategy}

Let $X$ be a scheme of finite type over an algebraically closed field $k$ of characteristic $0$.
Suppose an affine algebraic group $G$ over $k$ acts on $X$.
Let $\stack X = X/G$ be the corresponding quotient stack.
We are interested in computing the Hochschild cohomology
\[
    \HCoh\bigl(\catDMod{\stack X}\bigr) = \Hom_{\catDMod{\stack X × \stack X}}(Δ_*ω_{\stack X},\, Δ_*ω_{\stack X}),
\]
where $Δ\colon \stack X → \stack X × \stack X$ is the diagonal map.
In this chapter we will outline a strategy for computing the Hochschild cohomology.
We will apply this strategy to the case $G = \Gm^n$ in Chapter~\ref{ch:d-mod:torus}.

\section{The morphism}

Let us for the moment just assume that $\stack X$ is a QCA stack [with schematic diagonal].

The dualizing module $ω_{\stack X}$ is always holonomic. 
Thus we can compute $\dualize Δ_* ω_{\stack X} = Δ_! k_{\stack X}$.
With this we observe that\todo{Because of duality this should be an anti-isomorphism?}
\begin{align*}
    \HCoh\bigl(\catDMod{\stack X}\bigr) 
    & = \Hom_{\catDMod{\stack X × \stack X}}(Δ_* ω_{\stack X},\, Δ_* ω_{\stack X}) & &\text{(definition)} \\
    & = \Hom_{\catDMod{\stack X × \stack X}}(Δ_! k_{\stack X},\, Δ_! k_{\stack X}) & & \text{(duality)} \\
    & = \Hom_{\catDMod{\stack X × \stack X}}(k_{\stack X},\, Δ^! Δ_! k_{\stack X}) & & \text{(adjunction)} \\
    & = \ΓdR\bigl(Δ^! Δ_! k_{\stack X}\bigr)
\end{align*}
Consider the Cartesian square
\[
    \begin{tikzcd}
        \ls{\stack X} \arrow[r, "p₁"] \arrow[d, "p₂"] & \stack X \arrow[d, "Δ"] \\
        \stack X \arrow[r, "Δ"] & \stack X × \stack X
    \end{tikzcd}
\]
Let us assume for the moment that $Δ$ (and hence $p_i$) was proper.
Then $Δ_* = Δ_!$ and $p_{2,*} = p_{2,!}$ and by Section~\ref{sec:d-mod:pre:monads} we have an isomorphism of monads
\begin{equation}
    \label{eq:d-mod:central_iso}
    p_{2,!} p₁^! → Δ^!Δ_!,
\end{equation}
which induces an isomorphism of algebras
\[
    \ΓdR\bigl(Δ^! Δ_! k_{\stack X}\bigr)
    →
    \HCoh\bigl(\catDMod{\stack X}\bigr).
\]
Of course, if $X$ is not an algebraic space, then $Δ$ is not proper (nor is it in general smooth).
Thus in general \eqref{eq:d-mod:central_iso} is not an isomorphism and there is no canonical structure of monad on $p_{2,!} p₁^!$.
Note however that on holonomic D-modules we always have a morphism as in \eqref{eq:d-mod:central_iso} given by the adjunctions
\begin{equation}
    \label{eq:d-mod:central_iso_adjunctions}
    p_{2,!}p₁^! →
    p_{2,!}p₁^! Δ^! Δ_! =
    p_{2,!}p₂^! Δ^! Δ_! →
    Δ^! Δ_!.
\end{equation}

\section{A lemma on base change}

Consider a Cartesian diagram of stacks
\[
    \begin{tikzcd}
        \stack Z \arrow[d, "p"] \arrow[r, "q"] & \stack X₁ \arrow[d, "f"] \\
        \stack X₂ \arrow[r, "g"] & \stack Y
    \end{tikzcd}
\]
with $f$ and $g$ schematic.
We have a morphism of functors $\catDModHol{\stack X₁} → \catDModHol{\stack X₂}$,
\begin{equation}
    \label{eq:d-mod:base-change-morphism}
     p_! q^! → g^! f_!
\end{equation}
induced by adjunctions
\begin{equation}
    \label{eq:d-mod:base-change-adjunctions}
    p_! q^! →
    p_! q^! f^! f_! =
    p_! p^! g^! f_! →
    g^! f_!.
\end{equation}
If $f$ is proper, then \eqref{eq:d-mod:base-change-morphism} is an isomorphism \cite[\RomanNum{III}.4.2.1.3]{GaitsgoryRozenblyum:prelim:StudyInDAG}.
To understand the behavior for non-proper $f$, we will approximate it by a proper morphism.

\begin{Def}
    A \emph{relative compactification} of a morphism $f\colon X → Y$ is a commutative diagram 
    \[
        \begin{tikzcd}
            \stack X \arrow[r, hook, "j"] \arrow[dr, "f"'] & \bar{\stack X} \arrow[d, "\bar f"] \\
            & \stack Y
        \end{tikzcd}
    \]
    where $j$ is an open embedding and $\bar f$ is proper.
\end{Def}

Let us assume that in the above situation there exists a relative compactification of $f\colon \stack X₁ → \stack Y$.
Let $\stack X₁^c$ be a closed complement of $\stack X₁$ in $\bar{\stack X}₁$.
We introduce some further notation in the following Cartesian diagrams.
\[
    \begin{tikzcd}
        \bar{\stack Z} \arrow[d, "\bar p"] \arrow[r, "\bar q"] & \bar{\stack X}₁ \arrow[d, "\bar f"] \\
        \stack X₂ \arrow[r, "g"] & \stack Y
    \end{tikzcd}
    \qquad
    \begin{tikzcd}
        \stack Z^c \arrow[r, hook, "i"] \arrow[d, "p^c"] & \bar{\stack Z} \arrow[d, "\bar f"] \\
        \stack X₁^c \arrow[r, hook] & \bar{\stack X}₁
    \end{tikzcd}
\]

\begin{Lem}
    \label{lem:d-mod:base-change-criterion}%
    The cone of the morphism~\eqref{eq:d-mod:base-change-morphism} is
    \[
        p^c_! i^* \bar{q}^! j_!.
    \]
\end{Lem}

\begin{proof}
    Let $\tilde\jmath\colon \stack Z \hookrightarrow \bar{\stack Z}$ be the open inclusion complement to $i$.
    We split the adjunction is \eqref{eq:d-mod:base-change-adjunctions} in two by using the compositions 
    \[
        f = \bar f ∘ j
        ,\ 
        p = \bar p ∘ \tilde\jmath
        \text{ and }
        q = \bar q ∘ \tilde\jmath.
    \]
    Thus the adjunction $p_!q^!→ p_!q^!f^!f_!$ becomes the sequence
    \[
        p_!q^! →
        p_!q^! j^! j_! →
        p_!q^! j^! \bar f^! \bar f_! j_!.
    \]
    The equality $p_! q^! f^! f_! = p_! p^! g^! f_!$ then becomes
    \[
        p_! q^! j^! \bar f^! \bar f_! j_! =
        p_! \tilde\jmath^! \bar q^! \bar f^! \bar f_! j_! =
        p_! \tilde\jmath^! \bar p^! g^! \bar f_! j_!.
    \]
    Finally the adjunction $p_! p^! g^! f_! → g^! f_!$ becomes
    \[
        p_! \tilde\jmath^! \bar p^! g^! \bar f_! j_! =
        \bar p_! \tilde\jmath_! \tilde\jmath^! \bar p^! g^! \bar f_! j_! → 
        \bar p_! \bar p^! g^! \bar f_! j_! → 
        g^! \bar f_! j_! = 
        g^! f_!.
    \]
    Let us apply the same adjunction morphisms in a different order.
    First the inclusions
    \[
        p_!q^!
        \xrightarrow{α}
        p_!q^! j^! j_! 
        =
        \bar p_! \tilde\jmath_! \tilde\jmath^! \bar q^! j_! 
        \xrightarrow{β}
        \bar p_! \bar q^! j_!,
    \]
    and then the actual base change
    \begin{equation}
        \label{eq:lem:d-mod:base-change-criterion:pf:split_morphism_base_change}
        \bar p_! \bar q^! j_!
        →
        \bar p_! \bar q^! \bar f^! \bar f_! j_!
        =
        \bar p_! \bar p^! g^! \bar f_! j_!
        →
        g^! \bar f_! j_!
        =
        g^! f_!.
    \end{equation}
    We note that the adjunction $\id → j^!j_!$ labeled $α$ is an isomorphism and the composition of the maps in \eqref{eq:lem:d-mod:base-change-criterion:pf:split_morphism_base_change} is exactly the isomorphism of proper base change \cite[\RomanNum{III}.4.2.1.3]{GaitsgoryRozenblyum:prelim:StudyInDAG}.
    Thus the cone of the whole composition composition is exactly the cone of the morphism $β$, i.e.
    \[
        \bar p_! \tilde\imath_* \tilde\imath^* \bar q^! j_! = p^c_! \tilde\imath^* \bar q^! j_!.
        \qedhere
    \]
\end{proof}

\section{How to obtain a relative compactification}
\label{sec:d-mod:strategy:compactification}%

Let us now consider the specific situation of a quotient stack $\stack X = X/G$ and the diagonal morphism $Δ\colon \stack X → \stack X × \stack X$.
We will show how to construct a relative compactification of $Δ$.

Constructing a relative compactification of $Δ\colon \stack X → \stack X × \stack X$ is the same as a first constructing a $G × G$-equivariant relative compactification of $(\proj2, a)\colon G × X → X × X$ (where $a\colon G × X → X$ is the action map) and then taking the quotient by the $G × G$ action\footnote{%
    It might be worth mentioning that here $G × G$ acts on $X × G$ by $(s₁,s₂) \cdot (x,g) = (s₁x,\, s₂gs₁^{-1})$.
}.
We let 
\[
    Γ = \bigl\{(g, x, x, gx) ∈ G × X × X × X\bigr\}
\]
be the graph of $(\proj2, a)$.

We pick a $G$-equivariant compactification $\bar G$ of $G$ and let $\bar Γ$ be the closure of $Γ$ in $\bar G × X × X × X$.
We have an open embedding $j$ of $G × X \cong Γ$ into $\bar Γ$ and proper map $f\colon \bar Γ → X × X$ given by projection on the last two factors.
The composition $f ∘ j$ is equal to $(\proj2, a)$.

Instead of viewing $Γ$ as the graph of $(\proj2, a)$ we can drop the third factor and regard it as the graph of the action map, i.e.
\[
    Γ_a = \bigl\{(g, x, gx) ∈ G × X × X\bigr\}.
\]
The closure $\bar Γ_a$ of $Γ_a$ in $\bar G × X × X$ identifies with $\bar Γ$.
Thus for ease of notation we will from now on always set $Γ = Γ_a$ and $\bar Γ = \bar Γ_a$.

\begin{Def}
    Let $\stack X = X/G$. 
    Then with the above construction we set 
    \[
        \bar{\stack X} = \rquot{\bar Γ}{G×G}.
    \]
    We have an open embedding $j\colon X \hookrightarrow \bar{\stack X}$ and a proper morphism $\bar Δ\colon \bar{\stack X} → \stack X × \stack X$ induced by the map $f$ above, such that $Δ = \bar Δ ∘ j$.
\end{Def}

[ToDo: Remark about torus case (smoothness and reference).]

Let $V$ be a $G$-stable subvariety of $\bar G$.
We can do the above construction with $V$ instead of $\bar Γ$ to obtain a partial compactification which we will denote by $\bar{\stack X}_V$.

\section{\Goodstack\ stacks}

Let $\stack X = X/G$ be a quotient stack and consider any morphism of stacks $h\colon \stack Y → \stack X$.
Consider the Cartesian diagram
\[
    \begin{tikzcd}
        \ls_{\stack Y} \stack X \arrow[r, "q"] \arrow[d] & \stack X \arrow[d, "Δ"] \\
        \stack Y \arrow[r, "h∘Δ"] & \stack X × \stack X
    \end{tikzcd}
\]
Let us fix a relative compactification $\bar Δ \colon \bar{\stack X} → \stack X × \stack X$ as in Section~\ref{sec:d-mod:strategy:compactification} and set $\tilde\imath_{\stack Y} = \tilde\imath$ and $\bar q_{\stack Y} = \bar q$.

\begin{Def}
    A quotient stack $\stack X = X/G$ is called \emph{\goodstack}\footnote{ToDo: Come up with a better name.} if for every quotient stack $\stack Y = Y/G$ and morphism $\stack Y → \stack X$ one has $\tilde\imath_{\stack Y}^* \bar q_{\stack Y}^! j_! = 0$.
\end{Def}

\begin{Thm}
    If $\stack X = X/G$ is \goodstack, then there is a canonical structure of monad on $p_{2,!}p₁^!$ and the morphism $p_{2,!}p₁^! → Δ^!Δ_!$ is an isomorphism of monads.
    In particular there is an isomorphism of algebras
    \[
        \ΓdR\bigl(p_{2,!} p₁^! k_{\stack X}\bigr)
        →
        \HCoh\bigl(\catDMod{\stack X}\bigr).
    \]
\end{Thm}

\begin{proof}[Outline of proof]
    Apply \goodness\ for $h = \id_{\stack X}$ to obtain the base change isomorphism $p_{2,!}p₁^! → Δ^!Δ_!$.
    Similarly use $\stack Y = \ls \stack X ×_{\stack X} \dotsc ×_{\stack X} \ls \stack X$ to obtain the further base change isomorphisms needed to obtain a monad structure on 
    \[
        p_{2,!}p₁^! \colon \catDModHol{\stack X} → \catDModHol{\stack X}
    \]
    induced by the groupoid $\ls \stack X \rightrightarrows \stack X$ (cf.~Lemma~\ref{lem:d-mod:pre:groupoid_monad_hol}).
    Then $p_{2,!}p₁^! → Δ^!Δ_!$ is an isomorphism of monads and the statement is evident.
    [ToDo: Write this up in more detail.]
\end{proof}

\begin{Ex}
    For any algebraic group $G$ the stack $\B G = \pt/G$ is \goodstack. [ToDo: Add proof.]
\end{Ex}

\begin{Ex}
    The stack $\ps1/\as1$ is not \goodstack.
    In fact a direct computation shows that Theorem~\ref{thm:d-mod:main} does not hold for this stack.
\end{Ex}

\begin{Lem}
    If $\stack X = X/G$ and $\stack Y = Y/H$ are \goodstack, then $\stack X × \stack Y$ is \goodstack.
\end{Lem}

\begin{proof}
    Follows from compatibility of $\tilde\imath^*\bar{p₂}^!j_!$ with $\boxtimes$ and coproducts.
    (Note that $\lsc (\stack X × \stack Y) = \lsc \stack X × \cls \stack Y ∪ \cls \stack X × \lsc \stack Y$.)
\end{proof}

\begin{Lem}
    \label{lem:d-mod:strategy:cover}%
    Let $U$ be a $G$-equivariant open subset of $X$ and assume that $X/G$ is \goodstack.
    Then $U/G$ is \goodstack.
    Conversely, if there exists a $G$-equivariant open cover $U_i$ of $X$ such that all stacks $U_i/G$ are \goodstack, then $X$ is \goodstack.
\end{Lem}

\begin{proof}
    [ToDo: Rewrite for stronger condition]
    Let $\stack U = U/G$ be the open substack.
    Consider the diagram
    \[
        \begin{tikzcd}
            \lsc \stack U \arrow[r, hook] \arrow[d, hook, "α"] & \cls \stack U \arrow[r] \arrow[d, hook, "β"] & \bar{\stack U} \arrow[d, hook, "γ"] & \stack U \arrow[l, hook'] \arrow[d, hook, "δ"] \\
            \lsc \stack X \arrow[r, hook] & \cls \stack X \arrow[r] & \bar{\stack X} & \stack X \arrow[l, hook'] 
        \end{tikzcd}
    \]
    The vertical arrows are open embeddings and all squares are Cartesian (where we use the same compactification of $G$ for $\bar{\stack X}$ and $\bar{\stack U}$).
    Thus
    \begin{equation}
        \label{eq:lem:d-mod:cover}
        \tilde\imath_{U}^*\bar p_{U,2}^!j_{U,!} k_{\stack U} = 
        \tilde\imath_{Z}^*\bar p_{Z,2}^!j_{Z,!} δ^* k_{\stack X} = 
        α^* \tilde\imath_{X}^*\bar p_{X,2}^!j_{X,!} k_{\stack X} = 
        0.
    \end{equation}
    The second statement follows form the fact that vanishing of a sheaf can be tested locally and reading~\eqref{eq:lem:d-mod:cover} backwards.
\end{proof}

[ToDo: Reduction to schemes?]
%Consider the diagram\footnote{Use of Cyrillic letters is probably not the best idea, especially since italic \enquote{и} and \enquote{п} look like \enquote{u} and \enquote{n}. However we will stick with it until I come up with a better notation.}
%\[
%    \begin{tikzcd}
%        \schemelsc \stack{X} \arrow[r, hook, "\tilde\schemei"] \arrow[d, "σ_{\lsc\stack X}"]
%            & \schemecls{\stack X} \arrow[r, "\bar\schemep₂"] \arrow[d, "σ_{\cls{\stack X}}"]
%                & \bar Γ \arrow[d, "σ_{\bar{\stack X}}"]
%                    & Γ \arrow[l, hook', "\schemej"'] \arrow[d, "σ_{\stack X}"]
%        \\
%        \lsc\stack X \arrow[r, hook, "\tilde\imath"]
%        & \cls{\stack X} \arrow[r, "\bar p₂"]
%                & \bar{\stack X}
%                    & \stack X \arrow[l, hook', "j"']
%    \end{tikzcd}
%\]
%where the top row are the obvious smooth covers of the stacks at the bottom and the squares are all Cartesian.
%The vertical arrows are all smooth of the same dimension. 
%Hence,
%\[
%    σ_{\lsc{\stack X}}^* \tilde\imath^*\bar{p}₂^!j_! k_{\stack X} =
%    \tilde{\schemei}^* \bar{\schemep}₂^! \schemej_! σ_{\stack X}^* k_{\stack X} =
%    \tilde{\schemei}^* \bar{\schemep}₂^! \schemej_! k_{Γ}.
%\]
%The functor $σ_{\lsc{\stack X}}^*$ is conservative, which implies the following lemma.
%
%\begin{Lem}
%    The stack $X/G$ is \goodstack\ if and only if $\tilde{\schemei}^* \bar{\schemep}₂^! \schemej_! k_{Γ} = 0$.
%\end{Lem}

\iffalse
\begin{Rem}
    Let $V_i$ be a $G$-equivariant open cover of $\bar G$.
    Then we can consider $Γ$ as a subset of $X × V_i × X$ and take its closure $\bar Γ_i$.
    The sets $\bar Γ_i$ for an open cover cover of $\bar Γ$ and hence the stacks $\stack U_i = \rquot{\bar Γ_i}{G×G}$ form an open cover of $\bar{\stack X}$.
\end{Rem}
\fi%

\begin{Lem}\label{lem:d-mod:strategy:cover-by-relative-compactifications}
    Let $V_i$ be a $G$-stable open cover of $\bar G$.
    Then $\stack X$ is good if and only $\tilde\imath_{\stack Y}^* \bar q_{\stack Y}^! j_! = 0$ for each relative compactification $\bar{\stack X}_{V_i}$ and each $\stack Y → \stack X$
\end{Lem}

[ToDo: Need better notation.]

[ToDo: Proof (obvious).]
