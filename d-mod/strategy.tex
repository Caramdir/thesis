\chapter{The strategy}
\label{ch:d-mod:strategy}

Let $X$ be a scheme of finite type over an algebraically closed field $k$ of characteristic $0$.
Suppose an algebraic group $G$ over $k$ acts on $X$.
Let $\stack X = X/G$ be the corresponding quotient stack.
We are interested in computing the Hochschild cohomology
\[
    \HCoh\bigl(\catDMod{\stack X}\bigr) = \Hom_{\catDMod{\stack X × \stack X}}(Δ_*ω_{\stack X},\, Δ_*ω_{\stack X}),
\]
where $Δ\colon \stack X → \stack X × \stack X$ is the diagonal map.
In this chapter we will outline a strategy for computing the Hochschild cohomology.
We will apply this strategy to the case $G = \Gm^n$ in Chapter~\ref{ch:d-mod:torus}.

\section{The morphism}

Let us for the moment just assume that $\stack X$ is a QCA stack [with schematic diagonal].

The dualizing module $ω_{\stack X}$ is always holonomic. 
Thus we can compute $\dualize Δ_* ω_{\stack X} = Δ_! k_{\stack X}$.
With this we observe that
\begin{align*}
    \HCoh\bigl(\catDMod{\stack X}\bigr) 
    & = \Hom_{\catDMod{\stack X × \stack X}}(Δ_* ω_{\stack X},\, Δ_* ω_{\stack X}) & &\text{(definition)} \\
    & = \Hom_{\catDMod{\stack X × \stack X}}(Δ_! k_{\stack X},\, Δ_! k_{\stack X}) & & \text{(duality)} \\
    & = \Hom_{\catDMod{\stack X × \stack X}}(k_{\stack X},\, Δ^! Δ_! k_{\stack X}) & & \text{(adjunction)} \\
    & = \ΓdR\bigl(Δ^! Δ_! k_{\stack X}\bigr)
\end{align*}
The adjunction morphisms always give sequence of morphisms
\begin{equation}
    \label{eq:d-mod:central_iso_adjunctions}
    p_{1,!}p₂^! k_{\stack X} →
    p_{1,!}p₂^! Δ^! Δ_! k_{\stack X} =
    p_{1,!}p₁^! Δ^! Δ_! k_{\stack X} →
    Δ^! Δ_! k_{\stack X}.
\end{equation}
The composite map
\begin{equation}
    \label{eq:d-mod:central_iso}
    p_{1,!}p₂^!k_{\stack X} \to Δ^! Δ_! k_{\stack X}.
\end{equation}
then induces a morphism
\[
    \ΓdR\bigl(Δ^! Δ_! k_{\stack X}\bigr)
    →
    \HCoh\bigl(\catDMod{\stack X}\bigr).
\]

\begin{Rem}
    In general it is not clear how to make $\ΓdR\bigl(Δ^! Δ_! k_{\stack X}\bigr)$ into an algebra.
\end{Rem}

\section{Understanding Hochschild cohomology via a relative compactification}

In general the morphism~\eqref{eq:d-mod:central_iso} is of course not an isomorphism.
However, if $Δ$ were proper, then it is exactly the base change isomorphism \cite[\RomanNum{III}.4.2.1.3]{GaitsgoryRozenblyum:prelim:StudyInDAG}.

Thus we introduce a relative compactification of $\stack X$ to force the pushforwards to be along proper maps.
Here by a relative compactification of $Δ$ we mean an open embedding $j\colon \stack X \hookrightarrow \bar{\stack X}$ together with a proper map $\bar Δ\colon \bar{\stack X} → \stack X × \stack X$ such that $Δ = \bar Δ ∘ j$.
We will discuss how to construct such a relative compactification for a quotient stack in Section~\ref{sec:d-mod:strategy:compactification}.
For now let us simply assume that it exists.

We let $i\colon \stack X^c \hookrightarrow \bar{\stack X}$ be the closed complement of $\stack X$ in $\bar{\stack X}$.
This gives us a compactified loop space $\cls \stack X$ and the complement $\lsc\stack X$ defined via the following Cartesian squares.
\[
   \begin{tikzcd}
        \cls\stack X \arrow[r, "\bar{p₂}"] \arrow[d, "\bar{p₁}"] & \bar{\stack{X}} \arrow[d, "\bar Δ"] \\
        \stack X \arrow[r, "Δ"] & \stack X × \stack X
    \end{tikzcd}
    \qquad
    \begin{tikzcd}
        \lsc\stack X \arrow[r, "p₂^c"] \arrow[d, "p₁^c"] & \stack{X}^c \arrow[d, "\res{\bar Δ}{\stack X^c}"] \\
        \stack X \arrow[r, "Δ"] & \stack X × \stack X
    \end{tikzcd}
\]
The various inclusions are summarized in the following squares.
\[
    \begin{tikzcd}
        \lsc\stack{X} \arrow[r, "p₂^c"] \arrow[d, hook, "\tilde\imath"] & \stack X^c \arrow[d, hook, "i"]\\
        \cls\stack{X} \arrow[r, "\bar{p₂}"] & \bar{\stack X}
    \end{tikzcd}
    \qquad
    \begin{tikzcd}
        \ls\stack X \arrow[r, "p₂"] \arrow[d, hook, "\tilde\jmath"] & \stack{X} \arrow[d, hook, "j"]\\
        \cls\stack X \arrow[r, "\bar{p₂}"] & \bar{\stack{X}}
    \end{tikzcd}
\]

\begin{Rem}
    The reader might find the following notation guidelines useful:
    We use bars to denote the relative closures, unadorned symbols for the original spaces and morphisms (which are now open subsets of the closure) and the index ${}^c$ for the complements (which are closed subsets).
    Furthermore, we will generally use $\widetilde{\phantom{X}}$ to denote the base change of morphisms to the (compactified) loop space.
\end{Rem}

\begin{Lem}
    \label{lem:d-mod:key}%
    The cone of the morphism~\eqref{eq:d-mod:central_iso},
    \[
         p_{1,!}p₂^!k_{\stack X} \to Δ^! Δ_! k_{\stack X}
    \]
    is given by
    \[
        p^c_{1,!}\tilde\imath^* \bar{p₂}^! j_! k_{\stack X}.
    \]
    In particular, if $\tilde\imath^* \bar{p₂}^! j_! k_{\stack X}$ vanishes, then \eqref{eq:d-mod:central_iso} is an isomorphism.
\end{Lem}

\begin{proof}
    We split the adjunction is \eqref{eq:d-mod:central_iso_adjunctions} into two by using the compositions $Δ = \bar Δ ∘ j$ and $p_i = \bar p_i ∘ \tilde\jmath$.
    Thus the adjunction $p_{1,!}p₂^!→ p_{1,!}p₂^!Δ^!Δ_!$ becomes the sequence
    \[
        p_{1,!}p₂^! →
        p_{1,!}p₂^! j^! j_! →
        p_{1,!}p₂^! j^! \bar Δ^! \bar Δ_! j_!.
    \]
    The equality $p_{1,!}p₂^! Δ^! Δ_! = p_{1,!}p₁^! Δ^! Δ_!$ then becomes
    \[
        p_{1,!} p₂^! j^! \bar Δ^! \bar Δ_! j_! =
        p_{1,!} \tilde \jmath^! \bar p₂^! \bar Δ^! \bar Δ_! j_! =
        p_{1,!} \tilde \jmath^! \bar p₁^! Δ^! \bar Δ_! j_!.
    \]
    Finally the adjunction $p_{1,!}p₁^! Δ^! Δ_! → Δ^! Δ_!$ becomes
    \[
        p_{1,!}\tilde\jmath^! \bar p₁^! Δ^! \bar Δ_! j_! = 
        \bar p_{1,!} \tilde\jmath_! \tilde\jmath^! \bar p₁^! Δ^! \bar Δ_! j_! → 
        \bar p_{1,!} \bar p₁^! Δ^! \bar Δ_! j_! → 
        Δ^! \bar Δ_! j_! = 
        Δ^! Δ_!.
    \]
    Let us apply the same adjunction morphisms in a different order.
    First the inclusions
    \[
        p_{1,!}p₂^!
        \xrightarrow{α}
        p_{1,!}p₂^! j^! j_! 
        =
        \bar p_{1,!} \tilde\jmath_! \tilde\jmath^! \bar p₂^! j_! 
        \xrightarrow{β}
        \bar p_{1,!} \bar p₂^! j_!,
    \]
    and then the actual base change
    \begin{equation}
        \label{eq:lem:d-mod:key:pf:split_morphism_base_change}
        \bar p_{1,!} \bar p₂^! j_!
        →
        \bar p_{1,!} \bar p₂^! \bar Δ^! \bar Δ_! j_!
        =
        \bar p_{1,!} \bar p₁^! Δ^! \bar Δ_! j_!
        →
        Δ^! \bar Δ_! j_!
        =
        Δ^! Δ_!.
    \end{equation}
    We note that the adjunction $\id → j^!j_!$ labeled $α$ is an isomorphism and the composition of the maps in \eqref{eq:lem:d-mod:key:pf:split_morphism_base_change} is exactly the isomorphism of proper base change \cite[\RomanNum{III}.4.2.1.3]{GaitsgoryRozenblyum:prelim:StudyInDAG}.
    Thus the cone of the whole composition composition is exactly the cone of the morphism $β$,
    \[
        \bar p_{1,!} \tilde\imath_* \tilde\imath^* \bar p₂^! j_! = p^c_{1,!} \tilde\imath^* \bar p₂^! j_!.
        \qedhere
    \]
\end{proof}

\iffalse
\begin{Rem}
    An open cover $\{ \stack U_i \}$ of $\bar{\stack X}$ induces an open cover $\{ \widetilde{\stack U}_i \}$ of $\cls \stack X$ and hence also an open cover $\{ \widetilde{\stack U}_i^c \}$ of $\lsc \stack X$.
    It clearly suffices to check that $\tilde\imath^* \bar{p₂}^! j_! k_{\stack X}$ vanishes when restricted to each $\widetilde{\stack U}^c_i$.
    We note that 
    \[
        \res{\bigl(\tilde\imath^* \bar p₂^! j_! k_{\stack X}\bigr)}{\widetilde{\stack U}^c_i} =
        \tilde\imath^{\prime,*} \bar p₂^{\prime,!} j'_! k_{\stack U_i},
    \]
    where the maps marked with primes are the corresponding restricted maps.
\end{Rem}
\fi

\section{How to obtain a relative compactification}
\label{sec:d-mod:strategy:compactification}%

So far we have simply assumed the existence of a relative compactification $\bar Δ$ of the diagonal map $Δ$.
We will now discuss an explicit method for constructing such a compactification.

Constructing a relative compactification of $Δ\colon \stack X → \stack X × \stack X$ is the same as a first constructing a $G × G$-equivariant relative compactification of $(\proj2, a)\colon G × X → X × X$ (where $a\colon G × X → X$ is the action map) and then taking the quotient by the $G × G$ action\footnote{%
    It might be worth mentioning that here $G × G$ acts on $X × G$ by $(s₁,s₂) \cdot (x,g) = (s₁x,\, s₂gs₁^{-1})$.
}.
We let 
\[
    Γ = \bigl\{(g, x, x, gx) ∈ G × X × X × X\bigr\}
\]
be the graph of $(\proj2, a)$.

We pick a $G$-equivariant compactification $\bar G$ of $G$ and let $\bar Γ$ be the closure of $Γ$ in $\bar G × X × X × X$.
We have an open embedding $j$ of $G × X \cong Γ$ into $\bar Γ$ and proper map $f\colon \bar Γ → X × X$ given by projection on the last two factors.
The composition $f ∘ j$ is equal to $(\proj2, a)$.

Instead of viewing $Γ$ as the graph of $(\proj2, a)$ we can drop the third factor and regard it as the graph of the action map, i.e.
\[
    Γ_a = \bigl\{(g, x, gx) ∈ G × X × X\bigr\}.
\]
The closure $\bar Γ_a$ of $Γ_a$ in $\bar G × X × X$ identifies with $\bar Γ$.
Thus for ease of notation we will from now on always set $Γ = Γ_a$ and $\bar Γ = \bar Γ_a$.

\begin{Def}
    Let $\stack X = X/G$. 
    Then with the above construction we set 
    \[
        \bar{\stack X} = \rquot{\bar Γ}{G×G}.
    \]
    We have an open embedding $j\colon X \hookrightarrow \bar{\stack X}$ and a morphism $\bar Δ\colon \bar{\stack X} → \stack X × \stack X$ induced by the map $f$ above, such that $Δ = \bar Δ ∘ j$.
\end{Def}


\section{\Goodstack\ stacks}

In view of Lemma~\ref{lem:d-mod:key}, let us call a quotient stack $\stack X = X/G$ \emph{\goodstack} if there exists a relative compactification $\bar{\stack X}$ such that $\tilde\imath^* \bar{p₂}^! j_! k_{\stack X} = 0$.

\begin{Ex}
    For any algebraic group $G$ the stack $\B G = \pt/G$ is \goodstack.
\end{Ex}

\begin{Ex}
    The stack $\ps1/\as1$ is not \goodstack.
    In fact a direct computation shows that Theorem~\ref{thm:d-mod:main} does not hold for this stack.
\end{Ex}

\begin{Lem}
    If $\stack X = X/G$ and $\stack Y = Y/H$ are \goodstack, then $\stack X × \stack Y$ is \goodstack.
\end{Lem}

\begin{proof}
    Follows from compatibility of $\tilde\imath^*\bar{p₂}^!j_!$ with $\boxtimes$ and coproducts.
    (Note that $\lsc (\stack X × \stack Y) = \lsc \stack X × \cls \stack Y ∪ \cls \stack X × \lsc \stack Y$.)
\end{proof}

\begin{Lem}
    \label{lem:d-mod:cover}%
    Let $U$ be a $G$-equivariant open subset of $X$ and assume that $X/G$ is \goodstack.
    Then $U/G$ is \goodstack.
    Conversely, if there exists a $G$-equivariant open cover $U_i$ of $X$ such that all stacks $U_i/G$ are \goodstack, then $X$ is \goodstack.
\end{Lem}

\begin{proof}
    Let $\stack U = U/G$ be the open substack.
    Consider the diagram
    \[
        \begin{tikzcd}
            \lsc \stack U \arrow[r, hook] \arrow[d, hook, "α"] & \cls \stack U \arrow[r] \arrow[d, hook, "β"] & \bar{\stack U} \arrow[d, hook, "γ"] & \stack U \arrow[l, hook'] \arrow[d, hook, "δ"] \\
            \lsc \stack X \arrow[r, hook] & \cls \stack X \arrow[r] & \bar{\stack X} & \stack X \arrow[l, hook'] 
        \end{tikzcd}
    \]
    The vertical arrows are open embeddings and all squares are Cartesian (where we use the same compactification of $G$ for $\bar{\stack X}$ and $\bar{\stack U}$).
    Thus
    \begin{equation}
        \label{eq:lem:d-mod:cover}
        \tilde\imath_{U}^*\bar p_{U,2}^!j_{U,!} k_{\stack U} = 
        \tilde\imath_{Z}^*\bar p_{Z,2}^!j_{Z,!} δ^* k_{\stack X} = 
        α^* \tilde\imath_{X}^*\bar p_{X,2}^!j_{X,!} k_{\stack X} = 
        0.
    \end{equation}
    The second statement follows form the fact that vanishing of a sheaf can be tested locally and reading~\eqref{eq:lem:d-mod:cover} backwards.
\end{proof}

Consider the diagram\footnote{Use of Cyrillic letters is probably not the best idea, especially since italic \enquote{и} and \enquote{п} look like \enquote{u} and \enquote{n}. However we will stick with it until I come up with a better notation.}
\[
    \begin{tikzcd}
        \schemelsc \stack{X} \arrow[r, hook, "\tilde\schemei"] \arrow[d, "σ_{\lsc\stack X}"]
            & \schemecls{\stack X} \arrow[r, "\bar\schemep₂"] \arrow[d, "σ_{\cls{\stack X}}"]
                & \bar Γ \arrow[d, "σ_{\bar{\stack X}}"]
                    & Γ \arrow[l, hook', "\schemej"'] \arrow[d, "σ_{\stack X}"]
        \\
        \lsc\stack X \arrow[r, hook, "\tilde\imath"]
        & \cls{\stack X} \arrow[r, "\bar p₂"]
                & \bar{\stack X}
                    & \stack X \arrow[l, hook', "j"']
    \end{tikzcd}
\]
where the top row are the obvious smooth covers of the stacks at the bottom and the squares are all Cartesian.
The vertical arrows are all smooth of the same dimension. 
Hence,
\[
    σ_{\lsc{\stack X}}^* \tilde\imath^*\bar{p}₂^!j_! k_{\stack X} =
    \tilde{\schemei}^* \bar{\schemep}₂^! \schemej_! σ_{\stack X}^* k_{\stack X} =
    \tilde{\schemei}^* \bar{\schemep}₂^! \schemej_! k_{Γ}.
\]
The functor $σ_{\lsc{\stack X}}^*$ is conservative, which implies the following lemma.

\begin{Lem}
    The stack $X/G$ is \goodstack\ if and only if $\tilde{\schemei}^* \bar{\schemep}₂^! \schemej_! k_{Γ} = 0$.
\end{Lem}

\iffalse
\begin{Rem}
    Let $V_i$ be a $G$-equivariant open cover of $\bar G$.
    Then we can consider $Γ$ as a subset of $X × V_i × X$ and take its closure $\bar Γ_i$.
    The sets $\bar Γ_i$ for an open cover cover of $\bar Γ$ and hence the stacks $\stack U_i = \rquot{\bar Γ_i}{G×G}$ form an open cover of $\bar{\stack X}$.
\end{Rem}
\fi%

%\section{The algebra structure}
%
%[The idea of this section is taken from the proof of \cite[Proposition~F.?]{ArinkinGaitsgory:arXiv:SingularSupport}.
%The whole section should eventually be expanded to explain the constructions.]
%
%Consider the monads $M = p_{1,!}p₂^!$ (coming from convolution) and $N = Δ^!Δ_!$ (from adjunction).
%The morphism \eqref{eq:central_iso} induces a morphism of monads $M → N$ and hence a morphism of algebras from
%\[
%        \ΓdR\bigl(p_{1,!}(p₂^!(k_{\stack X}))\bigr) =
%        \Hom\bigl(k_{\stack X},\, p_{1,!}(p₂^!(k_{\stack X}))\bigr)
%\]
%to
%\[
%        \HCoh\bigl(\catDMod{\stack X}\bigr) =
%        \Hom\bigl(k_{stack X},\, Δ^!Δ_!k_{\stack X}\bigr).
%\]
%[Actually, $\HCoh\bigl(\catDMod{\stack X}\bigr) = \Hom\bigl(ω_{\stack X},\, Δ^*Δ_*ω_{\stack X}\bigr)$.
%So we also have to apply duality.
%Does this make the morphism an anti-isomorphism?]
