\chapter{Proof of the main theorem}

For this section let $G = T = \Gm^n$.

\begin{Thm}
    \label{thm:main:repeated}%
    If the action of $T$ on $X$ is locally linear, then the morphism \eqref{eq:central_iso} induces an isomorphism of algebras
    \[
        \HCoh\bigl(\catDMod{\stack X}\bigr) \cong \ΓdR\bigl(p_{1,!}(p₂^!(k_{\stack X}))\bigr)
    \]
\end{Thm}

Before we can prove this theorem we need to do some preparatory work.

Let $S$ be closed subgroup of $G = T$ and let $X^S$ be $S$-fixed points of $X$.
Since $T$ is Abelian (and hence $S$ a normal subgroup), $X^S$ is a $T$-stable (closed) subscheme of $X$.
Hence $X^S/T$ is a closed substack of $\stack X$.
Let $\bar S$ be the closure of $S$ in $\bar T$ and consider the space
\[
    \schemecls[S] \stack X =
    \{
        (x, t₁, t₂, t₁x) ∈ X^S × T × \bar T × X^S : (x, t₂, t₁x) ∈ \bar Γ \text{ and } t₂ ∈ t₁\bar S
    \}.
\]
We have a Cartesian square of closed embeddings
\[
    \begin{tikzcd}
        \schemelsc[S] \stack X \arrow[r, hook, "i_S^c"] \arrow[d, hook, "\tilde\schemei^S"] & \schemelsc\stack X \arrow[d, hook, "\tilde\schemei"] \\
        \schemecls[S] \stack X \arrow[r, hook, "i_S"]   & \schemecls\stack X 
    \end{tikzcd}
\]
Finally let $\Stab\stack X$ be the set of all closed subgroups of $T$ that are stabilizers of points of $X$, i.e.
\[
    \Stab\stack X = \{ T_x : x ∈ X \}.
\]

\begin{Lem}
    If $X$ is affine then $\Stab\stack X$ is finite.
\end{Lem}

\begin{proof}
    $X$ can be embedded $T$-equivariantly into some $\as m$ with a linear $T$-action and for $\as m/T$ the statement is easy to see.
\end{proof}

\begin{Lem}
    The subspaces $\schemecls[S] \stack X$ for $S ∈ \Stab\stack X$ cover $\schemecls\stack X$.
\end{Lem}

\begin{Lem}
    Consider the spaces
    \[
        \schemels[S] \stack X = 
        \{
            (x, t₁, t₂, t₁x) ∈ X^S × T × \bar T × X^S : (x, t₂, t₁x) ∈ \bar Γ \text{ and } t₂ ∈ t₁S
        \}.
    \]
    The closure of $\schemels[S] \stack X$ in $\schemecls \stack X$ is exactly $\schemecls[S] \stack X$.
    It is easy to see that the subspaces $\schemels[S] \stack X$ for $S ∈ \Stab \stack X$ cover $\schemels[S]$.
    Now the statement follows from the fact that the closure of a finite union is the union of the individual closures.
\end{Lem}

The following theorem is a generalization of \cite[Theorem~2.?]{DrinfeldGaitsgory:2014:OnATheoremOfBraden}.
It can be proved in an analogous way to that theorem, see \texttt{torus-actions.pdf}.
However it would be nicer if there was some easy induction with \cite[Theorem~2.?]{DrinfeldGaitsgory:2014:OnATheoremOfBraden} as the base case.

\begin{Thm}
    \label{thm:torus_graph_closure}
    If $X$ is affine then the morphism $q\colon \bar Γ → \bar T$ is smooth.
\end{Thm}

\begin{Lem}\label{lem:key_for_stablizier}%
    Let $X$ be affine.
    For any $S ∈ \Stab\stack X$ we have
    \[
        \tilde\schemei^{S,*} i_S^! \schemep₂^! \schemej_! k_Γ = 0.
    \]
\end{Lem}

\begin{proof}
    The scheme $\schemelsc[S] \stack X$ is of the form
    \[
        \schemelsc[S] \stack X  = 
        \{
            (x, t₁, t₂, t₁x) ∈ X^S × T × \bar T × X^S : (x, t₂, t₁x) ∈ \bar Γ \text{ and } t₂ ∈ t₁(\bar S \setminus S)
        \}.
    \]
    If $S = \bar S$ the statement is trivially true.
    Otherwise the schemes $\bar S \setminus S$ is the union of hyperplanes $H_i$ of $\bar S$.
    It suffices to prove the statement when further restricting to 
    \[
        \{
            (x, t₁, t₂, t₁x) ∈ X^S × T × \bar T × X^S : (x, t₂, t₁x) ∈ \bar Γ \text{ and } t₂ ∈ t₁H_i
        \}
    \]
    for all $i$. 
    Let $H$ be one such hyperplane.
    We will assume that $H$ is contained in the closure of the connected component of $1 ∈ S$. 
    The proof for $H$ is a different component is the same, up to a shift by an element of $T$.
    Let $\schemei_H$ be the inclusion of
    \[
        Z = \{
            (x, t₁, t₂, t₁x) ∈ X^S × T × \bar T × X^S : (x, t₂, t₁x) ∈ \bar Γ \text{ and } t₂ ∈ t₁H
        \}
    \]
    into $\schemecls[S] \stack X$.
    We want to compute
    \[
        \schemei_H^{*} i_S^! \schemep₂^! \schemej_! k_Γ.
    \]

    Consider the Cartesian square
    \[
        \begin{tikzcd}
            Γ \arrow[r, hook, "\schemej"] \arrow[d, "\tilde q"] & \bar Γ \arrow[d, "q"] \\
            T \arrow[r, hook, "\schemej_T"] & \bar T
        \end{tikzcd}
    \]
    By Theorem~\ref{thm:torus_graph_closure}, the vertical arrows are smooth.
    Thus,
    \[
        \schemej_! k_Γ = 
        \schemej_! \tilde q^* k_T = 
        q^* \schemej_{T,!} k_T = 
        q^! \schemej_{T,!} k_T.
    \]
    Let $f = q ∘ \schemep₂ ∘ i_S$, so that
    \[
        \schemei_H^{*} i_S^! \schemep₂^! \schemej_! k_Γ = 
        \schemei_H^{*} i_S^! \schemep₂^! q^! \schemej_{T,!} k_T = 
        \schemei_H^{*} f^! \schemej_{T,!} k_T.
    \]
    
    Write $T \cong T₁ × S$ for some subgroup $T₁$ of $T$.
    Then $\bar T = \bar T₁ \bar S$.
    Let $H' = \bar T₁ H$.
    We note that $T ∩ H' = \emptyset$.

    We chose an action $μ$ of $\Gm$ on $\bar S$ that contracts $\bar S$ onto $H$.
    This induces an action of $\Gm$ on $\bar T = \bar T₁ \bar S$ by $u \cdot ts = tμ(u,s)$, contracting $\bar T$ onto $H'$.
    The D-module $k_T$ is monodromic with respect to this action.
    Let $π \colon \bar T → H'$ and $\tilde π\colon \schemecls[S] \stack X → Z$ be the induced contraction morphisms.
    
    By the contraction principle \cite[Theorem~?]{DrinfeldGaitsgory:2014:OnATheoremOfBraden}, $\schemei_H^* = \tilde π_\renstar$ on $\Gm$-monodromic D-modules.
    Thus,
    \[
        \schemei_H^{*} f^! \schemej_{T,!} k_T = 
        \tilde π_\renstar f^! \schemej_{T,!} k_T.
    \]
    The square
    \[
        \begin{tikzcd}
            \schemecls[S] \arrow[r, "\tilde π"] \arrow[d, "f"] & Z \arrow[d, "f_Z"] \\
            \bar T \arrow[r, "π"] & H'
        \end{tikzcd}
    \]
    is Cartesian.
    Base change yields
    \[
        \tilde π_\renstar f^! \schemej_{T,!} k_T = 
        f_Z^! π_\renstar \schemej_{T,!} k_T.
    \]
    Finally, let $i_{H'}\colon H' \hookrightarrow \bar T$ be the inclusion.
    Applying the contraction principle again we obtain
    \[
        f_Z^! π_\renstar \schemej_{T,!} k_T =
        f_Z^! i_{H'}^* \schemej_{T,!} k_T =
        0.
        \qedhere
    \]
\end{proof}

\begin{proof}[Proof of Theorem~\ref{thm:main:repeated}]
    We use Lemma~\ref{lem:key}.
    By Lemma~\ref{lem:cover} we can assume that $X$ is affine.
    Now the statement follows from an inductive argument using Lemma~\ref{lem:key_for_stablizier}.
\end{proof}
